The Gebührenordnung für Ärzte (GOÄ) is the coding catalog used for billing medical services covered by private health insurance.

The GOÄ has the reputation of being highly outdated\cite{heller2015goa}.
The current version was defined in 1982, last update took place in 1996.
Not only the lack of modern procedures in the catalog and as well as ambiguities and interpretation difficulties presents issues for healthcare providers.
Since 1996 the consumer prices have increased by more than 50\%, inflation has effectively reduced professional fees by more than 28\% \cite{schmitzgoa}.

The only thing practitioners can do to manage this discrepancy on their own is to create correct billings that compensate them for all services provided.
Fortunately, the GOÄ has some characteristics that allow for flexible billing price optimization.

Given this growing financial discrepancy healthcare providers find themselves in, applying those optimization techniques correctly plays a more and more important role in modern medical billing.
In the following sections we take a closer look at GOÄ characteristics that are relevant for optimization in private billing.

\section{Billing Multipliers}\label{sec:billing-multipliers}
Unlike statutory billing, the GOÄ allows for the application of billing multipliers (Steigerungsfaktoren).
Billing multipliers are an important tool to tailor private billing and to take into account unexpected difficulties and increased time expenditure of treatments\cite{walter2008abrechnung}.

They are assignable to individual billing codes in a treatment billing and increase the price for the respective position by multiplication of the GOÄ base fee and the multiplier value.

\[
    \mathrm{final\ price} = \mathrm{base\ code\ points} \times \mathrm{billing\ multiplier} \times \mathrm{point\ value}
\]


These multipliers can vary depending on the complexity, time expenditure, or special circumstances of the treatment.

However, billing position costs cannot be freely increased.
Each GöA code specifies ranges for valid multiplier values.
Below are commonly used multipliers in the GOÄ\cite[]{bruck1998kommentar}:

\begin{itemize}
    \item \textbf{Simple Rate (1.0 Multiplier)}: This is the base rate for most medical services and corresponds to the fee listed in the GOÄ catalog.
    \item \textbf{1.15 to 1.8 Multiplier}: This range is used for general medical services that go beyond the base rate.
    The exact multiplier can vary depending on the complexity and time required for the treatment.
    \item \textbf{2.3 to 3.5 Multiplier}: This range is used for specialized or particularly complex services.
    The application of these multipliers typically requires a detailed justification.
    \item \textbf{Higher multipliers}: In exceptional cases, even higher multipliers can be applied.
    However, this must be particularly justified and clarified on a case-by-case basis with the private health insurance company.
\end{itemize}

GOÄ codes from section A are an exception, as only the maximum multiplier 2.5 is applicable\cite[]{hermanns2011gebuhrenordnung}.


\section{Billing Justifications}\label{sec:billing-justifications}

Applying a specific multiplier to a billing position always requires a justification.
multiplier justifications, however, are not standardized by the GOÄ\cite[]{bruck1998kommentar}.
There is no official collection of applicable and excepted justifications for multipliers.
They are essentially free texts that appear next to the respective multiplier in the billing.


The billing framework should define a way of expressing multiplier justifications as rules.
As part of private billing, the system automatically checks for factor justifications that can be applied, i.e. multiplier rules that hold.
This way the system makes sure that the user is notified about all possible multiplier optimizations and is also able to back up the multipliers with the relevant information in the structured data.

Part of the research of this work is to collect commonly used billing justifications from experts.
Additionally, it is also important to understand what the relevant data is that multiplier justification relies on.
If necessary, the existing code base of Avelios needs to be extended in such a way that it collects all the data necessary needed to justify multipliers.

The results of this research will be discussed in chapter\ldots


\section{Surcharges}\label{sec:surcharges}
Surcharges are additional billing codes that can be added to the base charge.
Serving a similar purpose as billing multipliers\ref{sec:billing-justifications}, they don't directly refer to the services provided by the practitioner but rather describe special conditions that were present during the treatment\cite{walter2008abrechnung}.
Unlike billing multipliers they are not attached to another billing position.
Instead, they are a standalone billing position that, however, can only be billed with a multiplier of \mete{1.0}.
For example, the surcharge \mete{GOÄ C} can be billed as an additional billing position for treatments carried out between 10pm and 6am.
One must take into account that there are strict exclusion rules that limit the application of surcharges\cite{kommentar2012zuschlage}.
Surcharge \mete{A} cannot be billed with \mete{B}, \mete{C} or \mete{D}.
Surcharges \mete{A}, \mete{B}, \mete{C}, \mete{D}, \mete{K1} cannot be billed in combination with \mete{E}, \mete{F}, \mete{G}, \mete{K2}\cite{kommentar2012zuschlage}.
