The Gebührenordnung für Ärzte (GOÄ) is the coding catalog used for billing medical services covered by private health insurance.

\section{Weight factors}\label{sec:weight-factors}

Unlike statutory billing, the GOÄ allows for the application of multipliers (Steigerungsfaktoren) to make the billing of medical services more flexible.
The primary purpose of multipliers is to offer a flexible billing structure that can be tailored to the specific circumstances of each medical case.
For instance, a simple consultation may be billed at the base rate, while a complex surgical procedure may warrant a higher multiplier to reflect the increased amount of time required.
These weight factors can vary depending on the complexity, time expenditure, or special circumstances of the treatment.

However, billing position costs cannot be freely increased.
Each GöA code specifies the applicable weight factor values.
Below are commonly used weight factors in the GOÄ\cite[]{bruck1998kommentar}:

\begin{itemize}
    \item \textbf{Simple Rate (1.0 Multiplier)}: This is the base rate for most medical services and corresponds to the fee listed in the GOÄ catalog.
    \item \textbf{1.15 to 1.8 Multiplier}: This range is used for general medical services that go beyond the base rate.
    The exact multiplier can vary depending on the complexity and time required for the treatment.
    \item \textbf{2.3 to 3.5 Multiplier}: This range is used for specialized or particularly complex services.
    The application of these weight factors typically requires a detailed justification.
    \item \textbf{Higher weight factors}: In exceptional cases, even higher weight factors can be applied.
    However, this must be particularly justified and clarified on a case-by-case basis with the private health insurance company.
\end{itemize}

GOÄ codes from section A are an exception, as only the maximum weight factor 2.5 is applicable\cite[]{hermanns2011gebuhrenordnung}.

\section{Billing Justifications}\label{sec:billing-justifications}

Applying a specific weight factor to a billing position always requires a justification.
Weight factor justifications, however, are not standardized by the GOÄ\cite[]{bruck1998kommentar}.
There is no official collection of applicable and excepted justifications for weight factors.
They are essentially free texts that appear next to the respective weight factor in the billing.


The billing framework should define a way of expressing weight factor justifications as rules.
As part of private billing, the system automatically checks for factor justifications that can be applied, i.e. weight factor rules that hold.
This way the system makes sure that the user is notified about all possible weight factor optimizations and is also able to back up the weight factors with the relevant information in the structured data.

Part of the research of this work is to collect commonly used billing justifications from experts.
Additionally, it is also important to understand what the relevant data is that weight factor justification relies on.
If necessary, the existing code base of Avelios needs to be extended in such a way that it collects all the data necessary needed to justify weight factors.

The results of this research will be discussed in chapter\ldots



