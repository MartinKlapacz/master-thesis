The Einheitlicher Bewertungsmaßstab (EBM) is a standardized coding catalog used in Germany for billing outpatient services for patients with statutory insurances.
The EBM is maintained by the Kassenärztliche Bundesvereinigung (KBV), the Federal Association of Statutory Health Insurance Physicians\cite[]{hermanns2015ebm}.

As mentioned earlier the EBM is more expressive than the GOÄ.
In fact the KBV maintains the EBM as a downloadable and well documented dataset in XML format which stores a lot of meta information for each rule.
Large parts of the meta information describe exclusion rules and conditions that decide if a rule is applicable or not.

This information is stored as structured data and can be loaded into the system where it can be used to implement the EBM code derivation pipeline.

The following information is relevant for rule derivation
\section{Service Area List}\label{sec:service-area-list}
Unlike the GOÄ each EBM code contains a list of pairs where one entry is a medical field and the second entry a physician type.
Only attending physicians that match to one of the pairs are authorized to bill the respective code.

For example, the EBM code


\section{Obligate Service Conditions}\label{sec:obligate-service-conditions}
Most EBM codes description have a section with obligate services.
It is a HTML content that precisely describes which exact services need to be provided so the code can be billed.
According to the EBM specification, HTML content describing the obligate services can be considered a logical expression.
For example the code 16232 "Diagnostik und/oder Behandlung von Erkrankungen der Wirbelsäule bei Jugendlichen und Erwachsenen" has the following obligate service description:


\begin{itemize}
    \item Diagnosis and/or treatment of spinal diseases
\end{itemize}
and/or
\begin{itemize}
    \item Segmental functional diagnostics and differential diagnosis,
    \item At least two personal doctor-patient contacts in the treatment case,
\end{itemize}

Obligate service descriptions make use of logical OR and AND expressions.
Items in a bullet point list are part of a AND expression, whereas the term "and/or" means the logical OR.

The current means EBM version contains 3xxx codes, but only 1855 distinct obligate services.
The latter are reused across many services. For exaple the most prominent obligate service
