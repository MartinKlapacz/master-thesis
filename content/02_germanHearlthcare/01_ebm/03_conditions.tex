\section{Conditions}\label{sec:conditions}


Each EBM code is represented by a EBM code record in the EBM dataset.
The KBV maintains an extensive xml schema documentation for the EBM data model.


Part of this work is to understand and identify all conditions that determine under which circumstances a code can be billed or not.
For automated derivation of applicable EBM codes these conditions need to be automatically checked.
In this section we go through all identified relevant EBM conditions, that the billing framework need to check automatically.


\subsection{Service Area List}


\subsection{Code Justification List}
Some EBM codes can have a code justification list.
A justification is one of the following:
\begin{itemize}
    \item An OPS code identifying a procedure conducted in this treatment.
    \item An ICD10 code of a diagnosis given to the patient during this treatment
    \item Another EBM code that is part of this billing
\end{itemize}
A code justification list condition holds if and only if it is empty or at least one code of the three justification categories is present.
Localization specific OPS codes are combined with a localization identifiers (left, right or both sides).
For example the EBM code \meco{31658} has the ops number \meco{5-184.1} in combination with the \meco{B} localization identifier in its code justification list.
For example, \meco{OPS 5-184.1} is listed in combination with the \meco{B} localization identifier and is thus a possible justification code for \meco{EBM 31658}.
This means, applying the billing code \mete{Postoperative Behandlung Hals-Nasen-Ohren VI/2a} (\meco{31658}) i.e. \mete{Postoperative treatment of ear, nose, and throat (ENT) VI/2a} to a \mete{plastic correction of protruding ears through excision of soft tissues} is only possible if both ears were operated on.

The idea behind EBM codes being justifications for other EBM codes is the following.
There are EBM codes with a very general description.

\subsection{Amount Condition}

Another class of conditions are amount conditions.
EBM codes can be limited in how often the code can be billed in a specific time context.
Generally, the KBV defines the following time contexts that are relevant for the EBM:
\begin{itemize}
    \item
\end{itemize}


\subsection{Exclusion List}

\subsection{Administrative Gender}

\subsection{Form Type List}

\subsection{Obligate Service Content}

\subsection{Additional Information}

Unlike the GOÄ each EBM code contains a list of pairs where one entry is a medical field and the second entry a physician type.
Only attending physicians that match to one of the pairs are authorized to bill the respective code.


788200