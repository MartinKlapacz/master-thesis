\section{Billing Catalogs}
In the German healthcare system, billing is a critical component that ensures the smooth functioning of medical services.
Patients in Germany are generally covered by either statutory health insurance or private insurance, and each type of insurance
utilizes a different coding catalog for billing purposes.
The most important coding catalogs are the Einheitlicher Bewertungsmaßstab (EBM) for statutory insurance and the Gebührenordnung für Ärzte (GOÄ) for private insurance.
Both use a point system to quantify the value of medical services, which is used for inflation adjustment.
These points are then converted into monetary amounts for billing.
For example the EBM code 34221 "Photographs of parts of the spine" is billed with 140 points, which translates to 16,09€.

