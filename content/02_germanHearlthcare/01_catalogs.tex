\section{Billing Catalogs}
In the German healthcare system, billing is a critical component that ensures the smooth functioning of medical services.
Patients in Germany are generally covered by either statutory health insurance or private insurance, and each type of insurance
utilizes a different coding catalog for billing purposes.
The most important coding catalogs are the Einheitlicher Bewertungsmaßstab (EBM) for statutory insurance and the Gebührenordnung für Ärzte (GOÄ) for private insurance.
Both use a point system to quantify the value of medical services, which is used for inflation adjustment.
These points are then converted into monetary amounts for billing.
For example the EBM code 34221 "Photographs of parts of the spine" is billed with 140 points, which translates to 16,09€.

\subsection{Gebührenordnung für Ärzte (GOÄ)}
The Gebührenordnung für Ärzte (GOÄ) is the coding catalog used for billing medical services covered by private health insurance.

\subsubsection{Structure}
It consists of approximately 2,400 individual billing codes that are organized into 16 distinct sections\cite[]{hermanns2011gebuhrenordnung}.

\begin{itemize}
    \item \textbf{General Provisions}: Outlines the scope and applicability of the GOÄ.
    \item \textbf{Billing Guidelines}: Provides instructions on how to bill for multiple services.
    \item \textbf{Fee Multipliers}: Explains the conditions under which fee multipliers can be applied.
    \item \textbf{Special Cases}: Addresses billing for complex or rare medical procedures.
    \item \textbf{Emergency Services}: Guidelines for billing emergency medical services.
    \item \textbf{Consultations}: Rules for billing consultation services.
    \item \textbf{Diagnostics}: Guidelines for billing diagnostic tests and procedures.
    \item \textbf{Therapeutic Services}: Rules for billing therapeutic procedures.
    \item \textbf{Surgical Procedures}: Guidelines for billing surgical operations.
    \item \textbf{Hospital Stays}: Rules for billing services related to hospitalization.
    \item \textbf{Preventive Services}: Guidelines for billing preventive healthcare services.
    \item \textbf{Additional Services}: Addresses billing for additional services not covered in other sections.
    \item \textbf{Administrative Fees}: Rules for billing administrative tasks.
    \item \textbf{Miscellaneous}: Covers any other billing scenarios not addressed in the previous paragraphs.
\end{itemize}
Sections can also be organized into different subsections.
This is relevant because often there are free text assigned to sections as well as subsections.
These texts specify rules and conditions that the billing framework must adhere to.

Part of the official GOÄ are also the following 15 paragraphs\cite[]{hermanns2011gebuhrenordnung}:
\begin{itemize}
    \item \textbf{General Provisions}: Outlines the scope and applicability of the GOÄ.
    \item \textbf{Billing Guidelines}: Provides instructions on how to bill for multiple services.
    \item \textbf{Fee Multipliers}: Explains the conditions under which fee multipliers can be applied.
    \item \textbf{Special Cases}: Addresses billing for complex or rare medical procedures.
    \item \textbf{Emergency Services}: Guidelines for billing emergency medical services.
    \item \textbf{Consultations}: Rules for billing consultation services.
    \item \textbf{Diagnostics}: Guidelines for billing diagnostic tests and procedures.
    \item \textbf{Therapeutic Services}: Rules for billing therapeutic procedures.
    \item \textbf{Surgical Procedures}: Guidelines for billing surgical operations.
    \item \textbf{Hospital Stays}: Rules for billing services related to hospitalization.
    \item \textbf{Preventive Services}: Guidelines for billing preventive healthcare services.
    \item \textbf{Additional Services}: Addresses billing for additional services not covered in other sections.
    \item \textbf{Administrative Fees}: Rules for billing administrative tasks.
    \item \textbf{Miscellaneous}: Covers any other billing scenarios not addressed in the previous paragraphs.
\end{itemize}


The GOÄ is designed in such a way that a typical GOÄ code has less meta information and is less restrictive compared to codes in the EBM.
This makes the GOÄ more flexible during billing than the EBM.



The fact that the GÖÄ does not provide much meta information per code is not very helpful when building a system to automatically derive codes from treatment documentation.
Which is indeed very helpful are the titles of the billing codes.
With the necessary domain specific knowledge and a rule language which is expressive enough one can implement automatic rule derivation per code.
The EBM, however, provides enough information to enable a more sophisticated approach.

\subsubsection{Weight factors}
Unlike statutory billing, the GOÄ allows for the application of weight factors (Steigerungsfaktoren) to make the billing of medical services more flexible.
The primary purpose of Steigerungsfaktoren is to offer a flexible billing structure that can be tailored to the specific circumstances of each medical case.
For instance, a simple consultation may be billed at the base rate, while a complex surgical procedure may warrant a higher multiplier to reflect the increased amount of time required.
These weight factors can vary depending on the complexity, time expenditure, or special circumstances of the treatment.

However, billing position costs cannot be freely increased.
Each GöA code specifies the applicable weight factor values.
Below are commonly used weight factors in the GOÄ\cite[]{bruck1998kommentar}:

\begin{itemize}
    \item \textbf{Simple Rate (1.0 Multiplier)}: This is the base rate for most medical services and corresponds to the fee listed in the GOÄ catalog.
    \item \textbf{1.15 to 1.8 Multiplier}: This range is used for general medical services that go beyond the base rate.
    The exact multiplier can vary depending on the complexity and time required for the treatment.
    \item \textbf{2.3 to 3.5 Multiplier}: This range is used for specialized or particularly complex services.
    The application of these weight factors typically requires a detailed justification.
    \item \textbf{Higher weight factors}: In exceptional cases, even higher weight factors can be applied.
    However, this must be particularly justified and clarified on a case-by-case basis with the private health insurance company.
\end{itemize}

GOÄ codes from section A are an exception, as only the maximum weight factor 2.5 is applicable\cite[]{hermanns2011gebuhrenordnung}.

\subsubsection{Billing Justifications}

Applying a specific weight factor to a billing position always requires a justification.
Weight factor justifications, however, are not standardized by the GOÄ\cite[]{bruck1998kommentar}.
There is no official collection of applicable and excepted justifications for weight factors.
They are essentially free texts that appear next to the respective weight factor in the billing.


The billing framework should define a way of expressing weight factor justifications as rules.
As part of private billing, the system automatically checks for factor justifications that can be applied, i.e. weight factor rules that hold.
This way the system makes sure that the user is notified about all possible weight factor optimizations and is also able to back up the weight factors with the relevant information in the structured data.

Part of the research of this work is to collect commonly used billing justifications from experts.
Additionally, it is also important to understand what the relevant data is that weight factor justification relies on.
If necessary, the existing code base of Avelios needs to be extended in such a way that it collects all the data necessary needed to justify weight factors.

The results of this research will be discussed in chapter\ldots

\subsubsection{Formal Billing Rules}



A GOÄ code is always assigned to a service area, i.e. the chapter in the catalog, a number of points and excluded codes.
Codes which are too similar to each other cannot be billed at the same time.

\subsubsection{Language semantics}



\subsection{Einheitlicher Bewertungsmaßstab (EBM)}
The Einheitlicher Bewertungsmaßstab (EBM) is a standardized coding catalog used in Germany for billing outpatient services for patients with statutory insurances.
The EBM is maintained by the Kassenärztliche Bundesvereinigung (KBV), the Federal Association of Statutory Health Insurance Physicians.

As mentioned earlier the EBM is way more expressive than the GOÄ.
In fact the KBV maintains the EBM as a downloadable and well documented dataset in XML format which stores a lot of meta information for each rule.
Large parts of the meta information describes exclusion rules and conditions which decide wether the code can be billed or not.

\subsubsection{Static Billing Code Conditions}


\subsubsection{Obligate Service Conditions}
Most EBM codes description have a section with obligate services.
It is a HTML content that precisely describes which exact services need to be provided so the code can be billed.
According to the EBM specification, HTML content describing the obligate services can be considered a logical expression.
For example the code 16232 "Diagnostik und/oder Behandlung von Erkrankungen der Wirbelsäule bei Jugendlichen und Erwachsenen" has the following obligate service description:


\begin{itemize}
    \item Diagnosis and/or treatment of spinal diseases
\end{itemize}
and/or
\begin{itemize}
    \item Segmental functional diagnostics and differential diagnosis,
    \item At least two personal doctor-patient contacts in the treatment case,
\end{itemize}

Obligate service descriptions make use of logical OR and AND expressions.
Items in a bullet point list are part of a AND expression, whereas the term "and/or" means the logical OR.

The current means EBM version contains 3xxx codes, but only 1855 distinct obligate services.
The latter are reused across many services. For exaple the most prominent obligate service
% todo: list obligate service statistics

Instead of implementing rules for specific content


\