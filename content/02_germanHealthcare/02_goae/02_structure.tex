\subsubsection{Structure}
It consists of approximately 2,400 individual billing codes that are organized into 16 distinct sections\cite[]{hermanns2011gebuhrenordnung}.

\begin{itemize}
    \item \textbf{General Provisions}: Outlines the scope and applicability of the GOÄ.
    \item \textbf{Billing Guidelines}: Provides instructions on how to bill for multiple services.
    \item \textbf{Fee Multipliers}: Explains the conditions under which fee multipliers can be applied.
    \item \textbf{Special Cases}: Addresses billing for complex or rare medical procedures.
    \item \textbf{Emergency Services}: Guidelines for billing emergency medical services.
    \item \textbf{Consultations}: Rules for billing consultation services.
    \item \textbf{Diagnostics}: Guidelines for billing diagnostic tests and procedures.
    \item \textbf{Therapeutic Services}: Rules for billing therapeutic procedures.
    \item \textbf{Surgical Procedures}: Guidelines for billing surgical operations.
    \item \textbf{Hospital Stays}: Rules for billing services related to hospitalization.
    \item \textbf{Preventive Services}: Guidelines for billing preventive healthcare services.
    \item \textbf{Additional Services}: Addresses billing for additional services not covered in other sections.
    \item \textbf{Administrative Fees}: Rules for billing administrative tasks.
    \item \textbf{Miscellaneous}: Covers any other billing scenarios not addressed in the previous paragraphs.
\end{itemize}
Sections can also be organized into different subsections.
This is relevant because often there are free text assigned to sections as well as subsections.
These texts specify rules and conditions that the billing framework must adhere to.

Part of the official GOÄ are also the following 15 paragraphs\cite[]{hermanns2011gebuhrenordnung}:
\begin{itemize}
    \item \textbf{General Provisions}: Outlines the scope and applicability of the GOÄ.
    \item \textbf{Billing Guidelines}: Provides instructions on how to bill for multiple services.
    \item \textbf{Fee Multipliers}: Explains the conditions under which fee multipliers can be applied.
    \item \textbf{Special Cases}: Addresses billing for complex or rare medical procedures.
    \item \textbf{Emergency Services}: Guidelines for billing emergency medical services.
    \item \textbf{Consultations}: Rules for billing consultation services.
    \item \textbf{Diagnostics}: Guidelines for billing diagnostic tests and procedures.
    \item \textbf{Therapeutic Services}: Rules for billing therapeutic procedures.
    \item \textbf{Surgical Procedures}: Guidelines for billing surgical operations.
    \item \textbf{Hospital Stays}: Rules for billing services related to hospitalization.
    \item \textbf{Preventive Services}: Guidelines for billing preventive healthcare services.
    \item \textbf{Additional Services}: Addresses billing for additional services not covered in other sections.
    \item \textbf{Administrative Fees}: Rules for billing administrative tasks.
    \item \textbf{Miscellaneous}: Covers any other billing scenarios not addressed in the previous paragraphs.
\end{itemize}
