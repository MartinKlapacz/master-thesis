\section{Conclusion}\label{sec:conclusion}
The objective of this work was to create a billing optimization framework, powered by a domain-specific rule language that allows the creation of optimized and correct billings.
The primary task of my research was to design the rule language and implement a corresponding framework.
This involved synthesizing knowledge from expert interviews with billing professionals,
participation in a GOÄ seminar offered by the PVS Forum, and extensive research into medical billing catalogs and documents provided by \AV.

The research resulted in 15 identified information sources and more than 39 language condition types.
The treatment simulations have shown mostly correct billing generations but also revealed limitations rooted in missing and untracked information sources.
Section \ref{sec:discussion} discusses and analyzes those limitations in great detail.

One of the selling points of the \AV system is the high emphasis on a systematic and complete collection of structured medical data oriented to medical documentation standards.
In this work, we have learned that complete automatic billing generation already works very well with a rich structured data schema aligned with conventional medical documentation.
However, this does not perfectly cover all requirements for a full automatic billing generation system.
Treatment documentation needs to make the transition from a conventional to medical- and billing-oriented treatment documentation.
Medical professionals must design documentation to cover the classic medical documentation but also to track billing-relevant data.
As a result, this will improve the overall patient documentation as it becomes more detailed, but also allow automatic billing generation systems to perform better.
This thesis highlights the crucial role of structured data collection in revolutionizing medical applications.
Detailed and systematic data gathering can truly accelerate digitization of the medical field and revolutionize automated medical billing.
