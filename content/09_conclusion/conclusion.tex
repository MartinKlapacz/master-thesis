\section{Conclusion}\label{sec:conclusion}
Summary of Research:
The main objective was to create a billing framework powered by a domain specific rule language that allows the creation of optimized and correct billings.
The primary research task was to design the rule language.
Designing the rule language included multiple expert interviews, teilnahme am PVS seminar and own research in medical billing catalogs to understand what type of information billing codes actually rely on.
based on this collected knowledge I designed the language and its features.

Key Findings and Contributions:
I identified 17 Information sources and more than 40 language conditions types.


Emphasize how the development of your JSON-based domain-specific rule language and its integration into a Java Spring Boot microservice addresses existing challenges in medical billing.

Mention the successful application of your framework with real-world scenarios from \AV.

Medical documentations to a billing oriented orientation
Clinic information systems need to make the transition from pure medical documentation to a billing-oriented medical orientation.


Limitations and Challenges:
Acknowledge the limitations encountered during your research.
This could include issues with data availability in Avelios Medical's system and the initial challenges faced by billing experts in adapting to the rule language.

Implications of the Research:
Discuss the broader implications of your work.
Detail how your framework contributes to the field of medical billing and what benefits it brings to healthcare providers and billing professionals.

Brief Nod to Future Work:
Without going into detail (since it's covered in a separate section), you can briefly mention that there are avenues for future research that could further enhance and expand the framework's capabilities.

Final Thoughts:
Conclude with reflective thoughts on the research process, its significance in the field, and its potential impact on the future of medical billing systems.
