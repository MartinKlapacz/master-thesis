\section{Experimental Rule}\label{sec:experimental-rule-base}
Obviously, in this work we cannot cover all rules that would be part of an actual rule base.
However, the first version of the billing system is highly


\subsection{Special GOÄ Consultation Code Rules}\label{subsec:special-consultation-code-rules}

\paragraph{\goa{1: Advice – also via telephone}}\label{par:goa-1}
\goa{1} is one of the most commonly used codes in the GOÄ.
It is applicable for a conversation between patient and doctor at the beginning of the visit that takes at most 10 minutes.
This type of conversation happens typically as part of the anamnesis.

Unfortunately, this is where the billing framework as well as the \AVS reaches its limits.
It is practically impossible to automatically track the time of the initial conversation between patient and practitioner.
One idea could be to track the time passed between the practitioner entering the anamnesis page and moving on to the physical examination page in the front-end.
This is, however, highly unreliable as doctors often enter the data into the software at the very end of the treatment.
One slightly less user-friendly way to solve is to make the user explicitly enter the duration of the anamnesis.
The current version of the \AVS does not offer this possibility, making reliably deriving \goa{1} not possible.
The user needs to add this code manually.

\paragraph{\goa{3: In-depth advice that goes beyond the usual level – also by telephone –}}
\goa{3} is similar to \goa{1} but applies to doctor patient consultations that take longer than 10 minutes.
Again, we have the same issue here as in \ref{par:goa-1} and are not able to automatically derive it.

\subsection{GOÄ Examination Code Rules}\label{subsec:examination-code-rules}
Examination Codes are frequently used codes that apply to provided physical examinations.
They primarily base on the existence of physical examination blocks in the treatment.
We consider the following codes in the experiments.

\paragraph{\goa{5: Symptom-related examination}}
\goa{5} applies to examinations the practitioner provides using simple medical tools.
According to the GOÄ, typical examinations that fall into that category are inspections, palpations, percussions and reflex testings.
The following rule checks for an existing examination of that type.

\lstinputlisting[
language=json,
style=json,
caption={\goa{5: Examination to determine the whole body status, including documentation if necessary}}
label={lst:goae-5-rule}
]{code/rules/experiments/goae-base-rules-5.json}

\paragraph{\goa{GOÄ 8: Examination to determine the whole body status, including documentation if necessary}}
According to its description in the GOÄ \cite{hermanns2015ebm}, it is applicable if the doctor provided the following types of examinations:
\begin{itemize}
    \item examinations of the skin
    \item examinations of the visible mucous membranes
    \item examinations of the breast or abdominal organs
    \item examinations of the supporting and movement organs
    \item orienting neurological examination
\end{itemize}
We need to conjugate the existence of all those five examinations to check if \goa{8} is applicable.
We can easily use the logical tree feature presented in \addref to implement this as follows.
The rule assumes that an examination of the skin is present if the practitioner has entered data into the \code{skinChangeExaminations} block as part of the physical examination stage.
It assumes that an examination of the supporting and movement organs is present if the practitioner has entered data into the \code{mobilityTestCervicalSpine}, the \code{mobilityTestThoracicSpine} or the \code{mobilityTestLumbarSpine} block.
Writing rules for examination codes like \goa{8} requires us to define a mapping between the anamnesis or physical examination blocks from the \AV data model and the concrete examinations required by the rule.

\lstinputlisting[
language=json,
style=json,
caption={\goa{8: Examination to determine the whole body status, including documentation if necessary}}
label={lst:goae-A-rule}
]{code/rules/experiments/goae-base-rules-8.json}

\paragraph{\goa{11: Digital examination of the rectum and/or prostate}}
This code applies for treatments with an examination of the rectum or prostate.
This condition translates to the existence of a \code{digitalRectalExamination} or a \code{prostateExamination} block.

\lstinputlisting[
language=json,
style=json,
caption={GOÄ 11}
label={lst:goae-11-rule}
]{code/rules/experiments/goae-base-rules-11.json}





This code is similar to



\lstinputlisting[
language=json,
style=json,
caption={GOÄ 6}
label={lst:goae-8-rule}
]{code/rules/experiments/goae-base-rules-8.json}


\goa{8} is formally defined as an examination of the following organ systems:
\begin{itemize}
    \item the Skin
    \item visible mocous membranes
    \item abdominal organs
    \item supporting and musculoskeletal system organs
    \item orienting neurological examination
\end{itemize}

Note that we use the \code{requiredPhysicalBlocks} field to express conditions on physical examination blocks.
The rule represents each of the mentioned organ systems as a child node of the root node.
The root node applies an \code{AND} operation to all its sub-children.
Each sub-child represents one of the organ systems required by \goa{8}.
The \AVS does not directly define an organ systems examination
However, we can flexibly define an organ system examination by a logical tree of required physical blocks.
This rule defines a skin examination to be present if a \code{skinTurgor}, a \code{skinColor}, a \code{skinThickness} or a \code{skinChangeExamination} block is present.
We can represent the rest of the organ systems in a similar way using the logical tree pattern.
Additionally there is already a respective single block for the orienting neurological examination.

Note that the rule just checks only for existence of examination blocks and does not impose any field conditions.
Unlike \goa{6}, \goa{8} does not mention any additional examination details.
%\goa{6} is very similar to \goa{8} but much larger as it

\subsection{GOÄ Surcharges}\label{subsec:goa-surcharges}

The GOÄ defines surcharges that are easily missed out in praxis.
We cover them in section \ref{sec:surcharges}.
The following rules are time related and cover inconveniences linked to service provision times.

\lstinputlisting[
language=json,
style=json,
caption={\goa{A: Surcharge for services provided outside office hours}}
label={lst:goae-A-rule}
]{code/rules/experiments/goae-base-rules-A.json}

\lstinputlisting[
language=json,
style=json,
caption={\goa{B: Surcharge for services provided outside office hours between 8 p.m. and 10 p.m. or 6 a.m. and 8 a.m}}
label={lst:goae-8-rule}
]{code/rules/experiments/goae-base-rules-B.json}

\lstinputlisting[
language=json,
style=json,
caption={\goa{C: Surcharge for services provided between 10 p.m. and 6 a.m}}
label={lst:goae-C-rule}
]{code/rules/experiments/goae-base-rules-C.json}

\lstinputlisting[
language=json,
style=json,
caption={\goa{D: Surcharge for services provided on Saturdays, Sundays or public holidays}}
label={lst:goae-D-rule}
]{code/rules/experiments/goae-base-rules-D.json}

The following surcharges are applicable to patients younger than 4 years old if one of the required GOÄ codes are present.

\lstinputlisting[
language=json,
style=json,
caption={\goa{K1: Surcharge to the benefits according to numbers 45, 46, 48, 50, 51, 55 or 56 for children up to the age of 4}}
label={lst:goae-K1-rule}
]{code/rules/experiments/goae-base-rules-K1.json}

\lstinputlisting[
language=json,
style=json,
caption={\goa{K2: Surcharge for examinations according to numbers 5, 6, 7 or 8 for children up to the age of 4}}
label={lst:goae-K2-rule}
]{code/rules/experiments/goae-base-rules-K2.json}


\goa{E} is a surcharge for services that require special urgency.
When a patient experiences severe pain, it is often an indicator of a highly urgent medical situation.
In such cases, the doctor assesses and documents the patient's level of pain.
Therefore, we have a high correlation between a patient's documented pain and the urgency of the treatment.
The following rule makes use of this connection.
\lstinputlisting[
language=json,
style=json,
caption={\goa{E: Surcharge for services requiring special urgency}}
label={lst:goae-E-rule}
]{code/rules/experiments/goae-base-rules-E.json}


\subsection{Multiplier Justification Rules}\label{subsec:multiplier-justification-rules}
Section \ref{sec:billing-multipliers} and \ref{sec:multiplier-justifications} introduce the concept of billing multipliers in private treatments.
Multipliers are applicable to existing GOÄ codes, however, require justifying and provable conditions in the treatment data.
As multiplier justifications are neither standardized nor pre-defined by any authority,
part of this work was to collect commonly used multiplier justifications.
Thanks to the medical experts at \AV, I received access to medical documents that contained real multiplier justifications used in the Dermato Oncology department of the LMU.
Many of the following rules implement those justifications.
As mentioned in section \ref{sec:language-design-workflow} I took part in the PVS Seminar, where I received more information about possible multiplier justifications.

Multiplier justifications either be target-specific or not.



\subsubsection{Non-specific multiplier justifications}


\subsubsection{Procedure-specific and duration-related multiplier justifications}
Procedures often take longer than expected.
This can be the case due to unforeseen complications and issues.
The following rule list shows the generic pattern used for procedure-specific and duration-related multiplier justifications.

\lstinputlisting[
language=json,
style=json,
caption={\goa{generic procedure duration multiplier justification rule}}
label={lst:goae-K2-rule}
]{code/rules/experiments/goae-base-rules-K2.json}

The rule base contains implementations of that pattern for each of the procedures that the system currently supports.

Note that the rule only checks for the existence of the respective procedure block.
We use the special block conditions \code{\$minDuration} and \code{\$maxDuration} to make the rule only match if the duration of the procedure is within the specified interval.
Each procedure may have multiple duration related justification rules, each one covering another time interval and another multiplier value.


\lstinputlisting[
language=json,
style=json,
caption={\goa{E: Surcharge for services requiring special urgency}}
label={lst:goae-E-rule}
]{code/rules/experiments/goae-base-rules-E.json}

%As described in section \ref{sec:billing-multipliers} they are merely descriptive comments attached to GOÄ billing positions.


Next, I
