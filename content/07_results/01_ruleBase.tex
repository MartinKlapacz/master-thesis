\section{Experimental Rule}\label{sec:experimental-rule-base}

Obviously, in this work we cannot cover all rules that would be part of an actual rule base.

However, the first version of the billing system is highly

\subsection{GOÄ}
\subsubsection{Base GOÄ rules}

\subsubsubsection{GOÄ 8}
The following rule defines \goa{8}, which is defined as an \mete{Examination to determine the whole body status, including documentation if necessary}.

\lstinputlisting[
language=json,
style=json,
caption={GOÄ 6}
label={lst:goae-8-rule}
]{code/rules/experiments/goae-base-rules-8.json}


\goa{8} is formally defined as an examination of the following organ systems:
\begin{itemize}
    \item the Skin
    \item visible mocous membranes
    \item abdominal organs
    \item supporting and musculoskeletal system organs
    \item orienting neurological examination
\end{itemize}

Note that we use the \code{requiredPhysicalBlocks} field to express conditions on physical examination blocks.
The rule represents each of the mentioned organ systems as a child node of the root node.
The root node applies an \code{AND} operation to all its sub-children.
Each sub-child represents one of the organ systems required by \goa{8}.
The \AVS does not directly define an organ systems examination
However, we can flexibly define an organ system examination by a logical tree of required physical blocks.
This rule defines a skin examination to be present if a \code{skinTurgor}, a \code{skinColor}, a \code{skinThickness} or a \code{skinChangeExamination} block is present.
We can represent the rest of the organ systems in a similar way using the logical tree pattern.
Additionally there is already a respective single block for the orienting neurological examination.

Note that the rule just checks only for existence of examination blocks and does not impose any field conditions.
Unlike \goa{6}, \goa{8} does not mention any additional examination details.
%\goa{6} is very similar to \goa{8} but much larger as it

\subsubsection{GOÄ Surcharges}

The GOÄ defines surcharges that are easily missed out in praxis.
Section \ref{sec:surcharges} introduces this type of GOÄ codes and mentions a few examples.
The following rules are time related and cover inconveniences linked to service provision times.

\lstinputlisting[
language=json,
style=json,
caption={\goa{A: Surcharge for services provided outside office hours}}
label={lst:goae-A-rule}
]{code/rules/experiments/goae-base-rules-A.json}

\lstinputlisting[
language=json,
style=json,
caption={\goa{B: Surcharge for services provided outside office hours between 8 p.m. and 10 p.m. or 6 a.m. and 8 a.m}}
label={lst:goae-8-rule}
]{code/rules/experiments/goae-base-rules-B.json}

\lstinputlisting[
language=json,
style=json,
caption={\goa{C: Surcharge for services provided between 10 p.m. and 6 a.m}}
label={lst:goae-C-rule}
]{code/rules/experiments/goae-base-rules-C.json}

\lstinputlisting[
language=json,
style=json,
caption={\goa{D: Surcharge for services provided on Saturdays, Sundays or public holidays}}
label={lst:goae-D-rule}
]{code/rules/experiments/goae-base-rules-D.json}

The following surcharges are applicable to patients younger than 4 years old if one of the required GOÄ codes are present.

\lstinputlisting[
language=json,
style=json,
caption={\goa{K1: Surcharge to the benefits according to numbers 45, 46, 48, 50, 51, 55 or 56 for children up to the age of 4}}
label={lst:goae-K1-rule}
]{code/rules/experiments/goae-base-rules-K1.json}

\lstinputlisting[
language=json,
style=json,
caption={\goa{K2: Surcharge for examinations according to numbers 5, 6, 7 or 8 for children up to the age of 4}}
label={lst:goae-K2-rule}
]{code/rules/experiments/goae-base-rules-K2.json}


\goa{E} is a surcharge for services that require special urgency.
When a patient experiences severe pain, it is often an indicator of a highly urgent medical situation.
In such cases, the doctor assesses and documents the patient's level of pain.
Therefore, we have a high correlation between a patient's documented pain and the urgency of the treatment.
The following rule makes use of this connection.
\lstinputlisting[
language=json,
style=json,
caption={\goa{E: Surcharge for services requiring special urgency}}
label={lst:goae-E-rule}
]{code/rules/experiments/goae-base-rules-E.json}


\subsubsection{Multiplier Justification Rules}\label{subsubsec:multiplier-justification-rules}
Section \ref{sec:billing-multipliers} and \ref{sec:multiplier-justifications} introduce the concept of billing multipliers in private treatments.
Multipliers are applicable to existing GOÄ codes, however, require justifying and provable condition in the treatment data.
As multiplier justifications are neither standardized nor pre-defined by any authority,
part of this work was to collect commonly used multiplier justifications.
Thanks to the medical experts at \AV, I received access to medical documents that contained real multiplier justifications used in the Dermato Oncology department of the LMU.
Many of the following rules implement those justifications.
Additionally, I took part in \todo{Explain PVC semianr}


\subsubsection{Non-specific multiplier justifications}



\subsubsubsection{Procedure-specific and duration-related multiplier justifications}
Procedures often take longer than expected.
This can be the case due to unforeseen complications and issues.
The following rule list shows the generic pattern used for procedure-specific and duration-related multiplier justifications.

\lstinputlisting[
language=json,
style=json,
caption={\goa{generic procedure duration multiplier justification rule}}
label={lst:goae-K2-rule}
]{code/rules/experiments/goae-base-rules-K2.json}

The rule base contains implementations of that pattern for each of the procedures that the system currently supports.

Note that the rule only checks for the existence of the respective procedure block.
We use the special block conditions \code{\$minDuration} and \code{\$maxDuration} to make the rule only match if the duration of the procedure is within the specified interval.
Each procedure may have multiple duration related justification rules, each one covering another time interval and another multiplier value.


\lstinputlisting[
language=json,
style=json,
caption={\goa{E: Surcharge for services requiring special urgency}}
label={lst:goae-E-rule}
]{code/rules/experiments/goae-base-rules-E.json}

%As described in section \ref{sec:billing-multipliers} they are merely descriptive comments attached to GOÄ billing positions.

\subsubsection{Procedure GOÄ Rules}
The procedure GOÄ rules are primariliy based on the work of Dr. Sarah Bojko, who has a deep understanding of the specification of the procedure available in the \AVS.

\goa{1530: Examination of the larynx with the laryngoscope} and \goa{1533: Floating or supporting laryngoscopy, each as an independent service} are both codes referring to a laryngoscopy procedure.
The \code{LaryngoscopyMsg} represents this procedure and has a boolean field \code{floatingOrSupportedLaryngoscopy}.
Both rules match check for the existence of a laryngoscopy and if \code{floatingOrSupportedLaryngoscopy} has the correct value.
Only one of both rule can match at one time.

\lstinputlisting[
language=json,
style=json,
caption={\goa{E: Surcharge for services requiring special urgency}}
label={lst:goae-1530-rule}
]{code/rules/experiments/goae-procedure-rule-473-474.json}


\goa{255: Injection, intra-articular or perineural} refers to a spine infiltration represented by the \code{spineInfiltration} procedure block.

\lstinputlisting[
language=json,
style=json,
caption={\goa{255: Surcharge for services requiring special urgency}}
label={lst:goae-255-rule}
]{code/rules/experiments/goae-procedure-rule-255.json}


\lstinputlisting[
language=json,
style=json,
caption={\goa{E: Surcharge for services requiring special urgency}}
label={lst:goae-1530-rule}
]{code/rules/experiments/goae-procedure-rule-1530-1533.json}

