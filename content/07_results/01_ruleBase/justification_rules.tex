\subsection{Multiplier Justification Rules}\label{subsec:multiplier-justification-rules}
Section \ref{sec:billing-multipliers} and \ref{sec:multiplier-justifications} introduce the concept of billing multipliers in private treatments.
Multipliers are applicable to existing GOÄ codes, however, require provable justifications and must rely on facts in the treatment data.
As multiplier justifications are neither standardized nor pre-defined by any authority,
part of this work was to collect commonly used multiplier justifications.
Thanks to the medical experts at \AV, I received access to medical documents that contained real multiplier justifications used in the Dermato Oncology department of the LMU.
Many of the following rules implement those justifications.
As mentioned in section \ref{sec:language-design-workflow} I took part in the PVS Seminar, where I received more information about possible multiplier justifications.
As described in subsection \ref{subsec:the-target-code-field}, Multiplier justifications can either be target-specific or generally applicable.



\lstinputlisting[
    language=json,
    style=json,
    caption={\goa{Multiplier Justification refering to communication issues}},
    label={lst:mj-communication-rule}
]{code/rules/experiments/justification-rule-communication.json}


\lstinputlisting[
language=json,
style=json,
caption={\goa{Multiplier Justification refering to issues caused by the patient's current drug abuse}},
label={lst:mj-cdrug-rule}
]{code/rules/experiments/justification-rule-multimorbidity.json}


\subsubsection{Procedure-specific and duration-related multiplier justifications}
Procedures often take longer than expected.
This can be the case due to unforeseen complications and issues.
The following rule list shows the generic pattern used for procedure-specific and duration-related multiplier justifications.

%\lstinputlisting[
%    language=json,
%    style=json,
%    caption={\goa{Generic procedure duration multiplier justification rule}},
%    label={lst:goae-K2-rule}
%]{code/rules/experiments/goae-base-rules-K2.json}

\todo{add rules}

The rule base contains implementations of that pattern for each of the procedures that the system currently supports.

Note that the rule only checks for the existence of the respective procedure block.
We use the special block conditions \code{\$minDuration} and \code{\$maxDuration} to make the rule only match if the duration of the procedure is within the specified interval.
Each procedure may have multiple duration related justification rules, each one covering another time interval and another multiplier value.


%\lstinputlisting[
%    language=json,
%    style=json,
%    caption={\goa{E: Surcharge for services requiring special urgency}},
%    label={lst:goae-E-rule}
%]{code/rules/experiments/goae-base-rules-E.json}
