\subsection{GOÄ Examination Code Rules}\label{subsec:examination-code-rules}
Examination Codes are frequently used codes that apply to provided physical examinations.
They primarily base on the existence of physical examination blocks in the treatment.
We consider the following codes in the experiments.

\paragraph{\goa{5: Symptom-related examination}}
\goa{5} applies to examinations the practitioner provides using simple medical tools.
According to the GOÄ, typical examinations that fall into that category are inspections, palpations, percussions, reflex testings and measurements.
The following rule checks for an existing examination of that type.
\lstinputlisting[
language=json,
style=json,
caption={\goa{5: Examination to determine the whole body status, including documentation if necessary}}
label={lst:goae-5-rule}
]{code/rules/experiments/goae-base-rules-5.json}

\paragraph{\goa{6}}
\goa{6} applies to examinations of at least on of the following organ systems:
\begin{itemize}
    \item All sections of the eye
    \item The entire ENT (ear, nose, throat) area
    \item The stomatognathic system
    \item The kidneys and urinary tract
    \item Examination to determine a complete vascular status
\end{itemize}
For the case of simplicity, the code snipped illustrating the rule for \goa{6} omits a few less interesting details.
\goa{6} further requirements on the provided examination \cite{bruck1998kommentar}.
For example, an eye related examination must striclty cover both eyes.
The same must hold for the ears, if the practitioner provided an ENT examination.
The rule implements this by not only requiring the existence of blocks, as it does for the \code{visualAcuity} and \code{oculomotorSkills},
but also expressing conditions on nested objects in the blocks.
For example, it specifies the existence of a \code{pupilLeft} and \code{pupilRight} block inside the \code{pupilsAndLightReaction} physical examination block.
The rule specifies an equivalent expression for \code{earsAndMastoidPalpation}.

\lstinputlisting[
language=json,
style=json,
caption={\goa{6}}
label={lst:goae-6-rule}
]{code/rules/experiments/goae-base-rules-6.json}


\lstinputlisting[
language=json,
style=json,
caption={\goa{6: Complete physical examination of at least one of the following organ systems: all eye sections, the entire ENT area, the stomatognathic system, the kidneys and urinary tract (in men, including the male genital organs if necessary) or examination to determine a complete vascular status - including documentation, if necessary}}
label={lst:goae-5-rule}
]{code/rules/experiments/goae-base-rules-5.json}


\paragraph{\goa{7}}
\goa{7} applies to detailed physical examinations of at least one of the following entire organ system:
\begin{itemize}
    \item Skin organ
    \item Support and movement organs
    \item Chest organs
    \item Abdominal organs
    \item Female genital tract
\end{itemize}
For the case of simplicity, the following rule only considers the skin and the support and movement organs.
The omitted organ systems are not relevant for the experimental treatment cases.
\lstinputlisting[
language=json,
style=json,
caption={\goa{7: Complete physical examination of at least one of the following organ systems: the entire skin organ, the supporting and locomotor organs, all breast organs, all abdominal organs, the entire female genital tract (including kidneys and urinary tract, if applicable) - including documentation, if applicable -}}
label={lst:goae-7-rule}
]{code/rules/experiments/goae-base-rules-7.json}


\paragraph{\goa{8}}
According to its description in the GOÄ \cite{hermanns2015ebm}, it is applicable if the doctor provided the following types of examinations:
\begin{itemize}
    \item Examinations of the skin
    \item Examinations of the visible mucous membranes
    \item Examinations of the breast or abdominal organs
    \item Examinations of the supporting and movement organs
    \item Orienting neurological examination
\end{itemize}
We need to conjugate the existence of all those five examinations to check if \goa{8} is applicable.
We can easily use the logical tree feature presented in \addref to implement this as follows.
The rule assumes that an examination of the skin is present if the practitioner has entered data into the \code{skinChangeExaminations} block as part of the physical examination stage.
It assumes that an examination of the supporting and movement organs is present if the practitioner has entered data into the \code{mobilityTestCervicalSpine}, the \code{mobilityTestThoracicSpine} or the \code{mobilityTestLumbarSpine} block.
Writing rules for examination codes like \goa{8} requires us to define a mapping between the anamnesis or physical examination blocks from the \AV data model and the concrete examinations required by the rule.

\lstinputlisting[
Language=json,
style=json,
caption={\goa{8: Examination to determine the whole body status, including documentation if necessary}}
label={lst:goae-8-rule}
]{code/rules/experiments/goae-base-rules-8.json}


\paragraph{\goa{11: Digital examination of the rectum and/or prostate}}
This code applies for treatments with an examination of the rectum or prostate.
This condition translates to the existence of a \code{digitalRectalExamination} or a \code{prostateExamination} block.

\lstinputlisting[
Language=json,
style=json,
caption={GOÄ 11}
label={lst:goae-11-rule}
]{code/rules/experiments/goae-base-rules-11.json}



%Note that we use the \code{physicalBlocks} field to express conditions on physical examination blocks.
%The rule represents each of the mentioned organ systems as a child node of the root node.
%The root node applies an \code{AND} operation to all its sub-children.
%Each sub-child represents one of the organ systems required by \goa{8}.
%The \AVS does not directly define an organ systems examination
%However, we can flexibly define an organ system examination by a logical tree of required physical blocks.
%This rule defines a skin examination to be present if a \code{skinTurgor}, a \code{skinColor}, a \code{skinThickness} or a \code{skinChangeExamination} block is present.
%We can represent the rest of the organ systems in a similar way using the logical tree pattern.
%Additionally there is already a respective single block for the orienting neurological examination.
%
%Note that the rule checks only for existence of examination blocks and does not impose any field conditions.


\subsection{Special GOÄ Consultation Code Rules}\label{subsec:special-consultation-code-rules}

\paragraph{\goa{1: Advice – also via telephone}}\label{par:goa-1}
\goa{1} is one of the most commonly used codes in the GOÄ.
It is applicable for a conversation between patient and doctor at the beginning of the visit that takes at most 10 minutes.
This type of conversation happens typically as part of the anamnesis.

Unfortunately, this is where the billing framework as well as the \AVS reaches its limits.
It is practically impossible to automatically track the time of the initial conversation between patient and practitioner.
One idea could be to track the time passed between the practitioner entering the anamnesis page and moving on to the physical examination page in the front-end.
This is, however, highly unreliable as doctors often enter the data into the software at the very end of the treatment.
One slightly less user-friendly way to solve is to make the user explicitly enter the duration of the anamnesis.
The current version of the \AVS does not offer this possibility, making reliably deriving \goa{1} not possible.
The user needs to add this code manually.

\paragraph{\goa{3: In-depth advice that goes beyond the usual level – also by telephone –}}
\goa{3} is similar to \goa{1} but applies to doctor patient consultations that take longer than 10 minutes.
Again, we have the same issue here as in \ref{par:goa-1} and are not able to automatically derive it.

\subsection{GOÄ Surcharges}\label{subsec:goa-surcharges}
The GOÄ defines surcharges that are easily missed out in praxis.
Section \ref{sec:surcharges} covers them in details where we also mention that surcharges have tight exclusion rules.
These exclusions are already part of the structured data in the GOÄ catalog, making the usage of the \code{excludedCodes} fields here obsolete.
The codes from \goa{401} up to \goa{410} are surcharges as well, but are not relevant for the experimental treatments provided by \AV.

\lstinputlisting[
language=json,
style=json,
caption={\goa{A: Surcharge for services provided outside office hours}}
label={lst:goae-A-rule}
]{code/rules/experiments/goae-base-rules-A.json}

\lstinputlisting[
language=json,
style=json,
caption={\goa{B: Surcharge for services provided outside office hours between 8 p.m. and 10 p.m. or 6 a.m. and 8 a.m}}
label={lst:goae-8-rule}
]{code/rules/experiments/goae-base-rules-B.json}

\lstinputlisting[
language=json,
style=json,
caption={\goa{C: Surcharge for services provided between 10 p.m. and 6 a.m}}
label={lst:goae-C-rule}
]{code/rules/experiments/goae-base-rules-C.json}

\lstinputlisting[
language=json,
style=json,
caption={\goa{D: Surcharge for services provided on Saturdays, Sundays or public holidays}}
label={lst:goae-D-rule}
]{code/rules/experiments/goae-base-rules-D.json}

The following surcharges are applicable to patients younger than 4 years old if one of the required GOÄ codes are present.

\lstinputlisting[
language=json,
style=json,
caption={\goa{K1: Surcharge to the benefits according to numbers 45, 46, 48, 50, 51, 55 or 56 for children up to the age of 4}}
label={lst:goae-K1-rule}
]{code/rules/experiments/goae-base-rules-K1.json}

\lstinputlisting[
language=json,
style=json,
caption={\goa{K2: Surcharge for examinations according to numbers 5, 6, 7 or 8 for children up to the age of 4}}
label={lst:goae-K2-rule}
]{code/rules/experiments/goae-base-rules-K2.json}

\goa{E} is a surcharge for services that require special urgency.
A patient experiencing severe pain often necessitates urgent medical intervention.
In these situations, the doctor documents the patient's pain level.
The pain level can be seen as an indicator for the urgency of the treatment and therefore used as a hint for \goa{E}, which is a surcharge applicable for urgent treatments.
\lstinputlisting[
Language=json,
style=json,
caption={\goa{E: Surcharge for services requiring special urgency}}
label={lst:goae-E-rule}
]{code/rules/experiments/goae-base-rules-E.json}


\subsubsection{Procedure GOÄ Rules}

The procedure GOÄ rules are primary based on the work of Dr. Sarah Bojko, who has a deep understanding of the specification of the procedure available in the \AVS.

\paragraph{1530 and 1533}
\goa{1530: Examination of the larynx with the laryngoscope} and \goa{1533: Floating or supporting laryngoscopy, each as an independent service} are both codes referring to a laryngoscopy procedure.
The \code{LaryngoscopyMsg} block represents this procedure and has a boolean field \code{floatingOrSupportedLaryngoscopy}.
Both rules check for the existence of a laryngoscopy and the required value of the \code{floatingOrSupportedLaryngoscopy} boolean field.
Only one of both rule can match at one time.
\lstinputlisting[
language=json,
style=json,
caption={Rules for \goa{1530} and \goa{1533}}
label={lst:goae-1530-1533-rules}
]{code/rules/experiments/goae-procedure-rule-1530-1533.json}

\paragraph{255}
\goa{255: Injection, intra-articular or perineural} refers to a spine infiltration represented by the \code{spineInfiltration} procedure block.
Subsection \ref{subsubsec:quantity-derivation-from-procedure-information} introduces the concept of quantity functions using the example of the spine infiltration.

The following rule matches if a spineInfiltration exists followed by the computation the quantity of its conclusion code \code{255}.
Code snippet \ref{lst:spineInfiltration} illustrates its protobuf representation, the field paths used in the rule must comply with.

\lstinputlisting[
    language=json,
    style=json,
    caption={Rule for \goa{255: Injection, intra-articular or perineural}}
    label={lst:goae-255-rule}
]{code/rules/experiments/goae-procedure-rule-255.json}

The spine infiltration block has a localization field for all possible injection localizations in the human spine,
making the object in the data model quite extensive.
The rule specifies a field evaluation sub-function for each of those localization fields.
The actual rule consists of more than 200 lines of codes,
which is why we can only include a small section of it in the code snippet.
\lstinputlisting[
    language=json,
    style=json,
    caption={\goa{E: Surcharge for services requiring special urgency}}
    label={lst:goae-1530-rule}
]{code/rules/experiments/goae-procedure-rule-1530-1533.json}

\paragraph{460 and 461}
\goa{460: Combination anesthesia with mask, device - also insufflation anesthesia -, up to one hour}
and \goa{461: Combination anesthesia with mask, device - also insufflation anesthesia - every additional half hour} are two GOÄ codes that refer to the same service but cover different time spans.
Subsection \goa{subsec:quantity-derivation-from-durations} already introduces this type of duration interconnections between GOÄ codes in more details.

\lstinputlisting[
language=json,
style=json,
caption={\goa{6}}
label={lst:goae-6-rule}
]{code/rules/experiments/goae-procedure-rule-460.json}

Code \code{460} holds if a \code{generalAnesthesia} block with the required fields exists.
Code \code{461} holds if a \code{generalAnesthesia} block with the required fields exists and the procedure has a duration of at least 60 minutes.
In this case, the quantity of \code{461} is equal to the number of 30-minute-spans that subsequently follow the first 60 minutes.
The quantity of \code{461} must ignore the first two 30-minute-spans, which is why the quantity function counts the number of commenced 30-minute-spans, subtracting the first two.
Adding a constant of \code{-2} to the procedure duration expression in the quantity function implements this logic.

\paragraph{473 and 474}
\goa{473: Induction and monitoring of continuous subarachnoid spinal anesthesia (lumbar anesthesia) or epidural (epidural) anesthesia with catheter, lasting up to five hours}
and
\goa{474: Induction and monitoring of continuous subarachnoid spinal anesthesia (lumbar anesthesia) or epidural (epidural) anesthesia with catheter, lasting more than five hours}
Are also two anesthesia billing codes that refer to the same type of service but distinguish between durations
The rules in the following code snippet implement both billing conditions
Code \code{473} applies if the practitioner provided a \code{neurexialAnesthesia} procedure with one of the required procedure types and the duration took less than five hours.
Code \code{474} applies for the case when the same procedure takes more than five hours.

\lstinputlisting[
language=json,
style=json,
caption={\goa{473 and 474}}
label={lst:goae-6-rule}
]{code/rules/experiments/goae-procedure-rule-473-474.json}

\paragraph{614}
\goa{614: Transcutaneous measurement(s) of oxygen partial pressure} is applicable for respiration measurements performed during the physical examination stage.
Practitioners document these measurements in the \code{respiration} block which stores a list of measurement objects representing one performed measurement.
\goa{614} is applicable once for each measurement, which is implemented in the following rule.

\lstinputlisting[
    language=json,
    style=json,
    caption={\goa{614}}
    label={lst:goae-614-rule}
]{code/rules/experiments/goae-procedure-rule-614.json}
