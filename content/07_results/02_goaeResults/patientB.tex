\subsection{Patient case 4: Neurological Examination due to speech difficulties}\label{subsec:patient-case-4:-neurological-examination-due-to-speech-difficulties}
The patient, a male individual, presented unaccompanied to the clinic with significant speech difficulties.

\paragraph{Anamnese}:
The anamnesis revealed the previously occurred stroke, which is likely to be the cause for the speaking issues the patient is experiencing now.
The patient was revealed to be a regular smoker, which is a risk factor for vascular diseases that might have favored the stroke.
The anamnesis took about 12 minutes.

\paragraph{Physical Examination}
The practitioner conducted a Speech examination (\code{SpeechBlock}) and documented that the patient had severe speaking issues.

\paragraph{Procedures}
The doctor decided to perform a laryngoscopy, which revealed a paralysis of the vocal cords.
This procedure provides an in-depth evaluation of the throat and vocal cords to check for any issues that might be affecting his speech.
This condition is often associated with neurological issues which is a direct consequence of the recent stroke,
explaining the patient's significant speech difficulties.

\paragraph{Difficulties and Inconveniences}
The patient was unaccompanied and had neurological issues to articulate himself.
This turned out to be a challenging factor complicating the treatment.

\subsubsection{Expected Results}
As the anamnesis, took longer that \goa{10} minutes, the framework should suggest to bill \goa{3}.
The framework should be able to distinguish between \goa{1: Advice – also by telephone} and \goa{3: In-depth advice that goes beyond the usual level – also by telephone –}
Both codes refer to the same service but require different durations.
\goa{1} is applicable for consultations that take less than 10 minutes while \goa{3} is applicable for the same type of medical conversation but requires a duration of at least 10 minutes.
The framework should also be able to detect challenging communication caused by the patient's health issues and suggest an appropriate multiplier.

Additionally, the provided speech examination can be billed with \todo{add goa code for this}.

The laryngoscopy should be billed with \goa{1530}.

\subsubsection{Actual Results}

% - anesthesia?

%- sprach untersuchung

% -
