\subsection{Patient case 4: Neurological Examination due to speech difficulties}\label{subsec:patient-case-4:-neurological-examination-due-to-speech-difficulties}
The patient, a male individual, presented unaccompanied to the clinic with significant speech difficulties.
The purpose of the treatment is to find the reason for the speech difficulties.
The outcome of this treatment will determine further the path for further treatments.

\paragraph{Anamnesis}
He revealed that he had a stroke six weeks ago, from which he has recovered well.
However, since then, he has had poorer speech quality and is coming for treatment to have it examined more closely.
The stroke is likely to be the cause for the speaking issues the patient is experiencing now.
The patient turned out to be a regular smoker, which is a risk factor for vascular diseases that might have favored the stroke.
It is likely to be a direct consequence of the stroke but can also be linked to an undiscovered tumor.
The anamnesis took about 12 minutes.

\paragraph{Physical Examination}
The practitioner conducted a speech examination and documented severe speaking issues.
Additionally, the practitioner inspected the patient's mouth and throat, documenting the outcomes in a \code{mouthThroatAndLarynxInspection} block.
\paragraph{Procedures}
The doctor decided to perform a laryngoscopy with free-floating tracheal anesthesia.
This procedure provides an in-depth evaluation of the throat and vocal cords to check for any issues affecting his speech.
This procedure revealed a paralysis of the vocal cords, which is a direct consequence of the recent stroke.
This explained the patient's significant speech difficulties and helped the practitioner exclude a tumor.

\paragraph{Difficulties and Inconveniences}
The patient was unaccompanied and had neurological issues to articulate himself.
This turned out to be a challenging factor complicating the treatment.

\subsubsection{Expected Results}
As the anamnesis, took longer that \goa{10} minutes, the framework should suggest to bill \goa{3}.
The framework should be able to distinguish between \goa{1} and \goa{3}
Both codes refer to the same service but require different durations.
The framework should also be able to detect challenging communication caused by the patient's health issues and suggest an appropriate multiplier.
It should also derive \goa{1530} for the performed laryngoscopy and \goa{5} for the speech examination and throat inspection.

\subsubsection{Actual Results}
Similarly to the first patient case in subsection \ref{subsec:patient-a---multiple-spine-infiltrations-due-to-backpain-with-obesity}, the framework could not derive \goa{3}.
The framework successfully detected the challenging condition caused by the patient's communication issues.
The multiplier justification \code{Disturbance of verbal communication due to medical condition} matched with the practitioner's documented \code{speech} block.
Rule \code{1530} in listing \ref{lst:goae-1530-1533-rules} matched with the patient's \code{laryngoscopy} procedure documentation.

%- sprach untersuchung

%3
%5
%1530

%-> informationen für weitere behandlung und logopädie, hätte auch tumor sein können, man schließt aus ob man operativ was machen muss, behadlungspfad anpassen
%move invasive -> need more aufwand which is why its a procedure
