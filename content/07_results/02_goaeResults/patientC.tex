\subsection{Patient Case 3: Anesthesia Provision For Birth}\label{subsec:patient-case-3:-anesthesia-for-birth}
The patient is a 32-year-old woman and is currently giving birth on a wednesday night at 3 a.m.
It is common practice to create two distinct bills for a birth.
There is usually one for the anesthesia and another one for the gynecology.
This scenario covers the billing for the anesthesia.
The current version of the \AVS does not cover birth treatments.
The patient is not a native German speaker, making it necessary to communicate in english.

\paragraph{Anamnese}:
The treatment starts with a short introductory conversation between the patient and the anesthetist.
This involves details about the anesthesia and the patient's current and previous pregnancies.

\paragraph{Physical Examination}
Monitoring the blood oxygen is part of an anesthesia procedure and assures the patient's safety during the anesthesia.
This happens in the stage of the physical examination.
This is why the doctor measures a pulse oximetry of 95 and documents the results (\code{RespirationBlock}).
Additionally, the doctor documents the patient's scale of pain, setting a NRS score of 7 (\code{PainBlock}).

\paragraph{Procedures}
Finally, the anesthetist performs a so-called spinal anesthesia with catheter technology.
The practitioner documents the details in the corresponding procedure block (\code{NeurexialAnesthesiaBlock}).
The anesthesia procedure takes 11.5 hours.

\paragraph{Difficult Conditions}
The requirement to communicate in english during a critical treatment is considerable as a challenging condition for the practitioners.

\subsubsection{Expected Results}
The introductory conversation in the anamnesis stage took longer than 10 minutes, making \goa{1} applicable.
The system should derive \goa{614: Transcutaneous measurement(s) of oxygen partial pressure} for the measurement of the pulse oxymetry during the anesthesia.
Additionally, the framework should suggest
\goa{474: Induction and monitoring of continuous subarachnoid spinal anesthesia (lumbar anesthesia) or epidural (epidural) anesthesia with catheter, lasting more than five hours}
for the actual anesthesia performed before the birth.
The framework must not confuse it with \goa{473} which applies to the same anesthesia service that, however, takes less than 5 hours.
Due to the critical nature of a birth and the pain experienced by the patient, this anesthesia treatment can be considered highly urgent.
This justifies the application of \goa{E: Surcharge for urgently requested and immediate execution}.
Additionally, the treatment happens at an inconvenient time at 3am during the week,
making \goa{D: Surcharge for services provided between 10 p.m. and 6 a.m} applicable.
Additionally, the framework should suggest an appropriate multiplier justification that compensates for the requirement to communicate in english.

\subsubsection{Actual Results}

%Codes:
%1 x GOÄ 1
%1 x GOÄ 5
%1 x GOÄ 614 (due to respiration)
%1 x GOÄ 473 (anesthesia)
%1 x GOÄ E (pain)
%
%zuschläge D/C/G


%-> respiration block mit pulseOximetry > 94
