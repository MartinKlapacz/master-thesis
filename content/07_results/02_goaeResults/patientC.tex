\subsection{Patient Case 3: Anesthesia Provision For Birth}\label{subsec:patient-case-3:-anesthesia-for-birth}
The patient is a 32-year-old woman and is currently about to give birth on a wednesday night at 3 am.
It is common practice to create two distinct bills for a birth, one for the anesthesist and another one for the gynecologist.
The treating anesthesist is not a doctor but a care professional.
This scenario covers the billing for the anesthesia.
The current version of the \AVS does not cover birth treatments.
The patient is not a native German speaker, which has been entered into the software during patient registration.

\paragraph{Anamnese}:
The treatment starts with a short introductory conversation between the patient and the anesthetist.
This involves details about the anesthesia and the patient's current and previous pregnancies.
Additionally, the practitioner documents the patient's high level of pain.
The practitioner enters a pain severity of 8 into the software.

\paragraph{Physical Examination}
Monitoring the blood oxygen is part of an anesthesia procedure and assures the patient's safety during the anesthesia.
This happens in the stage of the physical examination.
The doctor makes a single pulse oximetry measurement of 95 and documents the results in a \code{Respiration} block.
Additionally, the doctor documents the patient's scale of pain in a \code{Pain} block,
setting a severity score of 8.

\paragraph{Procedures}
Finally, the anesthetist performs a so-called spinal anesthesia with catheter technology.
The practitioner documents the details in the corresponding \code{neurexialAnesthesia} procedure block.
The anesthesia procedure takes 11.5 hours.

\paragraph{Difficulties and Inconveniences}
The requirement to communicate in english during such a critical treatment is considerable as a challenging condition for the practitioners.

\subsubsection{Expected Results}
The introductory conversation in the anamnesis stage took longer than 10 minutes, making \goa{1} applicable.
The system should derive \goa{614: Transcutaneous measurement(s) of oxygen partial pressure} for the measurement of the pulse oxymetry during the anesthesia.
Additionally, the framework should suggest \goa{474} for the actual anesthesia performed before the birth.
The framework must not confuse it with \goa{473} which applies to the same anesthesia service that, however, takes less than 5 hours.
Due to the critical nature of a birth and the pain experienced by the patient, this anesthesia treatment can be considered highly urgent.
This justifies the application of \goa{E: Surcharge for urgently requested and immediate execution}.
Additionally, the treatment happens at an inconvenient time at 3am during the week,
making \goa{D: Surcharge for services provided between 10 p.m. and 6 a.m} applicable.
Additionally, the framework should suggest an appropriate multiplier justification that compensates for the requirement to communicate in english.

\subsubsection{Actual Results}

The framework was unable to derive \goa{1} due to the issues already described in subsection \ref{subsec:special-consultation-code-rules}.
It successfully derived a multiplier justification for the communication problems.
The issue is that this multiplier applies to a consultation service such as \goa{1}, which is not in der derivation set.

A \code{neuraxialAnesthesiaMsg} block was also part of the treatment, which triggered rules in \addref{473,474}.
Both rules check the procedure duration of \code{neuraxialAnesthesiaMsg}
As the anesthesia has a duration of 11.5 hours, the framework correctly rejected \code{473} and accepted \code{474}.

Code \code{614} matches due to the respiration documentation.
Its quantity function returns the length of the list containing all measurements inside the \code{respiration} block.
As the practitioner entered a single block, the rule engine derives a single \goa{614} code.

The framework also derived the \goa{E} surcharge applicable for urgent treatment.
Rule \addref matched due to the high pain documented in the \code{pain} block during anamnesis.
The anesthesist documented a pain score of 8 whilst the \code{pain} requires a pain score of at least 6.

The treatment takes place at 3 am on a wednesday night, which is outside any office hour.
This triggers surcharges \goa{A} and \goa{C}, which mutually exclude themselves.
The conflict resolution algorithm choses \goa{C} (320 points) over \goa{A} (70 points).
This results in the following bill.

%Codes:
%1 x GOÄ 1
%1 x GOÄ 614 (due to respiration)
%1 x GOÄ 473 (anesthesia)
%1 x GOÄ E (surcharge for urgency)
%1 x GOÄ C (surcharge for )


%-> respiration block mit pulseOximetry > 94
