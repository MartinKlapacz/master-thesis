\subsection{Patient Case 3: Anesthesia Provision For Birth}\label{subsec:patient-case-3:-anesthesia-for-birth}
The patient is a 32-year-old woman and is currently about to give birth on a Wednesday at 3 am.
It is common practice to create two bills for birth, one for the anesthetist and another for the gynecologist.
The treating anesthetist is not a doctor but a care professional.
This scenario covers the billing for the anesthesia.
The current version of the \AVS does not cover birth treatments.
The patient did not register as a native German speaker but has sufficient English capabilities.

\paragraph{Anamnesis}
The treatment starts with a short introductory conversation between the patient and the anesthetist.
This involves details about the anesthesia and the patient's current and previous pregnancies.
Additionally, the practitioner documents the patient's high level of pain.
The practitioner enters a pain severity of 8 into the software.

\paragraph{Physical Examination}
Blood oxygen monitoring is an essential part of an anesthesia procedure to ensure the patient's safety, which happens during the physical examination stage.
The doctor makes a single pulse oximetry measurement of 95 and documents the results in a \code{Respiration} block.

\paragraph{Procedures}
Finally, the anesthetist performs a so-called spinal anesthesia with catheter technology.
The practitioner documents the details in the corresponding \code{neurexialAnesthesia} procedure block.
Finally, the anesthetist performs a so-called spinal anesthesia with catheter technology.
The practitioner documents the details in the corresponding \code{neurexialAnesthesia} procedure block.
The anesthesia will be kept stable during the entire birth and will last for 11.5 hours.

\paragraph{Difficulties and Inconveniences}
The requirement to communicate in english during such a critical treatment is considerable as a challenging condition for the practitioners.

\subsubsection{Expected Results}
The introductory conversation in the anamnesis stage took longer than 10 minutes, making \goa{1} applicable.
The system should derive \goa{614: Transcutaneous measurement(s) of oxygen partial pressure} for measuring the pulse oximetry during the anesthesia.
Additionally, the framework should suggest \goa{474} for the actual anesthesia performed before the birth.
The framework must not confuse it with \goa{473}, which applies to equal anesthesia services that take less than 5 hours.
Due to the critical nature of birth and the pain experienced by the patient, this anesthesia treatment is highly urgent, justifying the application of \goa{E: Surcharge for urgently requested and immediate execution}.
Additionally, the treatment happens at an inconvenient time at 3 am during the week,
making \goa{D: Surcharge for services provided between 10 pm and 6 am} applicable.
Additionally, the framework should suggest an appropriate multiplier justification that compensates for the requirement to communicate in English.

\subsubsection{Actual Results}
The framework could not derive \goa{1} due to the issues already described in subsection \ref{subsec:special-consultation-code-rules}.
It successfully derived a multiplier justification for the communication problems.
However, this multiplier should apply to a consultation service such as \goa{1}, which the user must manually add.

A \code{neuraxialAnesthesiaMsg} block was also part of the treatment, triggering rules in listing \ref{lst:goae-473-474-rule}.
Both rules check the procedure duration of \code{neuraxialAnesthesiaMsg}.
As the anesthesia lasts 11.5 hours, the framework correctly rejected \code{473} and accepted \code{474}.

Code \code{614} matches due to the respiration documentation.
Rule \ref{lst:goae-614-rule} checks for the presence of a respiration measurement that includes a pulse oximetry measurement.
The anesthetist performed a single measurement with an oximetry value of 95, which matches the rule.

The framework also derived the \goa{E} surcharge applicable for urgent treatment.
Rule \ref{lst:goae-E-rule} matched due to the high pain documented in the \code{pain} block during anamnesis.
The anesthesist documented a pain score of 8, which is greater than the minimum limit of 6 specified in the rule.

The treatment occurs at 3 am on a Wednesday night, outside any office hours.
This triggers surcharges \goa{A} and \goa{C}, which mutually exclude themselves.
The conflict resolution algorithm chose \goa{C} (320 points) over \goa{A} (70 points).
This results in the following bill.

%Codes:
%1 x GOÄ 1
%1 x GOÄ 614 (due to respiration)
%1 x GOÄ 473 (anesthesia)
%1 x GOÄ E (surcharge for urgency)
%1 x GOÄ C (surcharge for )


%-> respiration block mit pulseOximetry > 94
