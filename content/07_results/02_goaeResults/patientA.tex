\subsection{Patient Case 1: Multiple spine infiltrations due to back pain with obesity}\label{subsec:patient-a---multiple-spine-infiltrations-due-to-backpain-with-obesity}
The patient is a 57-year-old female suffering from chronic back pain.
The patient has two other visits with the same issue concerning back pain and was previously diagnosed with the so-called facet syndrome (\mete{M47.2}).

\paragraph{Anamnesis}
In the anamnesis stage the

\paragraph{Physical Examination}
In the physical examination stage the doctor conducts several examinations.
The doctor firstly checks for neurological issues.
She performs a orienting neurological examination and examines the posture and stand of the patient.
In both examinations she does not detect anything conspicuous.
Next, she examines the patient's spine mobility.
She performs a mobility test for the thoracic spine (\code{MobilityTestThoracicSpineBlock}) and the lumbar spine (\code{mobilityTestLumbarSpineBlock}) and detects existing mobility issues.
Additionally, the practitioner notes down the patient's nutrition status, leading to the results that the patient suffers from obesity.

\paragraph{Procedures}
The previous stages have revealed chronic back pain due to a diagnosed facet syndrome, missing neurological issues and obesity.
This is why she decides to perform non-operative and conservative treatment, namely a spine infiltration.
A spine infiltration is applicable to chronic and acute back pain issues to reduce the pain.
As the patient suffers from the facet syndrome, the doctor decides to perform facet injections in the spine infiltration procedure.
In a Facet Infiltration, the performer makes direct injections into the spine joints instead of the surrounding muscles.
She performs the following injections:
\begin{itemize}
    \item one nerve root injection on the left side of TH2 in the thoracic spine
    \item one facet injection on both sides of TH6 in the thoracic spine
    \item one facet injection on the right side of L1 in the lumbar spine
\end{itemize}

\paragraph{Difficulties and Inconveniences}
Overweight individuals have a thicker layer of subcutaneous fat, making locating veins, muscles and joins more difficult.
Additionally, doctors may need special syringes with longer needles to penetrate the fat layer to reach the underlying joint.


\subsubsection{Expected Results}
The billing for the treatment should compensate the doctor for the performed examinations and each single spine infiltration.
The challenging condition of the spine infiltration caused by the patient's nutrition status is a valid justification for a multiplier applied to the spine infiltration billing code.
The physical examination consisted of an orienting neurological examination and multiple spine mobility tests.


\subsubsection{Actual Results}
The framework was unable to derive \goa{1} due to the issues already described in subsection \ref{subsec:special-consultation-code-rules}.

The framework successfully derived \goa{8}, covering all the mobility tests and the neurological examination.
Other examination codes failed due to missing blocks.

The framework derived \goa{255} with the correct quantity of four.
Rule \ref{lst:goae-255-rule} checked for the existence of a \code{spineInfiltration} block which was given in the documentation.
The rule engine also computed the quantity by evaluating each child function for the corresponding localization field in the \code{spineInfiltration} object and returned the correct quantity.
Additionally, the framework detected the patient's obesity as a challenging factor for the infiltration,
sucessfully deriving the multiplier justification \addref.




%spine infiltration
%-> 4 spine infiltrations -> 4 x GOÄ 255
%multipliers:
%Erhöhter Aufwand bei SpineInfiltration aufgrund von Übergewicht

%Codes:
%1 x GOÄ 1
%1 x GOÄ 8
%4 x GOÄ 255 mit multiplier


