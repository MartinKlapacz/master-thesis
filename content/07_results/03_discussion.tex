\section{Discussion}\label{sec:discussion}

I want to express the following:

the experiments have shown that the framework and the rule language successfully derive most codes with a correct quantity.
however, the results have also shown concrete limitations.





As concluded in section \ref{sec:common-issues-with-modern-billing-automation-products} the central issue for automatic billing generation is the requirement for structured data.
The upper bound for the quality of the generation results is the completeness of the structured data that the framework has access to.

The reason for the missing billing codes in the previous experiments are condition features that the \RL currently does not provide.


One class code types is the set of services that focus on the patient-practitioner contact.
This includes \goa{1} and \goa{3}, \goa{4}, \goa{21} and \todo{orther}.
Rules in this code class have a high variety in used condition type.
Duration-related conditions are of high importance for this code class.
For example, we have seen that \goa{1} and \goa{3} rely on the anamnesis duration.
There are other codes and use cases for durations of specific stages and actions during a treatment, but those durations are unfortunately not part of the structured data.
The reason for that is that durations are not easily trackable in practice.
Measuring the time of an anamnesis, would require the front-end to start a timer when the anamnesis starts and to stop it when it finishes.
According to the engineers at \AV, we unfortunately cannot rely on specific user actions in the UI.
A user does not necessarily switch from the anamensis to the physical examination stage at the very moment when the anamnesis actually finishes.
Making the user manually enter durations into the user interface is neither a practical or user-friendly solution.
Durations the only information sources missing in the \AV data model.
\goa{21}, \goa{22}, \goa{23}, \goa{27}, \goa{30} are all consultation services that rely on consultation details that the \AV currently does not track,
either because tracking them automatically is infeasible or the current software version simply does not cover these service areas.




Another class of billing codes are examination related codes.
We have proven that the framework is indeed capable of successfully deriving this class of codes.
In this case, checking for existences of blocks using logical combinations has turned out to be extremely useful here.
This also requires a correct mapping of blocks to the examinations mentioned in the rule.

Finally, we have the class of billing codes relating to specific procedures and which represent the overwhelming majority of the GOÄ.
Writing rules for these codes is comparably simple.
Most of the rules in this class purely rely on a specific procedure which can be expressed using the \requiredProcedure condition.
The future production rule-base will probably contain multiple groups of procedure-related rules.
Each group references a procedure defined in the \AV model.
Each rule in this group checks has a slightly different conclusion code and checks for other conditions in the procedure block.


In cases where codes cannot be automatically derived, users need to manually add them in the billing stage.
As long as this issue persists, the billing framework should not be considered to generate a full and perfectly acceptable billing,
but rather a set of suggestions that might require manual updates.





