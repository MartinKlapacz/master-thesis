\chapter{Introduction}\label{ch:introduction}

The financial situation of clinics in Germany has been increasingly challenging in recent years.
One reason is the COVID-19 pandemic, which negatively affected the financial situation in many clinics \cite{Gandjour2023-rn}.
Additionally, inefficient medical billing practices further complicate the financial landscape.
Inaccurate medical billing leads to revenue losses, causing the financial difficulties to become worse \cite{MedicalBillingErrorsImpact}.

Up-coding uses billing codes that suggest more complex procedures than provided \cite{coustasse2021upcoding}.
This practice leads to higher reimbursements from insurance than what is justified by the actual services provided.
However, the financial damage caused by up-coding is substantial \cite{Joiner2024}.
Healthcare providers that engage in excessive up-coding practices may face disputes with insurers and legal consequences.
This makes up-coding a practice that does not pay off in the long run.
Healthcare providers should focus on right-coding, which is the practice of accurately coding medical services to reflect the care provided \cite{Joiner2024}.
Right-coding is the practice we focus on in this work.
In this work, we develop a billing optimization framework for clinics.

Firstly, we define billing optimization as follows.
\begin{description}
    \item Fee Optimization focuses on an appropriate revenue for practitioners while adhering to all legal regulations.
    The framework should cover all provided services, including those that are easy to miss out.
    Medical professionals should receive full compensation for their services.
    \item Process Optimization focuses on improvement of the overall billing process.
    Our goal is to automate code derivation to reduce manual efforts and time in selecting appropriate billing codes.
    An integral aspect of process optimization is enhancing the accuracy of billing, crucial for preventing disputes with insurers, delays in payments and loss of revenue.
\end{description}
Correct coding ensures that hospitals and healthcare providers receive quick reimbursed for their provided services.
Inaccurate coding can lead to delayed payments or lost revenue.
It's a key element in revenue cycle management, translating medical services into universal codes used for claims and reimbursement.

% research aims
The research aims to develop a sophisticated billing optimization framework that enables custom billing automation.
This involves designing a novel rule language, called \RL, that billing experts can use to express their knowledge in rule files.
The framework can use the knowledge encoded in those rules to generate billings for outpatient treatments.
The design of the framework and the rule language requires a deep understanding of the nuances and logic hidden in many billing codes.
It requires identifying the underlying information sources that billing codes rely on.
Research methods are conducting expert interviews, participating in a billing seminar, and performing an in-depth analysis of medical documents and billing catalogs.
The framework is an embedded component of a new microservice integrated into the \AV back-end architecture.
It intelligently uses the structured data that the \AV system collects.

% structure
This work is structured as follows.
In the first part, we discuss Germany's outpatient billing landscape.
We introduce processes and software solutions used in medical billing, followed by comparing and discussing the presented solutions.
Next, we talk about the regulatory landscape in Germany.
We present billing catalogs and their details that we must consider in the design of the billing framework.
The second part covers the theoretical aspects of rule-based systems, helpful for the design of the billing framework.
Next, we provide a formal specification of \RL.
Then, we cover the framework architecture, give an overview of crucial architectural design decisions, and present the final code derivation pipeline.
In the third part, present and validate the performance of the billing framework for realistic and non-trivial treatment cases provided by billing experts at \AV.
This also includes presenting the experimental rule base used for the simulations and the treatment cases.
For each case, we present the generated claims and compare them with the results required by the billing experts.
We illustrate the results and discuss the billing framework's strengths and limitations.
