\chapter{Introduction}\label{ch:introduction}

%\input{content/01_introduction/01_introduction}

critical financial situation of


Correct coding ensures that hospitals and healthcare providers receive reimbursed for their provided services.
Inaccurate coding can lead to delayed payments or lost revenue.
It's a key element in revenue cycle management, translating medical services into universal codes used for claims and reimbursement.


We firstly introduce current billing concepts.



In this work, we define the term billing optimization as follows.

Manual billing is an error-prone and tedious task.

- Define the term billing optimization:

fee optimization:
\begin{description}
    \item Fee Optimization: The billing should strictly cover all provided services, making use of all regularities
    \item Process Optimization: automatic derivation, less time spent on billing
\end{description}


- present structure of thesis

This work is structured as follows.

In the first part, we talk about the modern landscape of medical billing in Germany.
We talk about typical procceses and software solutions used in medical billing, followed by a comparison and discussion of the presented solutions.
Next, we talk about the regulatory landscape in Germany.
We present billing catalogs and their details that we must consider in the design of the billing framework.

In the second part, we firstly cover theoretical aspects of rule-based systems, followed by a formal specification of \RL.
Then we cover the framework architeture, give an overview of important architectural design decisions and present the code derivation pipeline.

In the third part, we simulate the treatment cases provided by billing experts at \AV.
This includes the presentation of the experimental rule-base used in for the simulations and the treatment cases.
For each case, we present the generated billing and compare it with the results required by the billing experts.
We explain the results and finally discuss and give reasons for framework strengths and limitations found in the experimental part.

