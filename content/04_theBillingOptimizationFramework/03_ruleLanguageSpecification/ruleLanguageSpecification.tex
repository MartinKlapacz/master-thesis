\section{Rule Language Specification}\label{sec:rule-language-specification}

Traditionally, billing experts create bills for treatments using their knowledge in this field.
They need to know the conditions which must hold and which make a billing code applicable.
This work introduces a new JSON-based domain specific rule language billing experts can use to encode this kind of knowledge.
The billing optimization framework loads and understands those rule files and uses their encoded knowledge to automatically derive billing codes for finalized treatments.
Section \todo{add section} describes this procedure in more detail.

As a computer scientist, I initially lacked important knowledge about medical billing.
This is why I conducted several expert interviews with Dr. Lorenz Thumann and Dr. Sahra Bojko, both Medical Managers at \AV.
In these interviews, I gained crucial knowledge in medical billing and collected formal requirements for the billing framework.
Other important sources of knowledge were the GOÄ as well as the EBM catalog.
Dr. Thumann provided me with more general insights on medical billing in Germany.
Dr. Bojko gave me useful feedback on my rule language designs while I supported her in writing first billing rules.

The development workflow of the rule language during my time at \AV worked was structured as follows:

\begin{enumerate}
    \item I regularly found previously unknown edge cases, either as the result of an expert interview or own research in one of the billing catalogs, that the billing framework but be able to handle.
    For those edge cases that could not be expressed in the rule language.
    \item I regularly conducted expert interviews and also did my research in the GOÄ and EBM catalogs.
    In doing so, I continually discovered relevant edge cases that the framework needed to cover.
    However, many of them were not expressible with the current version of the rule language.
    \item Based on those edge case, I established new formal requirements for the framework.
    \item I then specified proposals for the formal requirements, outlining how to extend the rule language in such a way that it can cover the respective edge case.
    This could be, for example, a new condition keyword or a more complex rule feature.
    \item Subsequently, I consulted with Sahra Bojko and presented the feature to her.
    It is essential that the language feature is intuitive and user-friendly for the medical professionals.
    If Dr. Bojko agreed with the proposals, I implemented them.
    \item At larger intervals, I met with my advisor Nicolas Jacob, who has a deep technical understanding of the \AVS.
    During these meetings and gave him updates on the new changes since our last meeting, and I received clarifications regarding the \AV architecture and data models.
\end{enumerate}

Firstly, we needed to specify the concrete use-cases for the rule language.
The initial plan was to implement two separate languages, one for statutory billing and one for private billing.
Section\ref{sec:billing-catalogs} introduces their billing catalogs and presents their differences and characteristics.

However, interviews with Dr. Lorenz Thumann have revealed further requirements for the billing framework.
This includes additional use-cases for the billing language.
The following list contains all rule types that the framework currently supports:
\begin{itemize}
    \item GOÄ billing rules for deriving GOÄ billing codes from private treatments
    \item EBM billing rules for deriving EBM billing codes from statutory treatments
    \item OPS code rules for deriving procedure codes from treatment documentation.
    OPS are a standardized
    OPS codes are relevant information for both GOÄ and EBM codes as well as multiplier justifications.
    \item Multiplier Justification rules for detecting challenging conditions in private treatments
    which serve as justifications for multipliers
    \item Special Flat Rate rules
    for detecting flat rates that can be applied to billing positions in the statutory treatment
\end{itemize}



The initial idea of having separate implementations for each rule type turned out to be not very scalable.
Each rule type has a set of relevant information sources that billing codes rely on.
Important information sources are patient age, the diagnoses, physical examination results and many more.
This does also mean that some information sources are relevant for multiple rule types.
This effectively indicates an n-to-m relationship between information sources and rule types.

For example, the patient age is highly relevant for GOÄ and EBM codes, Multiplier Justifications but not for OPS codes.
Previous allergies are relevant for Multiplier Justifications but not necessarily for GOÄ codes.
This is the reason, I decided to implement a rule type independent rule engine that provides all conditions used by any rule type.

This is why in the design of the billing framework information sources and condition types are primarily independent of any rule type.

\subsection{Information Scope}
Part of this work was
to identify billing-relevant information sources through expert interviews and research in the billing catalogs \addcite.
From a higher level perspective, all found information sources are part of the following domains:
\begin{itemize}
    \item The treatment documentation that contains all medical information of the treatment
    \item Treatment-independent patient information
    \item Information referring to the providing practitioner
    \item Treatment history containing all previous billing positions
\end{itemize}

In the following sections, we formally define the rule language and introduce its language features.
Additionally, we back up each language feature with use cases that lead to their implementation.
These requirements are the result of either conducted expert interviews or research in medical billing documents and catalogs.

The following code snipped \addref shows all upper level fields in the rule object.

\lstinputlisting[
language=json,
style=json,
caption={Rule root}
label={lst:rule-root}
]{code/rules/rule-root.json}

Subsection \ref{subsec:the-condition-predicate} introduces the \code{condition} and \code{code}  field.
Subsection \addref introduces the \code{quantity} field.
Subsection \addref introduces the \code{multiplier}, \code{targetEbmCode} and \code{targetGoaeCode} fields.

\subsection{The Condition Predicate}\label{subsec:the-condition-predicate}

\subsubsection{Block Content Related Sub-rules}

A medical treatment consists of multiple stages of patient care.
Each stage has its specific aims and tools.
Anamnesis,
physical examination and procedures represent different stages of patient care and are highly relevant for billing.

\paragraph{Anamnesis}
Anamnesis is the first stage of a treatment.
Its purpose is to collect a detailed medical history from the patient\cite{lino2021medical}.
More specifically, it is about understanding the patient's general health conditions,
their lifestyle and previous patients' diagnoses.
If possible, the practitioner reviews available medical records,
interviews the patient directly or using a questionnaire\cite{zhang2011anamnevis}.

\paragraph{Physical Examination}
Anamnesis is followed by the Physical Examination.
In this practical stage, the practitioner assesses the current patient's conditions,
looking for further information about the patient's issues\cite{seidel2010mosby}.
The practitioner decides based on patient's symptoms which exact examinations are conducted.
Common examinations are checking of vital sings like blood pressure, pulse and temperature.

\paragraph{Procedures}
The procedure stage contains the actual patient-specific interventions that \todo

\paragraph{Block Contents}
During Anamnesis,
Physical Examination and Procedures the practitioner enters medical treatment specific information into the \AVS.
The data is part of the treatment documentation and critical for billing.

The documentations of all three stages are filled out by the practitioner and are structured in sections,
cards and blocks.

For example, inside the physical examination there is the section \mete{abdomen},
which contains the\mete{liver} section.
Blocks are essentially small questionnaires for the practitioner with input fields to be filled out by the practitioner.
They play a huge role in the treatment documentation
and contain all relevant medical information specific to this conducted examination,
procedure anamnesis.
One example for a physical block in the \code{liver}section is\code{LiverPalpation}.
A liver palpation involves investigating the following questions

\begin{itemize}
    \item Is the liver palpable?
    \item Is the liver texture soft or tender?
    \item Is the liver surface smooth or knotty?
    \item Is tenderness present?
    \item Are pulsations present?
    \item Is a hepato jugular reflux present?
\end{itemize}

Each of the more than 100 block contents in the \AVS are represented by a dedicated Hibernate entity classes.
Each attribute of that class stores a questionnaire input entered by the practitioner.
Additionally, for each entity, there is a protobuf message enabling transmission of block content data from one service
to another one.

The protobuf message of \code{LiverPalpationMsg} looks like this:

\lstinputlisting[
    language=protobuf2,
    style=protobuf,
    caption={LiverPalpation protobuf messages}
]{code/proto/liverPalpation.proto}

The exact principle is applied throughout all the procedures and anamnesis as well.
A crucial part of the rule language is not only to check for block content availabilities in the treatment
but also to look inside the blocks and check for conditions on fields.
A GOÄ code might have as a condition that a liver palpation examination is part of the treatment and
the liver turned out to be palpable.
Or an OPS code should only be derived if a \code{ECG}procedure was provided and the
input field\code{telemetricExamination} inside its block has the value\code{true}.
Logical combinations of such single field conditions may also occur.
Expressing such conditions in a well-defined and user-friendly way is an important requirement for the rule language.

\paragraph{Block Content Field Types}
A block content can contain information of different input types.
The billing framework must handle and validate each input type differently.

The most basic field types are boolean flags and numeric inputs.
The \AVS uses the following scalar wrapper protobuf messages for them:

\lstinputlisting[
    language=protobuf2,
    style=protobuf,
    caption={Scalar wrapper types}
]{code/proto/scalarWrapper.proto}

The purpose of wrapping scalars in custom protobuf messages is to make scalar default values distinguishable from missing data.
Initializing a protobuf message without explicitly setting the value of field has the consequence that the field gets initialized with its default value.
The default value for \code{int32} is 0.
If a gRPC server receives a protobuf message with an \code{int32} field set to 0, it does not know if the field was purposely initialized with 0 or not set at all.
Wrapping integers in \code{Int32W} messages makes this distinguishable.


Liver tenderness can either be present or not.
This information is thus entered in a \code{BoolW]} field.
We use enumeration fields for inputs that have a limited number of predefined values
For example, the most frequently used enum type is \code{AbnormalitiesExaminationResult}.
It denotes the result of a specific examination, which can be either conspicuous or inconspicuous.

\lstinputlisting[
    language=protobuf2,
    style=protobuf,
    caption={\code{AbnormalitiesExaminationResult}}
]{code/proto/enum.proto}

Knowledge inputs are another important input type.
They serve similar purposes as enum types but are not hard-coded into the code base.
The system fetches them from a dedicated microservice that stores them in a MeiliSearch database.

The \AVS reuses the concept auf storing medical data as protobuf message questionnaires accross the anamnesis, physical examination and procedure stages.

Block content conditions are a powerful tool to make codes dependent on concrete medical information from anamnesis,
physical examination and procedures.


\subsubsection{Time Related conditions}

\subsubsection{Medical Coding Related Conditions}

\paragraph{minNumberOfDiagnosesInCurrentTreatment}
This field specifies the minimal number of diagnoses specified in this treatment.
One of the commonly used
In my research, I have learned that multimorbidity is a commonly used multiplier justification in the LMU dermatology.
Multimorbidity in medical terms defines the combination of two or more chronic medical conditions in an individual \cite{Reste2013The}.
These medical conditions may be can be a chronic disease, biopsychosocial factor or a somatic risk factor.
\code{minNumberOfDiagnosesInCurrentTreatment} is useful for implementing the multimorbidity multiplier justification rule.
The information source for this condition is \code{@InformationSourceTreatmentDiagnoses}.

\paragraph{requiredIcdCodes/requiredOpsCodes}

Any types of rules can require ICD and OPS codes as conditions.

To provide maximal flexibility I introduced the concept of a logical code matching trees.

\lstinputlisting[
    language=json,
    style=json,
    caption={Logical Code matching tree rules},
    label={lst:logical-code-matching-tree}
]{code/rules/specification/codeMatchingTree.json}

Code \mete{1} holds if the current treatment has an ICD10 diagnose that matches with \code{B20}.
Rule \mete{2} makes use of the logical \code{OR} operator and holds if the current treatment has a diagnose that matches with \code{R07.0}, \code{R07.1} or \code{L}.

We define term \"match\" as follows:
Let \( s_1 \) and \( s_2 \) be strings. We say \( s_1 \) matches with \( s_2 \) if and only if \( s_1 \) is a prefix of \( s_2 \).
\[
    s_1 \text{ matches with } s_2 \iff \exists k \in \mathbb{N} \cup \{0\} : s_1 = s_2[0:k]
\]

Note that \code{L} is not a single ICD10 code but an ICD10 chapter.
According to definition \addref, \mete{L} matches with every single ICD10 code in the L chapter as all of them start with the character \"L\".
This makes it easy to specify a large group of ICD10 codes in a logical code matching tree.
The ICD10 catalog is structured as a tree with each node being a prefix of its child nodes.
Nested nodes are chapters, sections and subsections.
Leaf nodes are the actual ICD10 codes.



\subsubsection{Patient Related Conditions}

\paragraph{minPatientAge/maxPatientAge}
\code{minPatientAge} and \code{maxPatientAge} are integer fields that enable restrictions on the patient age.
Both fields store a number of years.
You can combine both to require the patient age to be within a certain age interval.

In medical practice, the age of a patient can significantly influence the complexity of a treatment.
In cases involving very young or very elderly patients, practitioners may encounter patient age specific challenges.
Due to these challenges, treatments may require more time, special care or other additional resources.
As a result, practitioners may apply billing multipliers to compensate for the additional effort.
This is why you can use the \code{minPatientAge} and \code{maxPatientAge} fields implement these types of multiplier justification rules.
There are also concrete GOÄ codes such as \goa{K1} and \goa{K2} that require the patient age.
The patient age is also highly relevant for various EBM codes, but already exists as structured conditions in the EBM catalog.
This is why using patient age conditions in EBM rules is redundant.
The information source for these two conditions is \code{@InformationSourcePatientAge}.

\paragraph{newBorn}
\code{newBorn} is syntactic sugar and a special case of the \code{maxPatientAge} condition.
It is a bool field that denotes whether a patient is a newborn or not.
According to the GOÄ, babies that are at most 28 days old are newborns\addcite.
It is relevant for \goa{25: Initial newborn examination – if necessary including advice from the caregiver(s) –}, which is essentially a provision of a physical examination for newborns.
It can also be interesting for multiplier justifications.
The information source for these two conditions is \code{@InformationSourcePatientAge}.

\paragraph{patientSpeaksGerman/patientSpeaksEnglish}
\code{patientSpeaksGerman} and \code{patientSpeaksEnglish} are both boolean fields that denote whether the patients are able to communicate in the respective language.
Communication problems can increase the effort and time a treatment may require and are often used as justifications for multipliers which compensate for that.
\goa{4: taking a third-party medical history} covers an anemnesis that \todo{finish}
\code{patientSpeaksGerman} and \code{patientSpeaksEnglish} subscribe to the information source \code{@InformationSourcePatientLanguages}.

\paragraph{gender}
The \code{gender} condition is one of the most important condition types.
It is relevant for various EBM and GOÄ codes.
Similarly to patient age, gender conditions are already a structured part of the EBM catalog and do not need to be used in EBM rules.
\goa{27}, \goa{28},  GOÄ codes in section \mete{H}, Obstetrics and Gynecology, and GOÄ codes in \mete{Urology} are gender-specific.
Rules implementing these codes can use the \code{gender} condition.
It supports the string values \code{\"female\"}, \code{\"male\"} and \code{\"diverse\"}.
\code{gender} subscribes to the information source \code{@InformationSourcePatientGender}.


\paragraph{isPregnant}
The \code{isPregnant} condition is syntactic sugar for a specific case of the \code{requiredAnamnesisBlocks}
The following rules are equivalent.

\lstinputlisting[
    language=json,
    style=json,
    caption={\code{isPregnant} rule},
    label={lst:is-pregnant}
]{code/rules/specification/performerIsDoctor.json}

A pregnancy can be reason for additionally required care or services during a treatment that is not related to the patient's pregnancy.
This is why it can be useful in multiplier justification rules.
It is also useful for \goa{23} which is applicable for an initial pregnancy-related examination.
\code{isPregnant} subscribes to the information source \code{@InformationSourceAnamnesisBlocks}.

\paragraph{minNumberOfAllergies}
\code{minNumberOfAllergies} is a field that specifies the minimal number of allergies a patient requires.
From medical experts at \AV I have learned that a high number of allergies can make treatments more challenging for several reasons:
\begin{itemize}
    \item It may reduce the number of medical options as medications contain allergens or can produce them in specific environments.
    \item It can increase the complexity of the diagnostic process, as allergy reactions can mimic symptoms of other diagnoses.
    \item Patients with a high number of allergies require increased caution and care.
    Practitioners need to spend more time for history review and patient monitoring.
\end{itemize}
This makes \code{minNumberOfAllergies} a useful condition for multiplier justifications.
It subscribes to the information source \code{@InformationSourcePatientAllergies}

\subsubsection{Previous Occurrence related Conditions}

\subsubsection{Service Performer related Conditions}

\paragraph{performerIsDoctor}
The \code{performerIsDoctor} condition is a boolean condition field that allows to impose conditions on the practitioner type.
The current version of the \AVS supports the following practitioner types:

\lstinputlisting[
    language=protobuf2,
    style=protobuf,
    caption={\code{AbnormalitiesExaminationResult}},label={lst:AbnormalitiesExaminationResult}]{code/proto/roleTypes.proto}

Given the rules in code snippet \ref{lst:performerIsDoctor}, \code{code1} would hold if the practitioner had the role \code{ROLE\_TYPE\_DOCTOR}
or \code{ROLE\_TYPE\_HEAD\_DOCTOR}.
Otherwise, \code{code2} would hold.
This condition is relevant for \goa{1} and \goa{2}, which are only applicable by doctors.

\lstinputlisting[
    language=protobuf2,
    style=protobuf,
    caption={\code{AbnormalitiesExaminationResult}},
    label={lst:performerIsDoctor}
]{code/rules/specification/performerIsDoctor.json}


\subsection{Numeric Functions}\label{subsec:numeric-functions}

\section{Quantity Functions}\label{sec:numeric-functions}

In the previous chapter we introduced the concept of a billing rule.

Not only which codes are part of the billing but also their quantity is highly important information.
In the previous section, we explained how the billing framework derives a set of applicable billing codes.
The result data type was essentially a set of billing codes.
In this section, we generalize the framework in such a way that it returns each code in combination with its quantity.

We extend the general rule structure by introducing the concept of quantity functions.
Similarly to rule conditions, quantity functions are evaluatable subcomponents of a rule.
The framework evaluates quantity functions of rules for a\code{RuleEvaluationInput}.
The result is an integer that indicates how often it should be billed in the billing.

\begin{equation}
    \label{eq:quantity-function-natural}
    f_{quantity}: \code{RuleEvaluationInput} \longrightarrow \mathbb{N}
\end{equation}

Billing code quantities typically rely on two types of billing information.
Subsection\ref{subsec:quantity-derivation-from-procedure-information} and\ref{subsec:quantity-derivation-from-durations} explain these sources.

\subsection{Quantity Derivation From Procedure Information}\label{subsec:quantity-derivation-from-procedure-information}

\goa{255: Injection, intra-articular or perineural}\cite{hermanns2013bemessung} is an example of a code that is likely to be billed multiple times in a single billing.

The so-called Spine Infiltration is a procedure, that includes injections into spine nerves.
Billing experts use \goa{255} for this medical service.
Procedure block \code{SpineInfiltrationMsg} represents this procedure.
However, \goa{255} is typically billed once for each injection the practitioner actually provided.
Therefore, a spine infiltration can include multiple injections at different localizations.

This is where a new requirement for the billing framework occurred.
The system should not only be able to derive \goa{255} for a provided spine infiltration, but should also be able to derive the correct quantity of \goa{255}.

To illustrate this example, it is worth mentioning that the human spine consists of three primary regions:
\begin{itemize}
    \item The cervical spine, the thoracic spine, and the lumbar spine.
    The cervical spine comprises the uppermost part of the spine, which is the neck and head.
    \item The thoracic spine, located in the middle, has twelve vertebrae attached to the rib cage, providing stability and structure to the upper body.
    \item Lastly, the lumbar spine at the lower back is made up of five larger vertebrae, designed to bear the body's weight and provide flexibility and movement.
\end{itemize}
Each region has well-defined localizations that can be targets for injections.
The \AVS stores the data of a spine infiltration in a \code{SpineInfiltrationMsg}procedure block.

The following code snipped displays a simplified version of it.
\lstinputlisting[
    language=protobuf3,
    style=protobuf,
    caption={Relevant localizations in a spine infiltration}
    label={lst:spineInfiltration},
]{code/proto/spineInfiltration.proto}


\code{SpineInfiltrationMsg} has a nested message\code{NerveRootBlockMsg}.
\code{NerveRootBlockMsg} contains the messages,\code{CervicalSpineMsg}\code{ThoracicSpineMsg} and\code{LumbarSpineMsg}.
These messages represent the before-mentioned sections of the human spine.
Each variable in these nested messages is of type \code{Laterality}and represents injections at the respective localization alongside the spine.
The laterality of an injection refers to the specific side or sides where the injection is administered in the spine localization.
This can be either on the left side, the right side, or on both sides.

To get the total number of injections, the system would need to peek into the nested messages,\code{CervicalSpineMsg}\code{ThoracicSpineMsg} and\code{LumbarSpineMsg} and read the total number of injections from the laterality values.
Additionally, the framework needs to be able to understand that \code{LATERALITY\_NONE} refers to zero and \code{LATERALITY\_LEFT} as well as\code{LATERALITY\_RIGHT} refer to one injection.
\code{LATERALITY\_BOTH} refers to two injections.
It is important to assign quantities to enum values.

\subsection{Quantity Derivation From Durations}\label{subsec:quantity-derivation-from-durations}

Many GOÄ codes specify not only a service content but also a time range.
If provided services exceed that time range the respective code can be billed more than once.

\goa{21: Incoming human genetic counseling, per commenced half-hour and session} is an example for such a code.
If the counseling takes one hour and 10 minutes, \goa{21} can be billed 3 times.
Similar codes are \goa{61: Assistance in the medical service of another doctor (Assistance), per commenced half-hour}
and \goa{85: Written expert opinion involving an effort exceeding the usual extent – possibly with scientific justification –, per commenced hour of work}
Codes that with that characteristic often end with \("\)per commenced hour\("\) or \("\)per commenced half-hour\("\) to indicate the time range.
Those are comparably simple cases.

\mete{Section D: Anesthesia Services} contains a specific class of edge case that use time ranges.

\goa{462: Combined anesthesia with endotracheal intubation, up to one hour} and \goa{463: Combined anesthesia with endotracheal intubation, each additional commenced half-hour}
both denote the same procedure provided, but as their names suggest, they distinguish between different periods of service provision.
\goa{462} is applicable only once per procedure.
On the other hand, code \goa{463} covers each additional half-hour beyond the initial hour.
Its billing number is variable and depends on the total duration of the anesthesia.
\goa{463} excludes the first hour of service provision, already covered under code \goa{462}.

The GOÄ distinguishes between \goa{462} and \goa{463} because of the following reasons.
\goa{462} covers both the induction and the maintenance of anesthesia for the first hour.
This is the most critical phase involving continuous administration of medication to make sure the patient remains asleep.
The time after the first hour typically requires only maintenance of anesthesia and is therefore less critical.
\goa{463} applies to subsequent commenced half-hours after the first hour and is therefore lower priced than \goa{462}.


Anesthesia services frequently use this pattern of code groups referring to the same provided procedure but distinguishing between service periods.
Equivalent examples are \goa{460/461}, \goa{476/477} and \goa{478/479}.
\goa{473, 474 and 475} are a more special example.
They refer to an initiation and monitoring of a continuous subarachnoid spinal anesthesia, again distinguishing between durations.
\goa{473} applies to durations shorter than five hours, \goa{474} to the subsequent five hours and \goa{475} applies once for the second and every subsequent day.

These examples lead to concrete requirements for the billing framework and underline the importance of \"quantity functions\".
They must be expressive enough to represent and manage the nuances of these examples.
Section\ref{sec:quantity-function-specification} illustrates how users can express these cases using the rule language.

\subsection{Partial Billing Quantitys}\label{subsec:partial-billing-quantitys}
In very special cases billing codes apply with a reduced quantity of points.

There are at least two cases:
\goa{D} is a surcharge for services provided on saturday, sunday and on holidays.
If service provision happens during an opening hour, \goa{D} only applies with half the fee, though.
This is also the case for surcharges \goa{E}, \goa{F}, \goa{G} and \goa{H} if \goa{51} is part of the billing, as well.

The framework must be able to express these edge cases but does not allow any point value manipulations of billing codes.
It solves this issue using quantity functions.
In those cases the quantity function must be able to evaluate to 0.5 which results in the same fee reduction as required by the GOÄ specification.

This requirement changes the formal definition given in formula \ref{eq:quantity-function-natural} to:
\begin{equation}
    \label{eq:quantity-function-real}
    f_{quantity}: \code{RuleEvaluationInput} \longrightarrow \mathbb{R}^{+} \\ \left{0\right}
\end{equation}
Quantity function now evaluate


\input{content/04_theBillingOptimizationFramework/03_ruleLanguageSpecification/02_numericFunctions/02_theMultiplierBlock}

