\section{Billing Relevant Information Sources}\label{sec:billing-relevant-information-sources}



Additionally,
it must be assured that the\AVS actually tracks and stores all information relevant for billing as structured data.
For that, the\AVS might require further adjustments and updates.

On one hand, the treatment documentation must contain all medically relevant data,
but now it also needs to cover billing-related information as well.





\subsection{Block Contents}
A medical treatment consists of multiple stages of patient care.
Each stage has its specific aims and tools.
Anamnesis,
physical examination and procedures represent different stages of patient care and are highly relevant for billing.

\subsubsection{Anamnesis}
Anamnesis is the first stage of a treatment.
Its purpose is to collect a detailed medical history from the patient\cite{lino2021medical}.
More specifically, it is about understanding the patient's general health conditions,
their lifestyle and previous patients' diagnoses.
If possible, the practitioner reviews available medical records,
interviews the patient directly or using a questionnaire\cite{zhang2011anamnevis}.

\subsubsection{Physical Examination}
Anamnesis is followed by the Physical Examination.
In this practical stage, the practitioner assesses the current patient's conditions,
looking for further information about the patient's issues\cite{seidel2010mosby}.
The practitioner decides based on patient's symptoms which exact examinations are conducted.
Common examinations are checking of vital sings like blood pressure, pulse and temperature.

\subsubsection{Procedures}
The procedures stage contains the actual patient specific interventions that \todo

\subsection{Block Contents}
During Anamnesis,
Physical Examination and Procedures the practitioner enters medical treatment specific information into the \AVS.
The data is part of the treatment documentation and critical for billing.

The documentations of all three stages are filled out by the practitioner and are structured in sections,
cards and blocks.

For example, inside the physical examination there is the section \mete{abdomen},
which contains the\mete{liver} section.
Blocks are essentially small questionnaires for the practitioner with input fields to be filled out by the practitioner.
They play a huge role in the treatment documentation
and contain all relevant medical information specific to this conducted examination,
procedure anamnesis.
One example for a physical block in the \code{liver}section is\code{LiverPalpation}.
A liver palpation involves investigating the following questions

\begin{itemize}
    \item Is the liver palpable?
    \item Is the liver texture soft or tender?
    \item Is the liver surface smooth or knotty?
    \item Is tenderness present?
    \item Are pulsations present?
    \item Is a hepato jugular reflux present?
\end{itemize}

Each of the more than 100 block contents in the \AVS are represented by a dedicated Hibernate entity classes.
Each attribute of that class stores a questionnaire input entered by the practitioner.
Additionally, for each entity, there is a protobuf message enabling transmission of block content data from one service
to another one.

The protobuf message of \code{LiverPalpationMsg} looks like this:

\lstinputlisting[
    language=protobuf2,
    style=protobuf,
    caption={LiverPalpation protobuf messages}
]{code/proto/liverPalpation.proto}

The exact principle is applied throughout all the procedures and anamnesis as well.
A crucial part of the rule language is not only to check for block content availabilities in the treatment
but also to look inside the blocks and check for conditions on fields.
A GOÄ code might have as a condition that a liver palpation examination is part of the treatment and
the liver turned out to be palpable.
Or an OPS code should only be derived if a \code{ECG}procedure was provided and the
input field\code{telemetricExamination} inside its block has the value\code{true}.
Logical combinations of such single field conditions may also occur.
Expressing such conditions in a well-defined and user-friendly way is an important requirement for the rule language.

\subsection{Block Content Field Types}
A block content can contain information of different input types.
The billing framework must handle and validate each input type differently.

The most basic field types are boolean flags and numeric inputs.
The \AVS uses the following scalar wrapper protobuf messages for them:

\lstinputlisting[
    language=protobuf2,
    style=protobuf,
    caption={Scalar wrapper types}
]{code/proto/scalarWrapper.proto}

Liver tenderness can either be present or not.
This information is thus entered in a \code{BoolW]} field.
We use enumeration fields for inputs that have a limited number of predefined values
For example, the most frequently used enum type is \code{AbnormalitiesExaminationResult}.
It denotes the result of a specific examination, which can be either conspicuous or inconspicuous.

\lstinputlisting[
    language=protobuf2,
    style=protobuf,
    caption={\code{AbnormalitiesExaminationResult}}
]{code/proto/enum.proto}

Another important input type are knowledge inputs.
They serve similar purposes as enum types but are not hard-coded into the code base.
The system fetches them from a dedicated microservice that stores them in a MeiliSearch database.
Enum field


The \AVS uses block content messages to store user inputs of medical questionnaires filled out by practitioners.

The \AVS uses this concept of blocks as a medical questionnaires storing anamnesis, physical examination and procedure
data as protobuf messages.

Block content conditions are a powerful tool to make codes dependent on concrete medical information from anamnesis,
physical examination and procedures.
They are one of multiple condition types in a rule and are called \code{BlockContentSubRules}


There is another condition
as well that are related to information outside the treatment such as general patient information,
that get

\subsection{EBM billing rules}\label{subsec:ebm-billing-rules}
As described in section\ref{sec:ebm-conditions} the Ebm catalog already contains structured,
treatment content unrelated conditions.
EBM rules therefore do not need to include them.
However, what is left are treatment-specific conditions.
Rules

\subsection{OPS code rules}\label{subsec:ops-code-rules}
OPS codes are identifiers of medical procedures.
They are completely independent of patient information and their previous medical history.
So in order to derive them automatically, they only need the procedure block as an input.

\subsection{GOÄ billing rules}\label{subsec:goa-billing-rules}


\subsection{Special Flat Rates}\label{subsec:special-flat-rates}

