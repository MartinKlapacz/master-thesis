\subsection{Quantity Functions}\label{subsec:quantity-functions}
In the previous chapter, we introduced the concept of a billing rule.
Not only which codes are part of the billing, but also their quantity is critical information.
In the previous section, we explained how the billing framework derives a set of applicable billing codes.
In this section, we generalize the framework such that it returns each code in combination with its quantity.
We extend the general rule schema by introducing the concept of quantity functions.
Similarly to rule conditions, quantity functions are subcomponents of a rule.
The framework evaluates quantity functions of rules for a \REI.
Formally, quantity functions are functions that map an \REI to a natural number.
\begin{equation}
    \label{eq:quantity-function-natural}
    f_{quantity}: \code{RuleEvaluationInput} \longrightarrow \mathbb{N}
\end{equation}
%Billing code quantities typically rely on two types of billing information.
%Subsection \ref{subsubsec:quantity-derivation-from-procedure-information} and \ref{subsec:quantity-derivation-from-durations} explain these sources.

\subsubsection{Quantity Derivation From Procedure Information}\label{subsubsec:quantity-derivation-from-procedure-information}
The so-called Spine Infiltration is a procedure, that includes injections into spine nerves.
Billing experts use \goa{255} for this medical service.
Procedure block \code{SpineInfiltrationMsg} represents this procedure.
However, \goa{255} typically applies once for each spine injection the practitioner actually provided.
Therefore, a spine infiltration can include multiple injections at different localizations.
The system should be able to derive the correct quantity of \goa{255} codes as the number of injections.
To illustrate this example, it is worth mentioning that the human spine consists of three primary regions \cite{BOGDUK2016675}:
\begin{itemize}
    \item The cervical spine, the thoracic spine, and the lumbar spine.
    The cervical spine comprises the uppermost part of the spine, which is the neck and head.
    \item The thoracic spine, located in the middle, has twelve vertebrae attached to the rib cage, providing stability and structure to the upper body.
    \item Lastly, the lumbar spine at the lower back is made up of five larger vertebrae, designed to bear the body's weight and provide flexibility and movement.
\end{itemize}
Each region has well-defined localizations that can be targets for injections.
The \AVS stores the data of a spine infiltration in a \code{spineInfiltration} procedure block.

The following code snipped displays a simplified version of it.
\lstinputlisting[
    language=protobuf3,
    style=protobuf,
    caption={Relevant localizations in a spine infiltration},
    label={lst:spineInfiltration},
]{code/proto/spineInfiltration.proto}


\code{SpineInfiltrationMsg} has a nested message\code{NerveRootBlockMsg}.
\code{NerveRootBlockMsg} contains the messages, \code{CervicalSpineMsg}, \code{ThoracicSpineMsg}, and \code{LumbarSpineMsg}.
These messages represent the before-mentioned sections of the human spine.
Each variable in these nested messages is of type \code{Laterality} and represents injections at the respective localization alongside the spine.
The laterality of an injection refers to the specific side or sides where the practitioner performs the injection.
This can be either on the left, right, or both sides.

To get the total number of injections, the system would need to peek into the nested messages \code{CervicalSpineMsg}, \code{ThoracicSpineMsg} and\ code{LumbarSpineMsg} and compute the total number of injections from the laterality values.
The framework must understand that \code{LATERALITY\_NONE} refers to zero and \code{LATERALITY\_LEFT} as well as\code{LATERALITY\_RIGHT} refer to one injection.
\code{LATERALITY\_BOTH} refers to two injections.
It is important to assign quantities to enum values.
Listing \ref{lst:quantity-function-basics} illustrates a rule with a simple quantity function expression.
\lstinputlisting[
    language=json,
    style=json,
    caption={Relevant localizations in a spine infiltration},
    label={lst:quantity-function-basics},
]{code/rules/specification/quantity-function-basics.json}
The value of the \code{quantity} field is an object with a \code{fieldPath} key and a \code{evaluationMode}.
The value of the \code{fieldPath} key is a list with the first entry being the procedure name.
The subsequent entries in the list is the path to the referred field.
The value of \code{evaluationMode} is a mapping that assigns enum values to integers as described before.
The framework evaluates this quantity function expression by applying this map to the value of the \code{c4} field.

Similarly to logical expression trees for block contents, we can also combine quantity functions in tree-like structures as shown in listing \ref{lst:quantity-function-combination}.
\lstinputlisting[
    language=json,
    style=json,
    caption={Relevant localizations in a spine infiltration},
    label={lst:quantity-function-combination},
]{code/rules/specification/quantity-function-combination.json}
The language supports \code{sum}, \code{mul}, \code{max} and \code{min} operators

\subsubsection{Quantity Derivation From Durations}\label{subsubsec:quantity-derivation-from-durations}
There are some GOÄ codes that have a quantity that depends on the duration of the referenced service.

\goa{21: Incoming human genetic counseling, per commenced half-hour and session} is an example for such a code.
If the counseling takes one hour and 10 minutes, \goa{21} can be applies with a quantity of three.
Similar codes are \goa{61: Assistance in the medical service of another doctor (Assistance), per commenced half-hour}
and \goa{85: Written expert opinion involving an effort exceeding the usual extent – possibly with scientific justification –, per commenced hour of work}
Those are comparably simple cases.

\mete{Section D: Anesthesia Services} contains a specific class of edge case that use time ranges.

\goa{462: Combined anesthesia with endotracheal intubation, up to one hour} and \goa{463: Combined anesthesia with endotracheal intubation, each additional commenced half-hour}
both denote the same procedure provided, but as their names suggest, they distinguish between different periods of service provision.
\goa{462} is applicable only once per procedure.
On the other hand, code \goa{463} covers each additional half-hour beyond the initial hour.
Its billing number is variable and depends on the total duration of the anesthesia.
\goa{463} excludes the first hour of service provision, already covered under code \goa{462}.

The GOÄ distinguishes between \goa{462} and \goa{463} because of the following reasons.
\goa{462} covers both the induction and the maintenance of anesthesia for the first hour.
This is the most critical phase involving continuous administration of medication to make sure the patient remains asleep.
The time after the first hour typically requires only maintenance of anesthesia and is therefore less critical.
\goa{463} applies to subsequent commenced half-hours after the first hour and is therefore lower priced than \goa{462}.

Anesthesia services frequently use this pattern of code groups referring to the same provided procedure but distinguishing between service periods.
Equivalent examples are \goa{460/461}, \goa{476/477} and \goa{478/479}.
\goa{473, 474 and 475} are a more special examples.
They refer to an initiation and monitoring of a continuous subarachnoid spinal anesthesia, again distinguishing between durations.
\goa{473} applies to durations shorter than five hours, \goa{474} to the subsequent five hours and \goa{475} applies once for the second and every subsequent day.

We can express such situations as follows:
\lstinputlisting[
    language=json,
    style=json,
    caption={Relevant localizations in a spine infiltration},
    label={lst:quantity-function-duration},
]{code/rules/quantityfunctions/per-n-hours.json}
The framework evaluates quantity function expressions with \code{blockName} and \code{perNMinutes} / \code{perNHours} fields by dividing the duration of the procedure block in minutes by the value of \code{perNMinutes} / \code{perNHours}.

