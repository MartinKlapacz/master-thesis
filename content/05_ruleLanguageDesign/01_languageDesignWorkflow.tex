\section{Language Design Workflow}\label{sec:language-design-workflow}
Traditionally, billing experts create bills for treatments using their knowledge in this field.
They need to know the conditions which must hold and which make a billing code applicable.
This work introduces \RL, a new JSON-based domain specific rule language for medical billing experts.
The billing optimization framework loads and understands those rule files and uses their encoded knowledge to automatically derive billing codes for finalized treatments.
Section \todo{add section} describes this procedure in more detail.

As a computer scientist, I initially lacked important knowledge about medical billing.
This is why I conducted several expert interviews with Dr. Lorenz Thumann and Dr. Sahra Bojko, both Medical Managers at \AV.
In these interviews, I gained crucial knowledge in medical billing and collected formal requirements for the billing framework.
Dr. Thumann provided me with more general insights on medical billing in Germany.
Dr. Bojko gave me useful feedback on my rule language designs while I supported her in writing first billing rules with \RL.
Other important sources of knowledge I made use of are the publicly available GOÄ and EBM catalog.
I also took part in an online GOÄ seminar provided by the PVS Forum.
The PVS Forum offers a wide range of seminars and training programs for medical professionals.
The GOÄ seminar, held by Dr. med. Markus Molitor,
consisted of three distinct sessions and covering GOÄ basics such as GOÄ paragraphs, services types and concrete GOÄ case studies.

The development workflow of the rule language during my time at \AV had the following structure:

\begin{enumerate}
    \item I regularly found previously unknown edge cases, either as the result of an expert interview or own research in one of the billing catalogs, that the billing framework but be able to handle.
    For those edge cases that could not be expressed in the rule language.
    \item I regularly conducted expert interviews and also did my research in the GOÄ and EBM catalogs.
    In doing so, I continually discovered relevant edge cases that the framework needed to cover.
    However, many of them were not expressible with the current version of the rule language.
    \item Based on those edge case, I established new formal requirements for the framework.
    \item I then specified proposals for the formal requirements, outlining how to extend the rule language in such a way that it can cover the respective edge case.
    This could be, for example, a new condition keyword or a more complex rule feature.
    \item Subsequently, I consulted with Sahra Bojko and presented the feature to her.
    It is essential that the language feature is intuitive and user-friendly for the medical professionals.
    If Dr. Bojko agreed with the proposals, I implemented them.
    \item At larger intervals, I met with my advisor Nicolas Jacob, who has a deep technical understanding of the \AVS.
    During these meetings and gave him updates on the new changes since our last meeting, and I received clarifications regarding the \AV architecture and data models.
\end{enumerate}

Firstly, we needed to specify the concrete use-cases for the rule language.
The initial plan was to implement two separate languages, one for statutory billing and one for private billing.
Section\ref{sec:billing-catalogs} introduces their billing catalogs and presents their differences and characteristics.

However, interviews with Dr. Lorenz Thumann have revealed further requirements for the billing framework.
This includes additional use-cases for the billing language.
The following list contains all rule types that the framework currently supports:
\begin{itemize}
    \item GOÄ billing rules for deriving GOÄ billing codes from private treatments
    \item EBM billing rules for deriving EBM billing codes from statutory treatments
    \item OPS code rules for deriving procedure codes from treatment documentation.
    OPS codes are not inherently crucial for billing but are part of medical documentation.
    \AV decided to use the billing service internal rule derivation engine to automatically derive OPS codes as part of medical documentation.
    \item Multiplier Justification rules for detecting challenging conditions in private treatments
    which serve as justifications for multipliers.
    \item Flatrate rules for special treatments at university clinics
\end{itemize}


The initial idea of having separate implementations for each rule type turned out to be not very scalable.
Each rule type has a set of relevant information sources that billing codes rely on.
Important information sources are patient age, the diagnoses, physical examination results and many more.
This does also mean that some information sources are relevant for multiple rule types.
This effectively indicates an n-to-m relationship between information sources and rule types.
For example, the patient age is highly relevant for GOÄ and EBM codes, Multiplier Justifications but not for OPS codes.
Previous allergies are relevant for Multiplier Justifications but not necessarily for GOÄ codes.
This is the reason, I decided to implement a rule type independent rule engine that provides all conditions that might be relevant for any rule type.
Thus, I designed the billing framework in such a way that, information sources and condition types are independent of any rule type.
The final result is a universal rule language employable in multiple domains including, non-billing related ones.

