\section{Protobuf Block Contents}\label{sec:protobuf-block-contents}

A medical treatment consists of multiple stages of patient care.
Each stage has its specific aims and tools.
Anamnesis, physical examination and procedures are treatment stages relevant for billing and mapped by the \AVS.

\paragraph{Anamnesis}
Anamnesis is the first stage of a treatment.
It aims to collect a detailed medical history from the patient\cite{lino2021medical}.
It is about understanding the patient's general health conditions,
lifestyle, and previous diagnoses.
If possible, the practitioner reviews available medical records,
interviews the patient directly or using a questionnaire\cite{zhang2011anamnevis}.

\paragraph{Physical Examination}
The Physical Examination follows anamnesis.
In this practical stage, the practitioner assesses the current patient's condition,
looking for further information about the patient's issues\cite{seidel2010mosby}.
Based on the patient's symptoms, the practitioner decides which examinations they perform.

\paragraph{Procedures}
In the procedures stage the practitioner performs medical therapies based on the outcomes of the previous stages.
The \AVS defines a procedure as a medical measure represented by an OPS code.
A surgery consists of one or more procedures.
At this stage, the practitioner documents medical details about the provided procedures.

\paragraph{Block Contents}
As part of the Anamnesis, Physical Examination and Procedures stage,
the practitioner enters medical treatment specific information into the \AVS.
All three stages have separate documentation with a tree-like structure consisting of sections, cards and blocks.
Blocks are topic-related questionnaires for to be filled out by the user.
\code{LiverPalpation} is an example for an example for a physical block in the \code{liver} section.
A classic liver palpation examination involves answering the following questions \cite{wolf1990evaluation}:
\begin{itemize}
    \item Is the liver palpable?
    \item Is the liver texture soft or tender?
%    \item Is the liver surface smooth or knotty?
%    \item Is tenderness present?
    \item Are pulsations present?
    \item Is a hepato jugular reflux present?
\end{itemize}
The physical examination block \code{liver} stores the outcomes of such an examination.
The system stores every block directly in the database in a structured way, making it accessible for other components of the system.
Each block has a Protobuf representation used for transmission between services, that we call protobuf block contents.
For example, the protobuf message of \code{LiverPalpationMsg} looks like this:
\lstinputlisting[
    language=protobuf2,
    style=protobuf,
    caption={LiverPalpation protobuf messages},
    label={lst:lstinputlisting2}]
{code/proto/liverPalpation.proto}
Data stored in such blocks is crucial for billing.
The billing framework must provide a way to express conditions on proto block contents in a treatment.
This does not only include conditions on the existence of blocks in a treatment, but also conditions on protobuf block content fields.
Paragraph \ref{par:block-content-field-types} covers this type of conditions in more detail.
