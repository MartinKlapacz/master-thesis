\section{Protobuf Block Contents}\label{sec:protobuf-block-contents}

A medical treatment consists of multiple stages of patient care.
Each stage has its specific aims and tools.
Anamnesis, physical examination and procedures are treatment stages relevant for billing and mapped by the \AVS.

\paragraph{Anamnesis}
Anamnesis is the first stage of a treatment.
Its purpose is to collect a detailed medical history from the patient \cite{lino2021medical}.
More specifically, it is about understanding the patient's general health conditions,
their lifestyle and previous patient's diagnoses.
If possible, the practitioner reviews available medical records,
interviews the patient directly or using a questionnaire \cite{zhang2011anamnevis}.

\paragraph{Physical Examination}
Anamnesis is followed by the Physical Examination.
In this practical stage, the practitioner assesses the current patient's conditions,
looking for further information about the patient's issues \cite{seidel2010mosby}.
The practitioner decides based on patient's symptoms which exact examinations are conducted.
Common examinations are checking of vital sings like blood pressure, pulse and temperature.

\paragraph{Procedures}
The procedure stage contains the actual patient-specific interventions that \todo

\paragraph{Block Contents}
As part of the Anamnesis, Physical Examination and Procedures stage,
the practitioner enters medical treatment specific information into the \AVS.
All three stages have separate documentation with a tree-like structure consisting of sections, cards and blocks.
Blocks are topic-related medical questionnaires that the practitioner selects and fills out.

%For example, the physical examination contains the section \mete{abdomen} that contains the \mete{liver} card.
%The \mete{liver} card contains the blocks \mete{liverPalpation} and \mete{liverSize}.
Blocks are essentially small questionnaires for the practitioner with input fields to be filled out by the practitioner.
\code{LiverPalpation} is an example for an example for a physical block in the \code{liver} section.
A classic liver palpation examination involves answering the following questions \cite{wolf1990evaluation}:
\begin{itemize}
    \item Is the liver palpable?
    \item Is the liver texture soft or tender?
%    \item Is the liver surface smooth or knotty?
%    \item Is tenderness present?
    \item Are pulsations present?
    \item Is a hepato jugular reflux present?
\end{itemize}
%Within the \AVS each block consists of meta-data such as timestamps, ids as well as a block content sub-object.
%The \AVS defines several hundreds block contents for physical examination, anamnesis and procedures blocks.
%It uses the object-relationship-mapping Hibernate to map all block content class definitions to database tables.
%Each attribute of that class stores a questionnaire input entered by the practitioner.


The system stores every block directly in the database in a structured way, making it accessible for other components of the system.
Each block has a Protobuf representation used for transmission between services, that we call protobuf block contents.
For example, the protobuf message of \code{LiverPalpationMsg} looks like this:
\lstinputlisting[
    language=protobuf2,
    style=protobuf,
    caption={LiverPalpation protobuf messages},
    label={lst:lstinputlisting2}]
{code/proto/liverPalpation.proto}

Data stored in such blocks is crucial for billing.
The billing framework must provide a way to express conditions on proto block contents in a treatment.
This does not only include conditions on the existence of blocks in a treatment, but also conditions on protobuf block content fields.
Subsection \addref covers this type of conditions in detail.
But firstly, we need to give an overview of relevant protobuf

%\paragraph{Block Content Field Types}
%A block content can contain information of different input types.
%The billing framework must handle and validate each input type differently.
%The most basic field types are boolean flags and numeric inputs.
%The \AVS uses the following scalar wrapper protobuf messages for them:
%
%\lstinputlisting[
%    language=protobuf2,
%    style=protobuf,
%    caption={Scalar wrapper types}
%]{code/proto/scalarWrapper.proto}
%
%The purpose of wrapping scalars in custom protobuf messages is to make scalar default values distinguishable from missing data.
%Initializing a protobuf message without explicitly setting the value of field has the consequence that the field gets initialized with its default value.
%The default value for \code{int32} is 0.
%If a gRPC server receives a protobuf message with an \code{int32} field set to 0, it does not know if the field was purposely initialized with 0 or not set at all.
%Wrapping integers in \code{Int32W} messages makes this distinguishable.
%
%Liver tenderness can either be present or not, which makes it a \code{BoolW} field.
%The \AVS uses enumeration fields for inputs that have a limited number of predefined values
%For example, the most frequently used enum type is \code{AbnormalitiesExaminationResult}.
%It denotes the result of a specific examination, which can be either conspicuous or inconspicuous.
%
%\lstinputlisting[
%    language=protobuf2,
%    style=protobuf,
%    caption={\code{AbnormalitiesExaminationResult}}
%]{code/proto/enum.proto}
%
%Knowledge inputs are another important input type.
%They serve similar purposes as enum types but are not hard-coded into the code base.
%The system fetches them from a dedicated microservice that stores them in a MeiliSearch database.
%Block content conditions are a powerful tool to make codes dependent on concrete medical information from anamnesis,
%physical examination and procedures.
%
%\todo{introduce nested objects, list objects}
