\chapter{Future Work}\label{ch:future-work}

\paragraph{Development of an IDE Plugin for Enhanced Rule Writing}
One of the challenges faced during this work was the error-prone nature of writing rules in \RF.
Billing experts often needed help following the language's semantics, leading to frequent mistakes I needed to fix.
Developing an Integrated Development Environment (IDE) plugin would be beneficial to address this.
Proper linting, error detection, and auto-completion could significantly reduce the likelihood of semantic errors
in the rule-writing process.
Such a tool would streamline the workflow and provide a more intuitive interface for billing experts.

\paragraph{Expansion of Condition Types}
As development goes on and the billing framework will cover more and more treatment cases,
further limitations of the current version will likely emerge.
To accommodate evolving needs, there should be a continuous development process aimed at introducing new features,
information sources, and condition types.
Regular updates and version upgrades must ensure the framework remains adaptable to evolving requirements.

\paragraph{Rule Evaluation Performance improvement}
Subsection \ref{subsec:initial-goä-rule-evalutation} mentions that a rule type evaluation always includes retrieving all rules from the database.
This is because condition evaluations such as \anamnesisBlocks, \physicalBlocks, and \requiredIcdCodes involve more complex logic implemented in the application code.
Partial condition evaluation on the database level is a more scalable solution in the long run.
Future versions should replace the currently used \code{findAll} query with a filter that receives a \REI and already evaluates a subset of conditions during database retrieval.

\paragraph{Enhancing Usability with Syntactic Sugar}
Usability is critical for non-technical domain experts when developing a domain-specific language.
One area of improvement could be the introduction of syntactic sugar –
simplified syntax that makes expressions more readable and more accessible to write.
For instance, representing linear functions for multipliers in private billing is currently complex.
Introducing a syntactic sugar expression that expects two points
and returns a linear function crossing both would simplify this task considerably.
This approach would make the rule language more accessible and user-friendly,
encouraging wider adoption and reducing the learning curve for new users.
