\section{Common Issues with Modern Billing Automation Products}\label{sec:common-issues-with-modern-billing-automation-products}
When comparing the approaches of the previously presented billing automation solutions, one might notice a particular pattern.
All those billing automation solutions employ a three-step-procedure:
\begin{enumerate}
    \item The first step is to scrape as much billing-relevant data as possible from multiple external services and resources.
    \item Next, the systems transform and map the collected unstructured information to their structured data schema.
    \item Given the structured data, the systems apply their techniques to generate billing codes.
\end{enumerate}

According to the engineers at \AV, the crucial issue that makes automatic billing generation such a difficult task is the following:
Automatic billing generation strictly requires structured and complete data.
This data often lives as unstructured information scattered across multiple sources, such as existing clinical systems or human-readable documents.
Medical records often lack accuracy and completeness, which correlates with accurate coding \cite{Farhan2005Documentation}.
Additionally, these information sources suffer from differences in quality and redundancies \cite{Scheurwegs2017Selecting}.
The systems presented in section \ref{sec:existing-billing-automation-solutions} rely on these information sources suffering from those issues.
They all do their best to optimize the data collection step and reduce the information loss to a minimum.

The approach in this work is to build a billing automation system that directly integrates with a clinic information system that guarantees access to all billing-relevant data.
The clinic information system must collect the full extent of the billing-relevant information as structured data, making fuzzy data scraping from external sources unnecessary.
