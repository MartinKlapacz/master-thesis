\section{Common Issues with Modern Billing Automation Products}\label{sec:common-issues-with-modern-billing-automation-products}
When comparing the approaches of the previously presented billing automation solutions one might notice a common pattern.
All those billing automation solutions employ a three step-procedure:
\begin{enumerate}
    \item The first step is to scrape multiple information sources to collect as much billing-relevant data as possible.
    \item Next, the systems transform and map the collected unstructured information to their own structured data schema.
    \item Given this now structured dataset, the systems then apply their own technique to generate billing codes from the data in the schema.
    Those techniques are often AI- or rule-based or a hybrid of both.
\end{enumerate}

This pattern tries to solve the issue that automatic billing code generation from treatment and patient data requires structured data.
This data typically lives as unstructured information scattered across multiple sources such as existing clinical systems or human-readable documents.
The systems do their best to optimize this data collection step and to reduce the information lost to a minimum.
This pattern, however, has a set of issues.


Before this text I introduced multiple automatic medical billing products, including 3m,tiplu momo, id clinical context coding.
This is what I have right now, i want to express that this pattern has some issues:
- documents and legacy systems are not guaranteed to deliver all the information that the system needs, documents of miss small details
- external systems track data not in a billing-oriented way. They are designed to track data in a patient documentation way but are designed with a different intention.

do you have other ideas? Maybe papers that may help


\begin{enumerate}
    \item
\end{enumerate}
