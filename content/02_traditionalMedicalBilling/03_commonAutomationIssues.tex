\section{Common Issues with Modern Billing Automation Products}\label{sec:common-issues-with-modern-billing-automation-products}
When comparing the approaches of the previously presented billing automation solutions, one might notice a certain pattern.
All those billing automation solutions employ a three-step-procedure:
\begin{enumerate}
    \item The first step is to scrape as much billing-relevant data as possible from multiple external services and resources.
    \item Next, the systems transform and map the collected unstructured information to their own structured data schema.
    \item Given the structured data, the systems then apply their own techniques to generate billing codes.
    Those techniques are often AI or rule-based approaches or a hybrid of both.
\end{enumerate}

One of the crucial issues that make automatic billing generation such a difficult task is the following:
Automatic billing generation strictly requires structured and complete data. \addcite
This data often lives as unstructured information scattered across multiple sources such as existing clinical systems or human-readable documents.
Medical records often lack accuracy and completeness of documentation, which strongly correlates with accurate coding \cite{Farhan2005Documentation}.
Additionally, these information sources suffer from differences in quality and redundancies \cite{Scheurwegs2017Selecting}.
The systems presented in section \ref{sec:existing-billing-automation-solutions} rely on these information sources suffering from those issues.
All of them do their best to optimize the data collection step and to reduce the information loss to a minimum.

A solution for that issue is a billing automation system directly integrated into a clinic information system that guarantees access to the entire billing-relevant data.
It is crucial that the clinic information system collects the full extent of the billing-relevant information as structured data, making fuzzy data scraping from external sources unnecessary.

