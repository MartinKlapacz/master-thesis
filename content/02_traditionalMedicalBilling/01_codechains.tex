\section{Manual Billing with Code Chains}\label{sec:billing-positions-and-code-chains}
This chapter is based on insights I received from interviews with Dr. Lorenz Thuman, Medical Manager at \AV.
Code chains (Leistungsketten) are a simple solution for manual billing in a medical software application.
A billing position refers to the data representation of a service provided during treatment.
It comprises the billing code and meta information, including a catalog type, quantity, service-specific details, and more.
Code chains are custom presets of billing position templates a user can apply to the current billing.
This action adds all billing position templates to the current billing.
After applying a code chain to the current billing, the user can edit the loaded billing positions and remove inappropriate ones.
The software usually organizes code chains in a folder structure,
making it easier for medical staff to navigate and select the billing codes for different medical procedures or treatments.
A medical staff member is typically responsible for initially setting up the code chains and code chain folders and keeping them updated.
Finally, the purpose of code chains is to quickly provide the user with a set of possible billing code candidates for the current billing.
The user, however, still needs to remove inappropriate codes and edit the final billing.



%is to facilitate manual coding by

%At the top level of this structure, there are main code chain folders such as \"Consultation\", \"Diagnostics\", and \"treatment\" (Treatment).
%Within the 'Behandlung' folder, for instance, there might be sub-folders for specific treatments like 'Kryotherapie' (Cryotherapy) and 'Impfung' (Vaccination).
%Delving deeper, within the 'Impfung' folder, there are code chains for specific types of vaccinations, such as COVID-19 vaccinations and MMR (Measles, Mumps, and Rubella) vaccinations.
