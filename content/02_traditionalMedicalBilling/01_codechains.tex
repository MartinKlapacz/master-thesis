\section{Manual Billing with Code Chains}\label{sec:billing-positions-and-code-chains}
Code chains (Leistungsketten) are one way of implementing manual billing in medical software solutions.
They are a feature requirement posed by billing experts at \AV and.
According to them, other software solutions employ them for medical billing.
The following description is based on the information from a requirement posed by the billing experts at \AV.

In this context, a billing position refers to the data representation of a service provided during treatment.
The data representation of a provided service during the treatment is what we call a billing position.
It consists of the billing code and meta-information which includes a catalog type, a quantity, service-specific details and more.

Code chains are custom presets of billing position templates a user can apply to the current billing.
This action adds all billing position templates to the current billing.
After applying a code chain to the current billing, the user should edit billing positions with appropriate codes and remove inappropriate billing positions.

Code chains are usually organized in a folder structure that corresponds to medical areas,
making it easier for medical staff to navigate and select the billing codes for different medical procedures or treatments.
A medical staff member is typically responsible for initially setting up the code chains and code chain folders and keeping them updated.
The purpose of code chains and code chain folders is to facilitate manual coding and avoid coding errors.
%At the top level of this structure, there are main code chain folders such as \"Consultation\", \"Diagnostics\", and \"treatment\" (Treatment).
%Within the 'Behandlung' folder, for instance, there might be sub-folders for specific treatments like 'Kryotherapie' (Cryotherapy) and 'Impfung' (Vaccination).
%Delving deeper, within the 'Impfung' folder, there are code chains for specific types of vaccinations, such as COVID-19 vaccinations and MMR (Measles, Mumps, and Rubella) vaccinations.
