
\subsection{3M}
%3M SmarteKI is primarily designed for use in both inpatient and outpatient settings, focusing on streamlining and automating various aspects of clinical billing. It aims to simplify the complex processes of DRG (Diagnosis-Related Group) and PEPP (Pauschalierendes Entgeltsystem Psychiatrie und Psychosomatik) coding, thereby assisting healthcare professionals in managing billing tasks more efficiently.
%The system integrates seamlessly with a range of common healthcare information systems used in German clinics, facilitating the collection and processing of patient and treatment data crucial for billing. This includes data from diverse sources such as patient visits, diagnoses, laboratory results, and other relevant information.
%3M SmarteKI employs advanced AI algorithms to conduct semantic analysis of various medical documents, including physician letters and consultation reports. This capability allows it to convert unstructured medical data into a structured format, essential for accurate billing and coding.
%One of the key features of SmarteKI is its ability to automatically generate billing statements and perform plausibility checks on existing codes, ensuring accuracy and compliance with medical billing standards.
%Furthermore, 3M has developed a robust machine learning network, trained on extensive clinical data, to enhance the system's capability in automated coding and billing case recognition. This network adheres to strict data privacy standards, ensuring that sensitive patient information remains secure.
%In addition to its AI-driven functionalities, 3M SmarteKI also includes a rule-based coding system, constantly updated and refined through ongoing feedback from healthcare providers and insurers. This ensures that the system stays current with the latest billing regulations and practices.


state of the art coding by analysing 100 percent of patient records

state of the art tool for scrubbing medical claims
