\subsection{Momo - Tiplu}\label{subsec:momo---tiplu}
Tiplu's billing product MOMO is a solution for operative medical controlling in hospitals.
It represents a significant advancement in the field of healthcare billing and management,
leveraging the power of artificial intelligence (AI) to streamline and improve the billing process.
I had the opportunity for an expert interview with Jan Willer, Senior Account Manager at Tiplu GmbH.
The following section is based on the knowledge and insights I gained from this interview.
MOMO's primary application is within the inpatient setting, focussing on automating DRG and PEPP coding and providing support for the medical user.
But it is planned to additionally cover outpatient billing starting from the first quarter of 2024.
It provides an overview of the patient cases which includes visits, diagnoses, laboratory results and other billing relevant information.
Momo makes use of two major information sources.
It firstly integrates with all common German clinic information systems, including Orbis, i.s.h.med, NEXUS and Medico.
Momo collects patient and treatment data relevant for billing from special interfaces in existing clinic systems.
Secondly, MOMO uses an AI model to perform semantic text analysis on various medical documents such as physician letters, consultation reports, and visit entries.
Using both approaches, MOMO transforms unstructured information from various sources into a structured data schema.
In practice, this aspect of development is the most labor-intensive part of creating MOMO.
Finally, MOMO uses the collected, structured data to provide automatic billing generation and plausibility checks of existing codes.
Tiplu has developed a large machine learning network in the German-speaking region which encompasses more than 140 clinics and which is trained with data from 12 million patients.
The network uses the concept of federated learning which ensures that patient data never leave the clinic.
Tiplu uses its AI model for automated coding and billing case recognition.
Additionally, Tiplu has developed a rule language for rule-based code derivation.
Maintenance and development of new rules is an ongoing process where employees receive feedback from clinics as well as ensurer to refine and update the rule base.

