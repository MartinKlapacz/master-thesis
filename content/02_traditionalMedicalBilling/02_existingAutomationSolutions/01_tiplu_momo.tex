\subsection{Momo - Tiplu}\label{subsec:momo---tiplu}
Tiplu’s billing product, MOMO, is a solution for operative medical controlling in hospitals.
I had the opportunity to have an expert interview with Jan Willer, Senior Account Manager at Tiplu GmbH. The following section is based on the knowledge and insights I gained from this interview. MOMO’s primary application is within the inpatient
setting, focussing on automating DRG and PEPP coding and providing support for
the medical user.
Within its user interface, it provides an overview of the patient case, which includes diagnoses, laboratory results, and other billing-relevant information.
Momo integrates with all major German clinic information systems.
It collects patient and treatment data relevant to billing from special interfaces in existing clinic systems.
Secondly, MOMO uses an AI model to perform semantic text analysis on various
medical documents such as physician letters, consultation reports, and visit entries.
Using both approaches, MOMO transforms unstructured information from various
sources into a structured data schema.
Finally, MOMO uses the collected, structured data to provide automatic billing generation and plausibility checks of existing codes.
Tiplu has developed a large machine learning network in the German-speaking region,
encompassing more than 140 clinics.
The network uses the concept of federated learning, which ensures that patient data never leaves the clinic.
Tiplu uses its AI model for auto-mated coding and billing case recognition.
Additionally, Tiplu has developed a rule language for rule-based code derivation from its structured data schema. Maintenance
and development of new rules is an ongoing process where employees receive feedback
from clinics as well as ensure updates to the rule base.
