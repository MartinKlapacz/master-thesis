\subsubsection{General GOÄ Characteristics}
The Gebührenordnung für Ärzte (GOÄ) is the coding catalog used for billing medical services covered by private health insurance.
The GOÄ is known for being less regulated and offering more leeway in practice.
However, it has been a subject of longstanding criticism.
The GOÄ has the reputation of being, unfortunately, highly outdated.
The current version appeared in 1982, and the last update took place in 1996 \cite{heller2015goa}.
Not only the lack of modern procedures in the catalog but also ambiguities and interpretation difficulties present issues for healthcare providers.
Since 1996, consumer prices have increased by more than 50\%, and inflation has effectively reduced professional fees by more than 28\% \cite{schmitzgoa}.
The initial point price for the GOÄ was 5.82873 ct.
It was increased to 5.62ct on July 1, 1988, and again increased to 5.83 ct from January 1, 1996\cite{hermanns2013bemessung}.
Of course, these point value updates do not compensate for the price increase associated with medical treatments in the last decades.
The only thing practitioners can do to manage this discrepancy independently is to create correct billings that cover the costs that occurred as well as possible.
Fortunately, the GOÄ has some characteristics that allow flexible billing price optimization.
Given this growing financial discrepancy, applying those optimization techniques correctly and appropriately is vital in modern medical billing.
In the following sections, we look closely at GOÄ characteristics relevant to optimization in private billing.


\section{Billing Multipliers}\label{sec:billing-multipliers}
Unlike the EBM, the GOÄ allows for the application of billing multipliers.
Billing multipliers are an essential tool to tailor private billing and to consider unexpected difficulties and increased time expenditure of treatments \cite{walter2008abrechnung}.
They are assignable to individual billing codes and increase the price for the respective position by multiplication of the GOÄ base fee and the multiplier value.
\[
    \mathrm{service\ price} = \mathrm{base\ code\ points} \times \mathrm{billing\ multiplier} \times \mathrm{point\ value}
\]
Doctors can adjust multiplier values depending on the complexity, time expenditure, or special circumstances of the treatment to certain limits.
Each GOÄ code specifies a base rate, a threshold and a max rate\cite{bruck1998kommentar}.
\begin{itemize}
    \item \textbf{Base Rate of 1.0}: This is the base rate for all GOÄ services and corresponds to the fee listed in each GOÄ record.
    \item \textbf{Threshold}: Applying multipliers up to this value does not require any justifications.
    \item \textbf{Maximum Rate}: This range is used for specialized or particularly complex services.
    The application of these multipliers typically requires a detailed justification.
\end{itemize}
The actual values of the threshold and the maximum rate depend on the code.
%Table\ref{tab:fee-frames} displays all fee frames with their corresponding values for the threshold and the maximum rate\cite{hermanns2013bemessung}.
%Services that include a personal patient-practitioner contact are part of the personal-medical service.
%The GOÄ allows higher multiplier values for services in that area as complications and directly affects the treating practitioner.
%Services in the technical frame and especially laboratory services affect have smaller direct effect on the practitioner and, thus, are more limited in terms of applicable multipliers.
%\begin{center}
%    \begin{tabular}{ |c|c|c| }\label{tab:fee-frames}
%        \hline
%        fee frame type & threshold & Max Rate \\
%        \hline
%        personal-medical service frame & 2.3 & 3.5 \\
%        technical service frame & 1.8 & 2.5 \\
%        laboratory service frame & 1.15 & 1.3 \\
%        \hline
%    \end{tabular}
%\end{center}

\section{Multiplier Justifications}\label{sec:multiplier-justifications}
Applying a specific multiplier that exceeds the multiplier threshold to a billing position requires a justification.
However, the GOÄ does not standardize multiplier justifications \cite{bruck1998kommentar}.
There is no official collection of applicable and accepted justifications for multipliers.
They are free texts appearing next to the respective multiplier in the final invoice.
The billing framework should provide a way of expressing multiplier justifications as rules.
The system should automatically check for applicable justifications as part of private billing.
This way, the system can ensure the user does not miss opportunities for appropriate and valid optimizations.
Multipliers must always have correct justifications backed up by the structured data in the treatment.
Part of the research of this work is to collect commonly used billing justifications from experts.
It is also essential to understand the relevant data that multiplier justifications rely on.
Subsection \ref{subsec:multiplier-justification-rules} contains a selected collection of billing justification rules that I have found as part of my research.

\section{Surcharges}\label{sec:surcharges}
Surcharges are GOÄ billing codes but serve similar purposes as billing multipliers.
They do not directly refer to the services provided by the practitioner but rather describe special conditions present during the treatment \cite{walter2008abrechnung}.
Unlike billing multipliers, they are ordinary billing codes defined by the GOÄ and do not attach to other codes in the invoice.
For example, the surcharge \goa{C} if applicable if the treatment took place between 10 pm and 6 am.
