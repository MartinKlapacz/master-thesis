\subsubsection{General GOÄ Characteristics}
The Gebührenordnung für Ärzte (GOÄ) is the coding catalog used for billing medical services covered by private health insurance.
The GOÄ is known for being less regulated and offering more leeway in practice.
However, it has been a subject of longstanding criticism.
The GOÄ has the reputation of being, unfortunately, highly outdated.
The current version was defined in 1982, last update took place in 1996 \cite{heller2015goa}.
Not only the lack of modern procedures in the catalog and as well as ambiguities and interpretation difficulties presents issues for healthcare providers.
Since 1996 the consumer prices have increased by more than 50\%, inflation has effectively reduced professional fees by more than 28\% \cite{schmitzgoa}.
The point price for the GOÄ was initially set to 5.82873 ct.
It was increased to 5.62ct in July 1, 1988 and again increased to 5.83 ct from January 1, 1996\cite{hermanns2013bemessung}.
Of course, these point value updates do not compensate for the actual price increase associated with medical treatments in the last decades.
The only thing practitioners can do to manage this discrepancy on their own is to create correct billings that cover the occurred costs as well as possible.
Fortunately, the GOÄ has some characteristics that allow for flexible billing price optimization.
Given this growing financial discrepancy, applying those optimization techniques in a correct and appropriate way plays an important role in modern medical billing.
In the following sections, we take a closer look at GOÄ characteristics that are relevant for optimization in private billing.

\section{Billing Multipliers}\label{sec:billing-multipliers}
Unlike the EBM, the GOÄ allows for the application of billing multipliers.
Billing multipliers are an important tool to tailor private billing and to take into account unexpected difficulties and increased time expenditure of treatments \cite{walter2008abrechnung}.
They are assignable to individual billing codes in a treatment billing and increase the price for the respective position by multiplication of the GOÄ base fee and the multiplier value.
\[
    \mathrm{service\ price} = \mathrm{base\ code\ points} \times \mathrm{billing\ multiplier} \times \mathrm{point\ value}
\]
Doctors can adjust multiplier values depending on the complexity, time expenditure, or special circumstances of the treatment to certain limits.
Each GOÄ code specifies a base rate, a threshold and a max rate\cite{bruck1998kommentar}.
\begin{itemize}
    \item \textbf{Base Rate of 1.0}: This is the base rate for all GOÄ services and corresponds to the fee listed in each GOÄ record.
    \item \textbf{Threshold}: Applying multipliers up to this value does not require any justifications.
    \item \textbf{Maximum Rate}: This range is used for specialized or particularly complex services.
    The application of these multipliers typically requires a detailed justification.
\end{itemize}

The actual values of the threshold and the maximum rate depend on the fee frame the code belongs to.
Table\ref{tab:fee-frames} displays all fee frames with their corresponding values for the threshold and the maximum rate\cite{hermanns2013bemessung}.
Services that include a personal patient-practitioner contact are part of the personal-medical service.
The GOÄ allows higher multiplier values for services in that area as complications and directly affects the treating practitioner.
Services in the technical frame and especially laboratory services affect have smaller direct effect on the practitioner and, thus, are more limited in terms of applicable multipliers.
\begin{center}
    \begin{tabular}{ |c|c|c| }\label{tab:fee-frames}
        \hline
        fee frame type & threshold & Max Rate \\
        \hline
        personal-medical service frame & 2.3 & 3.5 \\
        technical service frame & 1.8 & 2.5 \\
        laboratory service frame & 1.15 & 1.3 \\
        \hline
    \end{tabular}
\end{center}

\section{Multiplier Justifications}\label{sec:multiplier-justifications}
Applying a specific multiplier that exceeds the multiplier threshold to a billing position requires a justification.
However, the GOÄ does not standardize multiplier justifications \cite[]{bruck1998kommentar}.
There is no official collection of applicable and excepted justifications for multipliers.
They are essentially free texts that appear next to the respective multiplier in the final invoice.

The billing framework should provide a way of expressing multiplier justifications as rules.
As part of private billing, the system should automatically check for applicable justifications.
This way, the system can make sure that the user does not miss out opportunities for appropriate and valid optimizations.
It is crucial that multipliers always have justifications backed up by the structured data in the treatment.

Part of the research of this work is to collect commonly used billing justifications from experts.
It is also important to understand what the relevant data is that multiplier justifications rely on.
Subsection \ref{subsec:multiplier-justification-rules} contains a selected collection of billing justifications rules that I have found during as part of my research.

\section{Surcharges}\label{sec:surcharges}
Surcharges are GOÄ billing codes but serve similar purposes as billing multipliers.
They don't directly refer to the services provided by the practitioner but rather describe special conditions that were present during the treatment \cite{walter2008abrechnung}.
Unlike billing multipliers, they are ordinary billing codes defined by the GOÄ and do not attach to other codes in the invoice.

For example, the surcharge \goa{C} if applicable if the treatment took place between 10pm and 6am.
One must take into account that there are strict exclusion rules that limit the application of surcharges\cite{kommentar2012zuschlage}.
Surcharge \goa{A} conflicts with \goa{B}, \goa{C} or \goa{D}.
Surcharges \goa{A}, \goa{B}, \goa{C}, \goa{D} and \goa{K1} are not applicable in combination with \goa{E}, \goa{F}, \goa{G}, \goa{K2} \cite{kommentar2012zuschlage}.
Surcharges are distinguishable from ordinary GOÄ codes by their name.
Ordinary GOÄ codes have only digit codes while surcharge codes consist of letters.

