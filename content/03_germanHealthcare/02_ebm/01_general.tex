\subsection{General EBM Characteristics}\label{subsec:general-ebm-characteristics}
The Einheitlicher Bewertungsmaßstab (EBM) is a standardized coding catalog used in Germany for billing outpatient services for patients with statutory insurance.
The Kassenärztliche Bundesvereinigung (KBV), which is the Federal Association of Statutory Health Insurance Physicians \cite[]{hermanns2015ebm}, is responsible for its maintenance.
The KBV maintains the EBM as a downloadable and well-documented dataset in XML format, which stores plentiful meta-information for each rule.
Large parts of the meta-information describe exclusion rules and conditions that decide whether a rule is applicable.
Statutory billing with the EBM is comparably conservative.
It is less flexible and does not offer any means for price optimization as the GOÄ does.


