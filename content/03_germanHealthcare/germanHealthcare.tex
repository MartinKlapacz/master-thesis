\chapter{Billing in the German Healthcare System}\label{ch:billing-in-the-german-healthcare-system}

In the German healthcare system, billing is a critical component that ensures the smooth functioning of medical services.
Patients in Germany can have statutory or private insurance.
There are different regulations and catalogs for both insurance types.
The most important coding catalogs are the Einheitlicher Bewertungsmaßstab (EBM) for statutory insurance and the Gebührenordnung für Ärzte (GOÄ) for private insurance and self-payers.
Both are essentially a collection code records representing services covered by insurances.
A code record has a schema containing the service code, the associated number of points, a description and additional meta-data.
The points quantify the price of the medical services with one point translating to a monetary value determined by the catalog.

Subsections \ref{sec:einheitlicher-bewertungsmastab-(ebm)} and \ref{sec:gebuhrenordnung-fur-arzte-(goa)} describe both catalogs in detail.
We firstly cover catalog-specific information and general catalog characteristics, focussing on those relevant for the billing framework.
Next, we depict all condition types living in the catalogs as structured meta-data that we need to consider for automatic billing generation.
The framework can automatically parse existing meta-data condition information from the catalog and evaluate it.
This makes it unnecessary to include them in any rules, which saves a lot of effort.

\section{Gebührenordnung für Ärzte (GOÄ)}\label{sec:gebuhrenordnung-fur-arzte-(goa)}

\subsubsection{General GOÄ Characteristics}
The Gebührenordnung für Ärzte (GOÄ) is the coding catalog used for billing medical services covered by private health insurance.
The GOÄ is known for being less regulated and offering more leeway in practice.
However, it has been a subject of longstanding criticism.
The GOÄ has the reputation of being, unfortunately, highly outdated.
The current version was defined in 1982, last update took place in 1996 \cite{heller2015goa}.
Not only the lack of modern procedures in the catalog and as well as ambiguities and interpretation difficulties presents issues for healthcare providers.
Since 1996 the consumer prices have increased by more than 50\%, inflation has effectively reduced professional fees by more than 28\% \cite{schmitzgoa}.
The point price for the GOÄ was initially set to 5.82873 ct.
It was increased to 5.62ct in July 1, 1988 and again increased to 5.83 ct from January 1, 1996\cite{hermanns2013bemessung}.
Of course, these point value updates do not compensate for the actual price increase associated with medical treatments in the last decades.
The only thing practitioners can do to manage this discrepancy on their own is to create correct billings that cover the occurred costs as well as possible.
Fortunately, the GOÄ has some characteristics that allow for flexible billing price optimization.
Given this growing financial discrepancy, applying those optimization techniques in a correct and appropriate way plays an important role in modern medical billing.
In the following sections, we take a closer look at GOÄ characteristics that are relevant for optimization in private billing.

\section{Billing Multipliers}\label{sec:billing-multipliers}
Unlike the EBM, the GOÄ allows for the application of billing multipliers.
Billing multipliers are an important tool to tailor private billing and to take into account unexpected difficulties and increased time expenditure of treatments \cite{walter2008abrechnung}.
They are assignable to individual billing codes in a treatment billing and increase the price for the respective position by multiplication of the GOÄ base fee and the multiplier value.
\[
    \mathrm{service\ price} = \mathrm{base\ code\ points} \times \mathrm{billing\ multiplier} \times \mathrm{point\ value}
\]
Doctors can adjust multiplier values depending on the complexity, time expenditure, or special circumstances of the treatment to certain limits.
Each GOÄ code specifies a base rate, a threshold and a max rate\cite{bruck1998kommentar}.
\begin{itemize}
    \item \textbf{Base Rate of 1.0}: This is the base rate for all GOÄ services and corresponds to the fee listed in each GOÄ record.
    \item \textbf{Threshold}: Applying multipliers up to this value does not require any justifications.
    \item \textbf{Maximum Rate}: This range is used for specialized or particularly complex services.
    The application of these multipliers typically requires a detailed justification.
\end{itemize}

The actual values of the threshold and the maximum rate depend on the fee frame the code belongs to.
Table\ref{tab:fee-frames} displays all fee frames with their corresponding values for the threshold and the maximum rate\cite{hermanns2013bemessung}.
Services that include a personal patient-practitioner contact are part of the personal-medical service.
The GOÄ allows higher multiplier values for services in that area as complications and directly affects the treating practitioner.
Services in the technical frame and especially laboratory services affect have smaller direct effect on the practitioner and, thus, are more limited in terms of applicable multipliers.
\begin{center}
    \begin{tabular}{ |c|c|c| }\label{tab:fee-frames}
        \hline
        fee frame type & threshold & Max Rate \\
        \hline
        personal-medical service frame & 2.3 & 3.5 \\
        technical service frame & 1.8 & 2.5 \\
        laboratory service frame & 1.15 & 1.3 \\
        \hline
    \end{tabular}
\end{center}

\section{Multiplier Justifications}\label{sec:multiplier-justifications}
Applying a specific multiplier that exceeds the multiplier threshold to a billing position requires a justification.
However, the GOÄ does not standardize multiplier justifications \cite[]{bruck1998kommentar}.
There is no official collection of applicable and excepted justifications for multipliers.
They are essentially free texts that appear next to the respective multiplier in the final invoice.

The billing framework should provide a way of expressing multiplier justifications as rules.
As part of private billing, the system should automatically check for applicable justifications.
This way, the system can make sure that the user does not miss out opportunities for appropriate and valid optimizations.
It is crucial that multipliers always have justifications backed up by the structured data in the treatment.

Part of the research of this work is to collect commonly used billing justifications from experts.
It is also important to understand what the relevant data is that multiplier justifications rely on.
Subsection \ref{subsec:multiplier-justification-rules} contains a selected collection of billing justifications rules that I have found during as part of my research.

\section{Surcharges}\label{sec:surcharges}
Surcharges are GOÄ billing codes but serve similar purposes as billing multipliers.
They don't directly refer to the services provided by the practitioner but rather describe special conditions that were present during the treatment \cite{walter2008abrechnung}.
Unlike billing multipliers, they are ordinary billing codes defined by the GOÄ and do not attach to other codes in the invoice.

For example, the surcharge \goa{C} if applicable if the treatment took place between 10pm and 6am.
One must take into account that there are strict exclusion rules that limit the application of surcharges\cite{kommentar2012zuschlage}.
Surcharge \goa{A} conflicts with \goa{B}, \goa{C} or \goa{D}.
Surcharges \goa{A}, \goa{B}, \goa{C}, \goa{D} and \goa{K1} are not applicable in combination with \goa{E}, \goa{F}, \goa{G}, \goa{K2} \cite{kommentar2012zuschlage}.
Surcharges are distinguishable from ordinary GOÄ codes by their name.
Ordinary GOÄ codes have only digit codes while surcharge codes consist of letters.


\subsection{GOÄ Catalog Condition Types}\label{subsec:goa-condition-types}
The great flexibility associated with German private billing is a consequence of the design of the GOÄ.
It only defines a single condition type in a structured way.

Code exclusions are the only structured condition types in the GOÄ.
The scope for this exclusion is the current billing.
There are more conditions on GOÄ codes that, however, only exist as unstructured free text.
There is, unfortunately, no direct way to make use of these unstructured conditions as we do in the EBM.
Instead, code derivation for private treatments will be mostly based on rules and less on structured catalog conditions.


\section{Einheitlicher Bewertungsmaßstab (EBM)}\label{sec:einheitlicher-bewertungsmastab-(ebm)}

\subsection{General EBM Characteristics}\label{subsec:general-ebm-characteristics}
The Einheitlicher Bewertungsmaßstab (EBM) is a standardized coding catalog used in Germany for billing outpatient services for patients with statutory insurance.
The Kassenärztliche Bundesvereinigung (KBV), which is the Federal Association of Statutory Health Insurance Physicians \cite[]{hermanns2015ebm}, is responsible for its maintenance.
The KBV maintains the EBM as a downloadable and well-documented dataset in XML format, which stores plentiful meta-information for each rule.
Large parts of the meta-information describe exclusion rules and conditions that decide whether a rule is applicable.
Statutory billing with the EBM is comparably conservative.
It is less flexible and does not offer any means for price optimization as the GOÄ does.



%\subsection{EBM Structure}

\subsection{EBM Condition Types}\label{sec:ebm-conditions}


Each EBM code is represented by a EBM code record in the EBM dataset.
The KBV maintains an extensive xml schema documentation for the EBM data model.


Part of this work is to understand and identify all conditions that determine under which circumstances a code can be billed or not.
For automated derivation of applicable EBM codes these conditions need to be automatically checked.
In this section we go through all identified relevant EBM conditions, that the billing framework need to check automatically.


\subsection{EBM reference contexts}\label{subsec:ebm-reference-contexts}
Before describing the EBM condition types, we need to introduce the concept of reference contexts.
Condition types such as exclusion rules often refer to a specific reference context.
The following is a description of those reference contexts defined by the EBM.
\begin{itemize}
    \item The smallest reference context is the Visit (in der selben Sitzung), which simply refers to all services occurred during the patients visit at the practitioner.
    It occurs in 2688 EBM codes.
    \item One level above is the case (Behandlungsfall), which refers to the current quarter of the year.
     The case encompasses all billing positions that happened between start and end of the quarter.
    It is even more important than the visit reference context.
    It is used 4091 times.
    \item The Illness Case (Krankheitsfall) refers to the current and the three following quarters.
    It is less important but still occurs in 343 conditions.
    \item A Reproduction Case (Reproduktionsfall) encompasses all services that are linked to a pregnancy.
    It refers to the entire pregnancy of a patient from the determination of the pregnancy to delivery, including postnatal care.
    It has just 13 usages.
    \item Similarly to the reproduction case, the Cycle Case encompasses all service provisions linked to the patients menstrual cycle
    All issues, consultations, or examinations associated with menstruation fall into that context.
\end{itemize}
Reference contexts for the current year, current month, current weeks and treatment day that are self-explanatory exist as well.
There are very specific reference contexts used by very few specialized EBM codes that need to be taken into consideration as well.
For the sake of completeness they are listed in the following:
\begin{itemize}
    \item Within a period of 21 days after the provision of a service from section 31.2:
    Section 31.2 contains the \glqq outpatient operations\glqq.
    \item Within a period of 3 days after the provision of a service from section 36.2:
    Section 36.2 contains the \glqq Attendant medical operations\glqq.
    \item Within a period of 3 days after the provision of a service from section 31.2:
\end{itemize}
In the current version of the EBM the latter two reference contexts are not used at all.
The system should, nonetheless, take them consideration as well as they can be used in future versions.




\subsection{Authorized Practitioner Types List}

The EBM maintains an extensive specification of what type of practitioners are allowed to bill which EBM codes.
For that it defines the two axes medical specialization and service area.

All values for service area of a practitioner are
\begin{itemize}
    \item Specialist (Facharzt)
    \item General Practitioner (Hausarzt)
    \item Specialist-General Practitioner (Fach- Hausarzt)
    \item Authorized Physician (Ermächtigter Arzt)
\end{itemize}

The EBM distinguishes 206 medical specializations.
Possible service medical specializations are the following:
\begin{itemize}
    \item TG Hematology (TG Hämatologie)
    \item Specialist/Consultant in Neurosurgery (FÄ/FA Neurochirurgie)
    \item Specialist/Consultant in General Surgery (FÄ/FA Allgemeine Chirurgie)
    \item TG Vascular Surgery (TG Gefäßchirurgie)
    \item Subspecialty Neuropediatrics (SP Neuropädiatrie)
\end{itemize}

Formally, each EBM code specifies a set of points from the space spanned by those two axes.
If and only if the practitioner's service area and medical specialization pair is an element of that set, they will be authorized to bill that EBM code.
\todo{Add example}


\subsection{Code Justification List}
Some EBM codes can have a code justification list.
A justification is one of the following:
\begin{itemize}
    \item An OPS code identifying a procedure conducted in this treatment.
    \item An ICD10 code of a diagnosis given to the patient during this treatment
    \item Another EBM code that is part of this billing
\end{itemize}
A code justification list condition holds if and only if it is empty or at least one code of the three justification categories is present.
Localization specific OPS codes are combined with a localization identifiers (left, right or both sides).
For example the EBM code \meco{31658} has the ops number \meco{5-184.1} in combination with the \meco{B} localization identifier in its code justification list.
For example, \meco{OPS 5-184.1} is listed in combination with the \meco{B} localization identifier and is thus a possible justification code for \meco{EBM 31658}.
This means, applying the billing code \mete{Postoperative Behandlung Hals-Nasen-Ohren VI/2a} (\meco{31658}) i.e. \mete{Postoperative treatment of ear, nose, and throat (ENT) VI/2a} to a \mete{plastic correction of protruding ears through excision of soft tissues} is only possible if both ears were operated on.

The idea behind EBM codes being justifications for other EBM codes is the following.
There are EBM codes with a very general description.

\subsection{Amount Condition}

Another class of conditions are amount conditions.
EBM codes can be limited in how often the code can be billed in a specific reference context.


\subsection{Exclusion List}\label{subsec:exclusion-list}
Almost all EBM codes maintain an exclusion list.

An exclusion condition can be seen as a quadrupel consisting of the current EBM code, the excluded EBM code, the reference context and a positive integer.
The practitioner must resolve exclusion conflicts by hand.

\subsection{Administrative Gender}\label{subsec:administrative-gender}
EBM codes that contain that condition type can only be explicitly applied for a specific gender.
The number of gender-specific EBM is 150.
Most gender-specific EBM codes come from the fields of gynecology and urology.

\subsection{Form Type List}\label{subsec:form-type-list}

\subsection{Obligate Service Content}\label{subsec:obligate-service-content}
Obligate service contents is a description text of \todo{}

\subsection{Additional Input Information}\label{subsec:additional-information}
Form Type lists are code specific additional input fields.
The KBV defines 35 different additional input information types.
An EBM code can define a subset of these as required additional input information.
This requires the practitioner to add more information to the billing.
Effectively, the billing framework obligates the practitioner to add the required additional information for billing codes that otherwise are perfect billing candidates.
This can be assured by displaying the inputs with the correct data types in the front-end.
\todo{Provide a example}
In the current version of the EBM only few additional input types are actually used.
Those are:
\begin{itemize}
    \item Finish time
    \item OPS number
    \item ICD code
    \item BSNR
\end{itemize}


