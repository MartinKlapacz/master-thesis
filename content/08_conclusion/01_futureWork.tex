\section{Future Work}\label{sec:future-work}

\paragraph{Development of an IDE Plugin for Enhanced Rule Writing}
One of the challenges faced during this work was the error-prone nature of writing rules in \RF.
Billing experts often struggled with following the language's semantics, leading to frequent mistakes that I need to fix.
To address this, the development of an Integrated Development Environment (IDE) plugin would be beneficial.
Proper linting, error detection, and  auto-completion, could significantly reduce the likelihood of semantic errors
in the rule-writing process.
Such a tool would not only streamline the workflow but also provide a more intuitive interface for billing experts.

\paragraph{Expansion of Condition Types}
As development goes on and the billing framework will cover more and more treatment cases,
it is very likely that further limitations of the current version will emerge.
To accommodate evolving needs, there should be a continuous development process aimed at introducing new features,
information sources, and condition types.
Regular updates and version upgrades must ensure that the framework remains adaptable to evolving requirements.

\paragraph{Rule Evaluation Performance improvement}
Subsection \label{subsec:initial-goä-rule-evalutation} mentions that a rule type evaluation always includes the retrieval of all rules from the database.
This is because condition evaluations such as \anamnesisBlocks, \physicalBlocks and \requiredIcdCodes involve more complex logic implemented in the application code.
Partial condition evaluation on the database level is a more scalable solution in the long run.
Future versions should replace the currently used \code{findAll} query by a filter that receives a \REI and already evaluates a set of conditions only returning the matching ones.

\paragraph{Enhancing Usability with Syntactic Sugar}
Usability is a critical aspect when developing a Domain-Specific Language (DSL) for non-technical domain experts.
One area of improvement could be the introduction of syntactic sugar –
simplified syntax that makes expressions more readable and easier to write.
For instance, representing linear functions for multipliers in private billing is currently complex.
Introducing a syntactic sugar expression that can take two points
and return a linear function crossing both would simplify this task considerably.
This approach would make the rule language more accessible and user-friendly,
encouraging wider adoption and reducing the learning curve for new users.
