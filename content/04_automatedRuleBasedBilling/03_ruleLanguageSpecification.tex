\section{Rule Language Specification}\label{sec:rule-language-specification}

The root level of a rule object looks like this:

The rule language follows the higher-level definition of a rule described in section\ref{ch:aspects-of-rule-based-systems}
The root level of a rule has the following keys:

\subsection{Semantics}
The challenge is to make sure the language is rich enough
to express all conditions that exist in the respective rule type.
On the other side, the users of the rule language don't have a technical background.
So whilst being expressive enough, its semantics need to be as simple as possible.
The framework must be able to catch semantic errors and provide an helpful error message.

Each rule type explicitly enables just a subset of all features that are relevant in the respective rule type.
Which conditions and information are relevant for which rule type is part of the research of this work.





\lstinputlisting[
language=json,
style=json,
caption={Rule root}
label={lst:rule-root}
]{code/rules/rule-root.json}

The value of \code{code}is equivalent to the conclusion of a rule\cite{abdullah2017performance}.
\code{description} is an explanatory text and does not have a specific function in the billing framework.

The \code{condition}object, however, follows a well-defined schema and allows users to specify complex nested conditions.
The framework evaluates this logical tree structure for an input to a boolean value.
If and only if the result is\code{true}, the conclusion of that rule holds.


\subsection{The Condition Rule-Component}\label{subsec:the-condition-component}

This section introduces


\subsubsection{Code Matching Conditions}


\subsection{The Amount Rule-Component}\label{subsec:the-amount-component}





It is a JSON object with multiple fields, each field being either a concrete condition with a defined data type.



