\section{Billing Relevant Information Sources}\label{sec:billing-relevant-information-sources}

Traditionally, billing experts create bills for treatments using their knowledge in this field.
They need to know the conditions which must hold and which make a billing code applicable.
This work introduces a new JSON-based rule language to automate this process.

Before introducing the rule language,
we need to firstly define all sources of information that are relevant for billing.
Its semantics and features were designed based on this knowledge.

Additionally,
it must be assured that the\AVS actually tracks and stores all information relevant for billing as structured data.
For that, the\AVS might require further adjustments and updates.

On one hand, the treatment documentation must contain all medically relevant data,
but now it also needs to cover billing-related information as well.


%The workflow is as follows:
%\begin{enumerate}
%    \item \Me
%    at\AV use their knowledge to define formal specifications for the anamnesis,
%    physical examination and procedures blocks.
%    This is a continuous process where requests and feedback from customers are taken into consideration.
%    \item \Se at\AV create new blocks or update existing ones according to changes in the medical specifications.
%    \item \Be at\AV with medical background,
%    an understanding of the block data models and the rule language implement formal billing rules
%    using the rule language.
%    \item At Back-End start-up the system loads the rule files into the Billing-Service
%which are then available for automated billing code derivation.
%\end{enumerate}

Firstly, we need to specify the concrete use-cases for the rule language.
The initial plan was to implement two separate languages, one for statutory billing and one for private billing.
Section\ref{sec:billing-catalogs} introduces their billing catalogs and presents their differences and characteristics.

However, collaboration with medical experts has revealed further requirements for the billing framework.
This includes additional use-cases for the billing language.
The following list contains all rule types that the framework currently supports:
\begin{itemize}
    \item GOÄ billing rules for deriving GOÄ billing codes from private treatments
    \item EBM billing rules for deriving EBM billing codes from statutory treatments
    \item OPS code rules for deriving procedure codes from treatment documentation.
    OPS codes are relevant information for both GOÄ and EBM codes as well as multiplier justifications.
    \item Multiplier Justification rules for detecting challenging conditions in private treatments
    which serve as justifications for multipliers
    \item Special Flat Rate rules
    for detecting flat rates that can be applied to billing positions in the statutory treatment
\end{itemize}

The initial idea of having separate implementations for each rule type turned out to be not very scalable.
Each rule type has a set of relevant information sources and therefore also sources that can be ignored.
But the key point is that there are many information sources shared by multiple rule types.

For example, the patient age is highly relevant for GOÄ and EBM codes, Multiplier Justifications but not for OPS codes.
Previous allergies are relevant for Multiplier Justifications but not necessarily for GOÄ codes.
This is the reason I decided to use a single rule engine that implements all conditions used by any rule type.
Each rule type defines its own interface and re-uses the rule engine.
Therefore,
we consider each information source illustrated in this section as primarily independent of any rule type or use case.

\subsection{Information Scope}

From a higher level perspective are all billing-relevant information sources can be part of the following domains:
\begin{itemize}
    \item The treatment documentation that contains all medical information of the treatment whose services we want to bill
    \item Patient characteristics that are independent of the current treatment
    \item Information referring to the providing practitioner
    \item Treatment history
\end{itemize}
When a new condition is identified during development,
it needs to be added to the core rule engine
and added to the information input and semantics of the respective rule type.
The condition implementation is part of the core engine and not specific to the rule type.
So if needed, the new condition can be added to other rule types with ease.
It makes sense to keep those rule types as simple as possible,
so only conditions that are explicitly relevant for the rule type are added.

\subsection{Block Contents}
A medical treatment consists of multiple stages of patient care.
Each stage has its specific aims and tools.
Anamnesis,
physical examination and procedures represent different stages of patient care and are highly relevant for billing.

\subsubsection{Anamnesis}
Anamnesis is the first stage of a treatment.
Its purpose is to collect a detailed medical history from the patient\cite{lino2021medical}.
More specifically, it is about understanding the patient's general health conditions,
their lifestyle and previous patients' diagnoses.
If possible, the practitioner reviews available medical records,
interviews the patient directly or using a questionnaire\cite{zhang2011anamnevis}.

\subsubsection{Physical Examination}
Anamnesis is followed by the Physical Examination.
In this practical stage, the practitioner assesses the current patient's conditions,
looking for further information about the patient's issues\cite{seidel2010mosby}.
The practitioner decides based on patient's symptoms which exact examinations are conducted.
Common examinations are checking of vital sings like blood pressure, pulse and temperature.

\paragraph{Procedures}
The procedures stage contains the actual patient specific interventions that \todo

\subsubsection{Block Contents}
During Anamnesis,
Physical Examination and Procedures the practitioner enters medical treatment specific information into the \AVS.
The data is part of the treatment documentation and critical for billing.

The documentations of all three stages are filled out by the practitioner and are structured in sections,
cards and blocks.

For example, inside the physical examination there is the section \mete{abdomen},
which contains the\mete{liver} section.
Blocks are essentially small questionnaires for the practitioner with input fields to be filled out by the practitioner.
They play a huge role in the treatment documentation
and contain all relevant medical information specific to this conducted examination,
procedure anamnesis.
One example for a physical block in the \code{liver}section is\code{LiverPalpation}.
A liver palpation involves investigating the following questions

\begin{itemize}
    \item Is the liver palpable?
    \item Is the liver texture soft or tender?
    \item Is the liver surface smooth or knotty?
    \item Is tenderness present?
    \item Are pulsations present?
    \item Is a hepato jugular reflux present?
\end{itemize}

Each of the more than 100 block contents in the \AVS are represented by a dedicated Hibernate entity classes.
Each attribute of that class stores a questionnaire input entered by the practitioner.
Additionally, for each entity, there is a protobuf message enabling transmission of block content data from one service
to another one.

The protobuf message of \code{LiverPalpationMsg} looks like this:

\lstinputlisting[
    language=protobuf2,
    style=protobuf,
    caption={LiverPalpation protobuf messages}
]{code/proto/liverPalpation.proto}

The exact principle is applied throughout all the procedures and anamnesis as well.
A crucial part of the rule language is not only to check for block content availabilities in the treatment
but also to look inside the blocks and check for conditions on fields.
A GOÄ code might have as a condition that a liver palpation examination is part of the treatment and
the liver turned out to be palpable.
Or an OPS code should only be derived if a \code{ECG}procedure was provided and the
input field\code{telemetricExamination} inside its block has the value\code{true}.
Logical combinations of such single field conditions may also occur.
Expressing such conditions in a well-defined and user-friendly way is an important requirement for the rule language.

\subsection{Block Content Field Types}
A block content can contain information of different input types.
The rule language needs to know how to validate and handle those types.

The most basic field types are boolean flags and numeric inputs.
The \AVuses the following scalar wrapper protobuf messages for them:

\lstinputlisting[
    language=protobuf2,
    style=protobuf,
    caption={LiverPalpation protobuf messages}
]{code/proto/scalarWrapper.proto}

Liver tenderness can either be present or not.
This information is thus entered in a boolean field.
The enumeration field type is used for fields that have a limited number of predefined values.
For example,
the most important enum type
used in a large number of physical examination block contents is \code{AbnormalitiesExaminationResult}.

\lstinputlisting[
    language=protobuf2,
    style=protobuf,
    caption={LiverPalpation protobuf messages}
]{code/proto/enum.proto}



This concept of blocks being a medical questionnaire
holding relevant data in the form of protobuf messages is used across the physical examination,
anamnesis and procedures stage.


Block content conditions are a powerful tool to make codes dependent on concrete medical information from anamnesis,
physical examination and procedures.
They are one of multiple condition types in a rule and are called \code{BlockContentSubRules}


There is another condition
as well that are related to information outside the treatment such as general patient information,
that get

\subsection{EBM billing rules}\label{subsec:ebm-billing-rules}
As described in section\ref{sec:ebm-conditions} the Ebm catalog already contains structured,
treatment content unrelated conditions.
EBM rules therefore do not need to include them.
However, what is left are treatment-specific conditions.
Rules

\subsection{OPS code rules}\label{subsec:ops-code-rules}
OPS codes are identifiers of medical procedures.
They are completely independent of patient information and their previous medical history.
So in order to derive them automatically, they only need the procedure block as an input.

\subsection{GOÄ billing rules}\label{subsec:goa-billing-rules}

\subsection{\MJ rules}\label{subsec:multiplier-justification-rules}

\subsection{Special Flat Rates}\label{subsec:special-flat-rates}

