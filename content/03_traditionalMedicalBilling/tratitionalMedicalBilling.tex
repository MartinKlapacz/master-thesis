\section{Billing Alternatives}\label{sec:billing-alternatives}

\subsection{Momo - Tiplu}


Tiplu's billing product MOMO is a solution for operative medical controlling in hospitals.
It represents a significant advancement in the field of healthcare billing and management,
leveraging the power of artificial intelligence (AI) to streamline and improve the billing process.

I had the opportunity for an expert interview with Jan Willer, Senior Account Manager at Tiplu GmbH.
The following section is based on the knowledge and insights I gained from this interview.

MOMO's primary application is within the inpatient setting, focussing on automating DRG and PEPP coding and providing support for the medical user.
But it is planned to additionally cover outpatient billing starting from the first quarter of 2024.
It provides an overview of the patient cases which includes visits, diagnoses, laboratory results and other billing relevant information.

Momo makes use of two major information sources.
It firstly integrates with all common German clinic information systems, including Orbis, i.s.h.med, NEXUS and Medico.
Momo collects patient and treatment data relevant for billing from special interfaces in existing clinic systems.
Secondly, MOMO uses an AI model to perform semantic text analysis on various medical documents such as physician letters, consultation reports, and visit entries.
Using both approaches, MOMO transforms unstructured information from various sources into a structured data schema.
In practice, this aspect of development is the most labor-intensive part of creating MOMO.

Finally, MOMO uses the collected, structured data to provide automatic billing generation and plausibility checks of existing codes.

Tiplu has developed a large machine learning network in the German-speaking region which encompasses more than 140 clinics and which is trained with data from 12 million patients.
The network uses the concept of federated learning which ensures that patient data never leave the clinic.
Tiplu uses its AI model for automated coding and billing case recognition.

Additionally, Tiplu has developed a rule language for rule-based code derivation.
Maintenance and development of new rules is an ongoing process where employees receive feedback from clinics as well as ensurer to refine and update the rule base.


\subsection{Flobotics}
Flobotics is a company specialized in the automation of repetitive, rule-based tasks within the medical cycle using Robotic Process Automation (RPA).
This also includes billing related processes such as patient registration, bill creation, claim generation and claim submission to insurers.
I had the opportunity to interview Karl Mielnicki, Co-Founder and Lead Architect at Flobotics.
The following information is based on the insights from this expert interview.

However, it's important to note that Flobotic's main focus is not the implementation of complex automated billing systems.
Nonetheless, they have implemented a simplified code generation process for pain treatments, which is a scenario with only a reduced number of available codes.
This was achieved through a basic token search in the texts, exemplifying a straightforward method for automated code derivation.
This experience highlights that billing optimization can also mean the automation of manual processes and that in certain simple scenarios,
coding can be solved in a basic yet effective manner.

\subsection{3M}
%3M SmarteKI is primarily designed for use in both inpatient and outpatient settings, focusing on streamlining and automating various aspects of clinical billing. It aims to simplify the complex processes of DRG (Diagnosis-Related Group) and PEPP (Pauschalierendes Entgeltsystem Psychiatrie und Psychosomatik) coding, thereby assisting healthcare professionals in managing billing tasks more efficiently.
%The system integrates seamlessly with a range of common healthcare information systems used in German clinics, facilitating the collection and processing of patient and treatment data crucial for billing. This includes data from diverse sources such as patient visits, diagnoses, laboratory results, and other relevant information.
%3M SmarteKI employs advanced AI algorithms to conduct semantic analysis of various medical documents, including physician letters and consultation reports. This capability allows it to convert unstructured medical data into a structured format, essential for accurate billing and coding.
%One of the key features of SmarteKI is its ability to automatically generate billing statements and perform plausibility checks on existing codes, ensuring accuracy and compliance with medical billing standards.
%Furthermore, 3M has developed a robust machine learning network, trained on extensive clinical data, to enhance the system's capability in automated coding and billing case recognition. This network adheres to strict data privacy standards, ensuring that sensitive patient information remains secure.
%In addition to its AI-driven functionalities, 3M SmarteKI also includes a rule-based coding system, constantly updated and refined through ongoing feedback from healthcare providers and insurers. This ensures that the system stays current with the latest billing regulations and practices.

\subsection{ID clinical context coding}
ID berlin is a german company that is specialized in creating systems that aid in medical documentation, coding and billing.
ID Clinical Context Coding is a system to improve accuracy of medical coding within clinical documentation.



ID Clinical Context Coding (ID CCC) is a digital coding solution that enhances the medical documentation and billing process in healthcare facilities.
It focuses on analyzing both digital and digitized medical documents to identify billable services.
This is achieved through a linguistic breakdown of the documentation, identifying relevant services. \cite{Diekmann2008}

ID DIACOS is a


diacos \cite{10.1007/978-3-642-82852-2_111}

\subsection{RICO - Dedalus}


\subsection{Common Issues}

The previously presented approaches face common issues.
Namely, automatically deriving billing codes from treatment and patient data requires structured data.
This data typically lives as unstructured information scattered across multiple sources such as existing clinical systems or human-readable documents.

When comparing the previously presented applications, one might note their solutions share the following reoccurring pattern.
\begin{enumerate}
    \item The first step is to collect as much billing-related data as possible and map the unstructured information they find to their own structured data schema.
    \item Given this structured dataset, the systems then apply their own technique to generate billing codes from the data in the schema.
    This own technique is often either AI or rule based or a mixture of both.
\end{enumerate}

This pattern, however, has a set of issues.

---
Before this text I introduced multiple automatic medical billing products, including 3m,tiplu momo, id clinical context coding.
This is what I have right now, i want to express that this pattern has some issues:
- documents and legacy systems are not guaranteed to deliver all the information that the system needs, documents of miss small details
- external systems track data not in a billing-oriented way. They are designed to track data in a patient documentation way but are designed with a different intention.

do you have other ideas? Maybe papers that may help


\begin{enumerate}
    \item
\end{enumerate}
