\section{Billing Alternatives}\label{sec:billing-alternatives}

\subsection{Momo - Tiplu}


Tiplu's billing product MOMO is a solution for operative medical controlling in hospitals.
It represents a significant advancement in the field of healthcare billing and management,
leveraging the power of artificial intelligence (AI) to streamline and improve the billing process.

I had the opportunity for an expert interview with Jan Willer, Senior Account Manager at Tiplu GmbH.
The following section is based on the knowledge and insights I gained from this interview.

MOMO's primary application is within the inpatient setting, focussing on automating DRG and PEPP coding and providing support for the medical user.
But it is planned to additionally cover outpatient billing starting from the first quarter of 2024.
It provides an overview of the patient cases which includes visits, diagnoses, laboratory results and other billing relevant information.

Momo makes use of two major information sources.
It firstly integrates with all common German clinic information systems, including Orbis, i.s.h.med, NEXUS and Medico.
Momo collects patient and treatment data relevant for billing from special interfaces in existing clinic systems.
Secondly, MOMO uses an AI model to perform semantic text analysis on various medical documents such as physician letters, consultation reports, and visit entries.
Using both approaches, MOMO transforms unstructured information from various sources into a structured data schema.
In practice, this aspect of development is the most labor-intensive part of creating MOMO.

Finally, MOMO uses the collected, structured data to provide automatic billing generation and plausibility checks of existing codes.

Tiplu has developed a large machine learning network in the German-speaking region which encompasses more than 140 clinics and which is trained with data from 12 million patients.
The network uses the concept of federated learning which ensures that patient data never leave the clinic.
Tiplu uses its AI model for automated coding and billing case recognition.

Additionally, Tiplu has developed a rule language for rule-based code derivation.
Maintenance and development of new rules is an ongoing process where employees receive feedback from clinics as well as ensurer to refine and update the rule base.


\subsection{Flobotics}
Flobotics is a company specialized in the automation of repetitive, rule-based tasks within the medical cycle using Robotic Process Automation (RPA).
This also includes billing related processes such as patient registration, bill creation, claim generation and claim submission to insurers.
I had the opportunity to interview Karl Mielnicki, Co-Founder and Lead Architect at Flobotics.
The following information is based on the insights from this expert interview.

However, it's important to note that Flobotic's main focus is not the implementation of complex automated billing systems.
Nonetheless, they have implemented a simplified code generation process for pain treatments, which is a scenario with only a reduced number of available codes.
This was achieved through a basic token search in the texts, exemplifying a straightforward method for automated code derivation.
This experience highlights that billing optimization can also mean the automation of manual processes and that in certain simple scenarios,
coding can be solved in a basic yet effective manner.

\subsection{3M}



\subsection{ID clinical context coding}
ID berlin is a german company that is specialized in creating systems that aid in medical documentation, coding and billing.
ID Clinical Context Coding is a system to improve accuracy of medical coding within clinical documentation.
It uses NLP algorithms to analyze medical documents and to derive appropriate codes matching their contents.
This way

ID Clinical Context Coding (ID CCC) is a digital coding solution that enhances the medical documentation and billing process in healthcare facilities.
It focuses on analyzing both digital and digitized medical documents to identify billable services.
This is achieved through a linguistic breakdown of the documentation, identifying relevant services.

diacos \cite{10.1007/978-3-642-82852-2_111}

\subsection{RICO - Dedalus}


\subsection{Common Issues}


%Sehr geehrte Damen und Herren,
%mein Name ist Martin Klapacz. Ich bin Student der Informatik an der TU München und schreibe meine Masterarbeit über die automatisierte Abrechnung von medizinischen Behandlungen.
%
%Teil der Recherche ist es die unterschiedlichen Ansätze bewährter Softwarelösungen auf dem deutschen Markt zu verstehen und aus technischer Sicht zu beleuchten. Dabei würde ich gerne Ihr System in meiner Masterarbeit vorstellen.
%
%Dazu habe ich aber noch einige Fragen, die ich gerne in einem persöhnlichen Telefonat besprechen wollen würde.
%
%Liebe Grüße,
%Martin Klapacz
