\section{Billing server}\label{sec:billing-server}
The billing server is a new microservice added to the microservice architecture of \AV.
It contains the full billing framework and implements the rule language specified in section \addref.

It exposes APIs for basic CRUD operations on billing positions, billing cases, code chains, code chain folder.
\code{applyCodeChainToBillingCase} merges all billing positions from the code chain into the current billing.

\lstinputlisting[
    language=protobuf2,
    style=protobuf,
    caption={BillingService gRPC service}
]{code/proto/architecture/billing-service.proto}


\lstinputlisting[
    language=protobuf2,
    style=protobuf,
    caption={BillingService gRPC service}
]{code/proto/architecture/catalog-service.proto}


\lstinputlisting[
    language=protobuf2,
    style=protobuf,
    caption={BillingService gRPC service}
]{code/proto/architecture/code-chains.proto}

The code derivation part is the most inte

\lstinputlisting[
    language=protobuf2,
    style=protobuf,
    caption={BillingService gRPC service}
]{code/proto/architecture/code-derivation.proto}

It exposes an API for basic CRUD operations on code chains and code chain folder.

\subsubsection{Rule Component}

\subsubsection{Rule Evaluation Input Fetcher}
A \code{RuleEvaluationInputFetcher} is a rule-type specific component responsible for fetching billing-relevant information from other services with the \AV microservice architecture.
They are auto-programmable components that hide fetching details from the rest of the system.
The term \"auto-programmable\" means that they configure themselves automatically upon rule loading.
This works as follows: The \code{Rule} entity class has as described before multiple condition fields.
The condition field names are equal to their corresponding condition keys in the rule language.
For each information source \todo{define information source}



It has access to the mi
