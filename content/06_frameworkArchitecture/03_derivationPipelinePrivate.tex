\section{The Code Derivation Pipeline for Private Treatments}\label{sec:the-code-derivation-pipeline-for-private-treatments}

The derivation pipeline for private treatments is structured into multiple stages.
In this section, we desecribe all derivation stages

\subsection{Initial GOÄ Rule Evaluation}\label{subsec:initial-goä-rule-evalutation}

The first step is to use the \REIF for the GOÄ rule type to fetch the data required for GOÄ rule evaluation.
At this point,
we assume that the fetcher has correctly configured itself at system startup or after rule loading
and fetches the data for all information sources any of the GOÄ rules has subscribed to.
The \REIF returns a \REI that we can further use.

Next, we use the \code{GoaeRuleComponent} to retrieve all GOÄ rules from the database and evaluate them for the previously fetched \REI.
Each rule evaluate happens isolated from any other rule.
For each rule, we accept the value of the \code{code} field for which the \code{condition} field has evaluated to true.

\subsection{Subsequent GOÄ Rule Evaluation Rounds}\label{subsec:susequent-goä-rule-evaluation-rounds}

Section \addref, introduces the \requiredTargetCodes} condition type.
To evaluate this condition, the system needs to know the final result set.
At this initial rule evaluation stage, this information is not available, because the final result set is yet currently being computed.
Therefore, any rule that specifies a set of required target codes will not match as the set of current target codes is empty.
To solve this issue,
I modified the derivation process by introducing subsequent rounds where we evaluate all rules of that rule type.
After each round, we manually update the \code{currentTargetCodes} field in the \REI with the recently computed result.
In these subsequent rounds, we again derive the codes from the previous round including those whose \requiredTargetCodes} condition finally match.
We repeat that process until there are no new changes in the result set.

This way we always have at least two rule evaluation rounds,
but are able to derive codes with arbitrarily long \requiredTargetCodes} dependency chains.

\subsection{Multiplier Justification Derivation Stage}\label{subsec:multiplier-justification-derivation-stage}
This stage is equivalent to the initial rule evaluation in subsection \ref{subsec:initial-goä-rule-evalutation}.
But now we use the \REIF for the multiplier justification rule type to fetch a \REI.
We use this \REI and the \code{MultiplierRuleComponent} to derive applicable multiplier justifications.
Dependencies between multiplier justifications typically do not exist.
We do not need to employ subsequent rule evaluation rounds as described in subsection \ref{subsec:susequent-goä-rule-evaluation-rounds}.

\subsection{Billing Points Computation}\label{subsec:fee-computation}
At this point we have applicable GOÄ code candidates and their quantities and
\begin{itemize}
    \item GOÄ code candidates and their quantities
    \item code-specific multiplier justifications with their suggested multiplier values.
    \item general multiplier justifications with their suggested multiplier values.
\end{itemize}
Next, we match code-specific multiplier justifications with their corresponding GOÄ codes and compute the points for each GOÄ code using formula \ref{eq:fee-function}.
\begin{equation}\label{eq:fee-function}
    \text{points}\left(\text{code}\right) = \text{points}\left(\text{code}\right) \times \text{quantity}\left(\text{code}\right) \times \text{multiplier}
\end{equation}

Later, we can compute the actual fee.
The final user should manually match general multiplier justifications with codes that didn't receive a multiplier.


\subsection{GOÄ Conflict Resolution}\label{subsec:goä-conflict-resolution}
At this point, we have GOÄ billing code candidates and their fees.
However, there is one last problem to solve.


Most of the condition fields in the \code{condition} block are effectively predicate functions we can evaluate for a \REI instance.
If any of those predicate functions inside a \code{condition} block returns false, we reject the conclusion of the rule.
However, this simple concept does not apply to all field types.

Some fields are fundamentally different and can be described as follows.
$c$ is the conclusion code of their rule they are part of.
$R$ is the set of the current evaluation results which are the conclusions of the matching rules.
This is the conceptual schema of these condition fields.
\[
    f: \left( c, R \right) \mapsto \left\{ \left( c, c_i \right) \lvert c \text{ conflicts with } c_i \right\}
\]
They require the conclusion code of their rule, and the current evaluation results to derive a set of conflicts.
A conflict is a tuple of two matching codes.
This condition schema does not determine if a rule matches or not, but rather derives conflicts between matching ones.
It does not tell the system how to resolve them.
The following list enumerates all known code conflict fields.
\begin{enumerate}
    \item The condition field \code{excludedCodes: List<String>} causes conflicts between the conclusion code and all nodes inside the list value of \code{excludedCodes}
    \item The GOÄ catalog specifies code exclusions in a structured way.
    This is equivalent to the \code{excludedCodes} condition field.
    \item The condition field \code{isSingleBillingCondition: Boolean} causes conflicts between the respective code and all other codes in the result set
\end{enumerate}

This has a fundamental impact on the derivation pipeline.
Instead of a set of applicable codes, the result type of the initial and subsequent rule evaluation rounds is a graph.
Each node in this graph is a derived code.
A conflict between two codes is an edge between the corresponding two nodes.
In the PVS seminar, I have learned that resolving these conflicts manually by hand is common practice for medical staff.
Billing experts typically eliminate conflicting nodes while trying to keep the fee loss as small as possible.
It turns out that from a theoretical point of view, this is a NP-Hard graph optimization problem called the Maximum Weight Independent Set problem
(MWIS) \cite{SAKAI2003313}.
Solving such problems automatically is the last step in the derivation pipeline
and required the selection and implementation of an appropriate solver algorithm.
Before formalizing the problem and presenting the selected solver we firstly state
and justify assumptions about the problems that will occur in practice.

\begin{itemize}
    \item The occurring MWIS problems are comparably simple.
    This is inherently the case because in traditional medical billing,
    medical staff resolves those conflict issues by hand.
    The majority of the nodes are unlikely to be involved in conflict at all.
    This means that most nodes in an average result graph are remote nodes the solver can effectively ignore.
    \item Star topology sub-graphs are comparably likely.
    A code
    being the conclusion of a rule with a condition \code{\{ isSingleBillingCondition: true \}} introduces conflicts with all other nodes in the graph.
    The actual implementation in the framework should be able to handle those types of graphs very well.
    \item The number of edge conflicts will be very low.
    This is due to the reason
    that codes that explicetly exclude themselves
    using the \code{excludedTargetCodes} condition type also logically exclude themselves.
    This means that
\end{itemize}

In the following, we formalize the problem and present the solver used in the implementation.

\subsection{Maximum Weight Independent Set problem}\label{subsec:maximum-weight-independent-set-problem}

The Maximum Weight Independent Set problem (MWIS)
is a well-studied and highly important problem in the field of graph optimization.
Its formal definition is as follows \cite{SAKAI2003313}:
Given a graph $G = (V, E)$ with node set $V$ and edge set $E$,
and a weight function $w: V \rightarrow \mathbb{R}$ that assigns a weight to each node,
the Maximum Weight Independent Set problem seeks
to find an independent set $S \subseteq V$ that maximizes the total weight.
The weight function $w$ for our use case is defined in equation \ref{eq:fee-function}.
An independent set is a set of nodes in the graph, no two of which are adjacent.

\begin{equation}
    \begin{aligned}
        & \text{Maximize}
        & & \sum_{v \in S} w(v) \\
        & \text{subject to}
        & & S \subseteq V, \\
        &&& \forall (u, v) \in E, \; u \notin S \text{ or } v \notin S.
    \end{aligned}\label{eq:mwis}
\end{equation}

It is a well-studied optimization problem and plays an important role in the field of graph optimization.
Nogueira, Pinheiro, and Subramanian \cite{Nogueira2018A} introduced a hybrid iterated local search algorithm for MWIS, showing significant efficiency on well-known benchmarks.
Dong et al. \cite{Dong2022A} developed a new local search algorithm based on the greedy randomized adaptive search framework for large-scale instances of MWIS.
they were also able to demonstrate competitive performance on well-known benchmarks.
Those approaches are highly sophisticated algorithm suited to large problem settings.
Given my assumptions and the fact that my use case is comparably simple, I have decided to use one of the heuristic greedy algorithms presented by Sakai, Togasaki, and Yamazaki in 2003 \cite{SAKAI2003313}.
In the following paragraph present the algorithm in a formal way.

\paragraph{GWMin}
GWMin is a simple heuristic solver that works as follows:
It starts up from an empty node set
and iteratively adds the node $v_i$ with the currently optimal relation between weight $w(v_i)$ and degree $d(v_i)$ to the result set.
The node selection can be formulated as the following optimization problem:
Given a graph $G = (V, E)$ with a weight function $W: V \to \mathbb{R}_{\geq 0}$ it selects the next node as follows :
\[
    v' = \max_{v \in V'} \frac{W(v)}{\deg(v) + 1}
\]
where $V'$ is the set of nodes not yet included in the independent set and not adjacent to any node in the current independent set.


\begin{algorithm}
    \caption{GWMIN Algorithm}
    \begin{algorithmic}[1]
        \Procedure{GWMIN}{$G(V, E, W)$}
            \State $I \gets \emptyset$ \Comment{Initialize independent set}
            \State $V' \gets V$ \Comment{Copy of the node set}
            \While{$V' \neq \emptyset$}
                \State $u \gets$ \Call{Selectnode}{$V', G, W$}
                \State $I \gets I \cup \{u\}$ \Comment{Add selected node to independent set}
                \State $V' \gets V' \setminus \{u\}$ \Comment{Remove selected node}
                \State $V' \gets V' \setminus$ Neighbors$(u, G)$ \Comment{Remove neighbors}
            \EndWhile
            \State \textbf{return} $I$
        \EndProcedure
    \end{algorithmic}\label{alg:algorithm-gwmin}
\end{algorithm}

The selection procedure works as follows:

\begin{algorithm}
    \caption{GWMIN Algorithm}
    \begin{algorithmic}[1]
        \Procedure{Selectnode}{$V', G, W$}
            \State $selected \gets$ null
            \State $maxScore \gets -\infty$
            \For{$v \in V'$}
                \State $score \gets W(v) / (\text{Degree}(v, G) + 1)$
                \If{$score > maxScore$}
                    \State $maxScore \gets score$
                    \State $selected \gets v$
                \EndIf
            \EndFor
            \State \textbf{return} $selected$
        \EndProcedure
    \end{algorithmic}\label{alg:algorithm-gwin-selection}
\end{algorithm}


\paragraph{GWMin2}
GWMin2 is a modified version of the GWMin algorithm with another node selection strategy \cite{SAKAI2003313}.
It selects the next node as follows:
\[
    v' = \max_{v \in V'} \frac{W(v)^2}{\sum_{u \in N_G^+(v)} W(u)}
\]
It focuses
on maximizing the ratio of the square of a node's weight $w\left(v\right)^2$ to the sum of the weights of all vertices in its neighborhood $N\left(v\right)$.


Finally, the derivation pipeline returns a set of applicable GOÄ codes and multiplier objects with target codes, multiplier value and justification.
