\section{Quantity Function Specification}\label{sec:quantity-function-specification}


As described in section\ref{sec:numeric-functions} quantity functions are evaluatable subcomponents of rules that return
the number for a billing code.

They are recursively defined evaluation trees that consist of two types of nodes, each node being an own quantity function:
\begin{itemize}
    \item leaf nodes with information how to directly evaluate this node
    \item inner nodes with an operator and a collection of child quantity functions
\end{itemize}

Code snippets\ref{lst:per-n-minutes-quantity-function} and\ref{lst:per-n-hours-quantity-function} illustrate the usage of time related
quantity functions using the fields\code{perNHours} and \code{perNMinutes}.

\lstinputlisting[
    language=json,
    style=json,
    caption={Rule root},
    label={lst:per-n-minutes-quantity-function}
]{code/rules/quantityfunctions/per-n-minutes.json}

\lstinputlisting[
    language=json,
    style=json,
    caption={Rule root},
    label={lst:per-n-hours-quantity-function}
]{code/rules/quantityfunctions/per-n-hours.json}

The framework evaluates this type of quantity function by dividing the procedure's duration by the number of minutes or hours and rounding up the result.
The \code{blockName} field specifies the procedure.

Field related quantity functions work similarly but use other keys.
They refer to a field within a block content using \code{blockName} and \code{fieldPath}.

Additionally, they must specify an \code{evaluationMode} which can be one of the following:

\begin{itemize}
    \item \code{nonNull} evaluates to 1 if anything was entered into the field in the procedure block and 0 if this field is empty.
    \item \code{length} evaluates to the total number of elements in the field.
    This only works for list type fields.
    \item \code{positive} evaluates to 1 if the field has value \code{true}.
    This only works for fields of type \code{BoolW}.
\end{itemize}

Code snippet \ref{lst:field-path-quantity-function} illustrates how to use a field related quantity function.

\lstinputlisting[
    language=json,
    style=json,
    caption={Field Path Quantity Function},
    label={lst:field-path-quantity-function}
]{code/rules/quantityfunctions/field-path.json}
365/351




Code snippet \ref{lst:constant-quantity-function} shows the definition of a constant quantity function.
It always evaluates to the value of the \code{constant} key.
This is particularly useful in more complex scenarios as a node in a deeper evaluation tree.

\lstinputlisting[
    language=json,
    style=json,
    caption={Constant},
    label={lst:constant-quantity-function}
]{code/rules/quantityfunctions/constant.json}

Using the previously introduced examples the user can only create evaluation tree solely consisting of the root node.
This is already sufficient for many time-related use cases.
However, what about more complex scenarios where the final quantity depends on multiple factors?

Quantity functions allow for arbitrarily deep evaluation trees using nested quantity functions.
They have an operator key of type \code{list} which can be one of the following:
\begin{itemize}
    \item \code{sum} adds up the results of all child functions
    \item \code{mul} multiplies the results of all child functions
    \item \code{max} returns the maximum value of all child functions
    \item \code{min} returns the minimum value of all child functions
\end{itemize}

\code{sum} is the most frequently used operator and enables users to count
\code{mul} is very useful in combination with constants to apply a factor to the final result of the quantity function.

Code snipped \ref{lst:combined-quantity-function} shows an example usage of a child quantity functions.
\lstinputlisting[
    language=json,
    style=json,
    caption={Combinations},
    label={lst:combined-quantity-function}
]{code/rules/quantityfunctions/combination.json}


